\shortstory
{Lihan himo}
{}
{Sketch}
{Markus Koskimies}
{http://mkoskim.drivehq.com}
{}
\published
{}
{}

* * *

Istun mustan limusiinin takapenkillä. Tummennettujen lasien läpi en näe, minne olemme menossa. Minua hermostuttaa, pelottaakin. Minusta tuntuu, että olemme olleet matkalla ikuisuuden, aivan liian kauan. Kaivan pienen kallisarvoisen kutsukortin jälleen taskustani. Hypistelen sitä sormissani ja työnnän sen uudelleen taskuun.


Limusiini pysähtyy. Astun ovesta ulos öiselle kapealle kadulle. Ympärillä ovat laitakaupungin levottomat korttelit, edessäni olevassa syvennyksessä on vaatimaton, mitäänsanomaton ovi. Hetken aikaa olen varma, että olen väärässä paikassa. Kaivan kutsukortin uudelleen taskustani ja tarkastan osoitteen. Ei sittenkään, olen saapunut perille.


Limusiinin moottori kehrää pehmeästi tyhjäkäynnillä selkäni takana. Kohennan päällystakkiani ja koputan oveen. Sen avaa mies, joka on pukeutunut huoliteltuun mustaan asuun. Mies tuijottaa minua. En keksi sanottavaa, vaan työnnän kutsukortin hänen käteensä. Mies tarkastelee sitä hetken, antaa sen minulle takaisin ja viittaa minut peremmälle kohteliaasti kumartaen: "Tervetuloa."


Astun hämärään eteiskäytävään. Ovimies osoittaa minulle narikkaa, jonne jätän päällystakkini. Katson itseäni nopeasti peilistä ja oikaisen rusettiani. Minun on pakko kaivaa taskusta esille nenäliina, jolla kuivaan otsalleni kihoavia hikipisaroita. Ryhdistäydyn ja jatkan matkaa eteenpäin hämärässä käytävässä, kohti oviaukkoa, josta kajastaa hämyinen valo. Sieltä kuuluu aterimien kilinää ja vaimeaa keskustelua.


Ruokasalissa on kymmenittäin pöytiä. Sali on valaistu tunnelmallisesti kynttilöillä. Herkullisen ruoan tuoksu täyttää sieraimeni. Pöytien ääressä istuu juhla-asuihin pukeutuneita arvokkaan näköisiä ihmisiä. Tunnen heti kuuluvani joukkoon. He syövät aterioitaan hitaasti, sivistyneesti ja nautiskellen. Lyhyet hiljaisella äänellä käydyt keskustelunpätkät ja aterimien vaimeat kilahdukset vahvistavat harrasta tunnelmaa.


Minut ohjataan pöytään. Eteeni annetaan menu, jossa on tarjolla vain muutama annos.


"Alkupalat, hyvä herra", tarjoilija sanoo kohteliaasti. "Ne ovat ilmaisia."


Listan läpi käyminen ei kestä kauan, mutta päätöksen tekeminen on vaikeaa. Päädyn neidon sormeen, lisukkeinaan kermainen kastike ja metsämarjahyydyke. Jään odottamaan annosta. Suutani kuivaa. Nostan vesilasin huulilleni, haluan, että makuaistini on mahdollisimman terävänä ja valmiina. Hämärässä valaistuksessakin panen merkille, että hyvin pukeutunut asiakaskunta on hieman erikoinen, vaikkei se yllätä minua. Monet näyttävät olevan rampoja, tavalla tai toisella.


Annos tuodaan minulle pienellä hopeisella lautasella, kannen alla. Lisukkeet ovat omissa pienissä eleganteissa astioissaan, ja palan painikkeeksi on tarjolla lasillinen tummaa punaviiniä. Pääosassa on kuitenkin liha. Nostan kannen. Ihoni kihelmöi jännityksestä, pitkä odotus on päättymässä. Naisen sormi höyryää vielä. Vedän huolellisesti hoidetun kynnen varovaisesti irti ja upotan terävän skalpellia muistuttavan veitsen ihon läpi sormen pehmeään lihaan. Suljen silmäni nostaessani haarukan kärjessä luusta erotettua lihaa kohti suutani. Huokaan nautinnollisesti pehmeän ihmislihan hyväillessä makunystyröitäni. Täydellistä. Niin täydellistä.






* * *

Olin jo aikaa sitten jättänyt tavanomaiset "eksoottiset" ruokalajit - kuten japanilainen pallokala ja kiinalainen Pekingin ankka - sille sivistymättömälle rahvaalle, jonka kieli oli kiiltolakattua mahonkia kuin toimistopöytä. En koe tämän loukkaavan pöytää, sillä se selvästi nauttii puhdistusaineiden jämerästä kosketuksesta. Minun hienostunut makuni kaipasi paljon enemmän tullakseen tyydytetyksi.


Olin tunkeutunut kulinarismin hämärille reuna-alueille, ensin nauttimaan harvinaisten ja uhanalaisten eläinten lihaa. Se ei ollut riittänyt minulle, olin osallistunut iltamiin, joissa söimme lajeja sukupuuttoon, nautiskellen niiden viimeisten edustajien lihasta - makunautinnoista, joita kukaan ei enää myöhemmin voinut saada. Kaivauduin syvemmälle ainutkertaisten makuelämysten maailmaan, yhä syvemmälle kohti sen ytimiä.


Olin kuullut huhuja ravintolasta, joka tarjoili asiakkailleen ihmislihaa. Eikä suinkaan mitään köyhyydessä kasvanutta, omissa ulosteissaan maannutta, kulkutautien runtelemaa, viikkoja pakastimessa lojunutta sitkeää ja jänteikästä orjakansaa, jota sai ostettua taskurahoilla maailman kaikilta levottomilta seuduilta. Ei, vaan rahassa ja yltäkylläisyydessä hitaasti ja huolella marinoitunutta, taatusti tuoretta, rasvaista ja ainutlaatuista eliittilihaa.


Hyvä ystäväni, arvoisa lordi d'Argenson, taipui lopulta lähettämään minusta suosituksen. Joitain viikkoja myöhemmin, maksettuani pienen omaisuuden annetulle sveitsiläiselle tilille, malttamattoman odottelun jälkeen postiluukustani kolahti kirjekuori, jonka sisällä oli kutsukortti. Minun ei tarvinnut harkita kauan soittaakseni kortissa olleeseen numeroon, saadakseni arvolleni soveltuvan kuljetuksen ravintolaan.






* * *

Muutama päivä myöhemmin ensimmäisen aterioinnin jälkeen pyörin levottomana kattohuoneistossani. Mikään muu ruoka ei maita, on vain yksi ruokalaji, jota makunystyräni himoitsevat. Pyörittelen sormissani korttia. Annan periksi väistämättömän edessä. Minun on astuttava askel eteenpäin.


Jätettyäni päällystakkini ravintolan eteispalveluun minua ei ohjata ruokasaliin, vaan toimistoksi kalustettuun huoneeseen. Pöydän toisella puolella istuu ravintolan omistaja. Hän kehottaa minua istumaan pöydän edessä olevalle tuolille.


"Meidän on keskusteltava maksusta", hän toteaa yksinkertaisesti. Nyökkään. Kuten lordi d'Argenson oli jo kertonut minulle, kutsukortti maksettiin rahalla, kunhan sai luotettavan suosittelijan. Mutta ruokaa ei rahalla saanut, lordi oli sanonut esitellessään käsiproteesiaan. Pyörittelen sormiani istuessani tuolissa. Epäröin, vaikka olen jo tehnyt päätökseni.


Nostan vasemman käteni pikkusormen ylös.


Ravintoloitsija nyökkää hyväksyvästi. Minulta otetaan verikoe, jolla varmistetaan, että kehoni on kauttaaltaan puhdas ylimääräisistä kemikaaleista. Kieltäydyn tarjotusta konjakkiryypystä, en halua turruttaa makuaistiani.


Hieman myöhemmin istun jälleen ruokasalin puolella. Vasemman käden pikkusormen tilalla on side. Kättäni jomottava kipu saa silmäni kostumaan, mutta työnnän sen taka-alalle niin hyvin kuin kykenen. Haluan nauttia annoksestani.


Odottaessani annostani katselen asiakaskuntaa, joka vilkuilee toisiaan arvioiden. Kuvittelen mielessäni, miltä kukin pöytien äärellä istuvista maistuisi lautasellani. Vilkaisen kädetöntä miestä. Eleganttiin iltapukuun pukeutunut avustaja leikkaa hienostuneesti tämän edessä olevaa annosta ja syöttää sitä miehelle. Herkullista.


Minulle tuodaan hopealautanen, jonka kannen alla on pieni lajitelma Lacostessa haudutettuja varpaita, kuin pikkuruisia lihapullia, joiden sisällä on valkoinen siemen. Silmä silmästä, varvas sormesta - niissä on yhtä paljon lihaa kuin pikkusormessani. Erottelen lihan luista varovasti ja valmistaudun jälleen kohtaamaan makunautinnon täyttymyksen.






* * *

Muutamaa sormea ja varvasta myöhemmin päätän ryhdistäytyä. Makunautinto jää puolittaiseksi kivuliaan amputoinnin jäljiltä. Haluan nauttia annoksistani ilman kipua, kärsimystä ja kyyneleitä.


Tapaan ravintoloitsijan jälleen tämän toimistossa. Keskustelemme tovin vaihtoehdoista.


"Ymmärräthän", mies sanoo, "ettemme voi käyttää nukutus- tai puudutusaineita, ne pilaisivat lihan. Mutta suosittelen, että nautit alle tuhdisti väkevää viskiä tai konjakkia, se helpottaa hiukan. Kärsimys ja konjakki - ne antavat lihalle sen hienostuneen aromin."


Tällä kertaa päätän noudattaa ohjetta ja tartun minulle tarjottuun pulloon. "Talo tarjoaa", ravintoloitsija sanoo. Kun minut myöhemmin sidotaan ravintolan kellarissa penkkiin, olen niin humalassa, etten edes näe eteeni. Vasemman reiteni ympärille kiristetty vyö tuntuu vain etäisenä aivoissani. Siitä huolimatta saha, joka pureutuu reiteni lihaan, saa minut karjumaan tuskasta.






* * *

Palaan ravintolaan vasta viikkoja myöhemmin, toivuttuani riittävästi voidakseni nauttia pöydän antimista. Autonkuljettaja avustaa minut kainalosauvoineni eteiseen, josta minut ohjataan pöytään. Asetan kainalosauvan nojaamaan tuoliani vasten ja saan käteeni menun.


Tällä kertaa minulle tarjotulla listalla on muutakin kuin yksinkertaisia napostelupaloja. Vesi herahtaa kielelleni. Tahtoisin tilata kokonaisen reiden, mutta maltan mieleni. Tahdon nauttia, en mässäillä. Sormia ja varpaita olen kuitenkin maistellut jo tarpeeksi, nyt haluan jotain erikoisempaa. Tällä kertaa listalta löytyisi sydäntä ja kieltä, taatusti tuoreista ja laadukkaista raaka-aineista valmistettuna, kuten aina. Päätän ottaa palan niskaa ja maistaa jälkiruoaksi siivutettua silmää.


Odotellessani _Entrecote_-pihviäni katselen, kuinka muutaman pöydän päässä olevalle seurueelle kannetaan eteen _Cotes levees_, kokonainen grillattu rintakehä avattuna. Valkoiseksi pesty rintalasta katkaistuine kylkiluineen kruunaa juhla-aterian kuin katedraali. Kuin katselisi pienoismallia rannalle ajautuneesta, puhtaaksi kalutusta valaasta. Jostain hyvin, hyvin harvinaisesta valaasta. Seurueen jäsenet alkavat veitsineen kuoria lihaa kylkiluiden ympäriltä. Minun on pakko kääntää katseeni pois, jottei kuola alkaisi valua suupielistäni. Nieleskelen, pyyhin suupieleni servettiin ja suljen silmäni, mutta se ei auta. Lihan täyteläinen, makea tuoksu täyttää sieraimeni.






* * *

Seuraavalla kerralla pöydässä istuessani ja ruokalistaa lukiessani huomioni kiinnittyy yksin istuvaan naiseen. Kutsun tarjoilijan paikalle.


"Voisitteko käydä kysymässä neidiltä, tahtoisiko hän liittyä seuraani?"


Tarjoilija poistuu naisen luokse, kumartuu hänen puoleensa ja esittelee asiansa. Nainen katsoo minua ja sitten tarjoilijaa, joka lähtee työntämään tätä rullatuolissa kohti pöytääni. Esiteltyämme toisemme - hän on kreivitär Helena von Gassner - syvennymme ruokalistaan. En malta olla katsomatta hänen muotojaan. Kuorin mielessäni iltapuvun hänen yltään, ja sitten upotan ajatukseni hänen ihonsa alle. Kuinka mehukkaan aterian hänen jäntevästä kaulastaan saisikaan. Tai olkapäistä. Vesi herahtaa kielelle ajatellessani hänen vasenta rintaa lautasellani.


Syömme toisiamme silmillämme. Nälkäiset katseemme kohtaavat.


"On hyvä tuntea, mitä syö", Helena sanoo ja minä nyökkään. Hymyilemme. On miellyttävää tavata sielunkumppani.


Syömme hiljaisuuden vallitessa. Mietin mielessäni, kumpi meistä antautuu intohimolleen aiemmin, kumpi saa toisesta parhaat palat lautaselleen.


Syötyämme kreivitär hivuttautuu puoleeni ja kuiskaa: "Näetkö tuon kädettömän ja jalattoman herran tuolla? Tule keskiviikkona paikalle, varttia vaille puolen yön, ja voimme tilata varsin maukasta kylkipaistia. Luota minuun", hän iskee minulle silmää.


Ja toden totta, keskiviikkoyönä ruokalistalla on varsin kattava satsaus ulko- ja sisäfilettä sekä erilaisia sisäelimiä, eikä kädetöntä miestä näy missään.






* * *

Vasemmasta jalasta saamani luotto hupenee miltei liiankin vauhdikkaasti nauttiessani ravintolan antimia. Päivä päivältä lähestyy tilanne, jolloin minun on hankittava sitä lisää. Odotan sitä pelonsekaisin tuntein. Tutkiskelen itseäni ja teen päätöksen. Menköön toinenkin jalka. Tahdon pitää molemmat käteni, jotta voin nauttia annokseni sivistyneesti veitsen ja haarukan kera.


En saa millään tarpeekseni ihmislihasta. Huomaan nauttivani aterioista usein Helenan seurassa. Hän kertoo olevansa ravintolan asiakkaita jo kolmannessa polvessa. Vähäiset keskustelumme koskettelevat ruokaa.


"Nuorempi sukupolvi turvautuu aivan liian usein aborttiin", Helena sanoo haikeana. "Siitä on ikuisuus kun viimeksi saimme maistaa pehmeää, suussa sulavaa vauvanlihaa. Toista se oli isoisäni aikaan."


Kuukauden kuluttua istun rullatuolissa toimistohuoneessa ja luovun vasemmasta kädestäni. En ehkä voi sen jälkeen enää syödä sivistyneesti, mutta voin kuitenkin syödä.


Sen jälkeen minulle tulee vaikeuksia päättää, mitä tarjoan vastineeksi lihasta. Saan hetkeksi tyydytystä luopuessani korvalehdistäni. Luovun ensin toisesta kiveksestä, sitten toisesta. Niillä saan vain pieniä härnääviä makupaloja, kuten isoäidin lihamakkaraa omassa peräsuolessa, ja minä kaipaisin kunnon pihviä. Sen jälkeen luovun peniksestäni, ja katselen vastapäätä, kuinka naistuttavani syö sen suihinsa. Naisen povi on nyt toispuoleinen, minä nautiskelen lautasella palaa hänen vasemman rintansa pehmeistä ja täyteläisistä kudoksista. Minä värähdän hiukan, kun Helena halkaisee lautasellaan olevan peniksen pituussuunnassa. Huomaan hänen värähtävän viiltäessäni omalla lautasellani olevan nännin kahtia.


Ei kestä kauan, kun istun jälleen toimistohuoneessa, mutta tällä kertaa neuvottomana. Oikea käsi minulla vielä olisi, mutta se antaisi vain hetkellisen rauhan piinatulle sielulleni. Epätoivoissani harkitsen jo luopuvani kielestä, mutta makuaistista luopuminen olisi kuolemaakin pahempi kohtalo.


Tarjoan ravintoloitsijalle koko omaisuuttani, mutta hän pudistaa päätään. "Saamme kutsukorttimaksuista tarpeeksi rahaa toiminnan pyörittämiseen. Tarvitsemme laadukkaita raaka-aineita, emme rahaa, ymmärräthän?"


Nyökkään alakuloisena.


"Onko teillä puolistoa?" ravintoloitsija kysyy. "Tai lapsia? Hyväksymme lähipiirin kehon osia maksuksi, mutta emme ulkopuolisia."


Pudistan päätäni. Valitettavasti minulla ei ole ollut aikaa moiselle, tyydyttäessäni kulinaristisia intohimojani. Suostuisikohan Helena menemään naimisiin kanssani? Tai suostuisiko hän tekemään lapsen kanssani? Yhdeksän kärsimyksen kuukautta ilman ravintolan antimia ei kuulosta minusta houkuttelevalta ratkaisulta.


"Meillä on kyllä eräs tarjous kaltaisillenne vakioasiakkaille", ravintoloitsija sanoo. "Se on La Cene, viimeinen ateria. Saatte syödä painonne verran ravintolan antimia, ja sen jälkeen valmistamme aterian sinusta itsestäsi. Harkitkaa toki tarjousta kaikessa rauhassa."


Minä harkitsen, mutta en pysty pitämään itseäni erossa lihasta. Palaan ravintolaan yhä uudelleen. Koetan venyttää vääjäämättömästi lähestyvää hetkeä nauttimalla yksittäisiä herkkupaloja. Viipaloitu kives, maksapihvi, munuaiskeitto.


Lopulta annan täydellisesti periksi himolleni. Tilaan kokonaisen seitsemän ruokalajin illallisen. Aloitan "täyskädellä", viiden sormen lajitelmalla, jotka imeskelen puhtaaksi luita myöten. Herkuttelen säästeliään herrasmiehen reidellä, ladyn hyvin istutulla pakaralla, neidon kärkevällä kielenpuolikkaalla ja mehevillä kylkipaloilla. Nostan eteeni puhtaan valkoiseksi pestyn kallonpuolikkaan, johon on upotettu hienopiirteiset hopeiset jalat. Upotan lusikkani sen sisältämään aivohyytelöön ja annan sen sulaa suussani. Luulen vatsani halkeavan, mutta yhä vain tahdon lisää, kunnes tarjoilija kieltäytyy kohteliaasti. Kun hän aterian päätteeksi alkaa työntää pyörätuoliani kohti keittiötä, asiakaskunnan katsellessa nälkäisenä, ymmärrän, mistä tarjous oli saanut nimensä.







