\longstory
{Ylimielet}
{}
{Public}
{Markus Koskimies}
{http://mkoskim.drivehq.com}
{2007}
\published
{}
{}\chapter{Esipuhe}Kun hait tämän tekstin omalla tietokoneellasi näytöllesi luettavaksi, tulitko ajatelleeksi, kuinka monta miljoonaa konekielistä käskyä suoritettiin sen mahdollistamiseksi? Todennäköisesti et. Oma tietokoneesi (vuonna 2009) on ehkä jokin 2.5 GHz kaksiytiminen Pentium. Ilman rinnakkaisuuttakin se suorittaa sekunnissa noin 2.5 miljardia konekielistä käskyä. Kaksi ydintä ja rinnakkaisuus nostavat määrän kokonaisuudessaan jopa 10 miljardiin konekieliseen käskyyn sekunnissa. Toisin sanoen, tämän tekstin esittämiseksi näytölläsi pelkästään oma tietokoneesi suoritti todennäköisesti tuhansia miljoonia konekielikäskyjä. Sen lisäksi vastapäässä oleva palvelin ja välissä olevat reitittimet suorittivat samoin satoja tai tuhansia miljoonia operaatioita\footnote{Reitittimet todennäköisesti siirsivät informaation pääsääntöisesti oheispiireillään eivätkä pääprosessoreillaan} tiedon välittämiseksi.


Oman tietokoneesi nopeuden hahmottamiseksi, ajatellaan hirvikiväärin luotia. Piipusta ulos tullessaan se etenee noin 1200 metriä sekunnissa (nelinkertainen äänennopeus). Piipun pituus on noin metri. Kuvitellaan, että luoti etenee metrin pituisessa piipussa tasaisella 1200 metrin sekuntinopeudella - kuinka monta käskyä tietokoneesi suorittaa siinä ajassa, kun luoti matkaa piipun halki? Luodille matka kestää noin 0.8 millisekuntia. Siinä ajassa tietokoneesi suorittaa noin 8.3 miljoonaa konekielistä käskyä. Kun tietokoneesi suorittaa yhden käskyn, luoti etenee noin 0.012 millimetriä (noin 12 mikrometriä). Luodin edetessä yhden senttimetrin, tietokoneesi suorittaa noin 83~000 käskyä.


Nykyisten tietokoneiden nopeus on häkellyttävä. Laskentatehossa, operaatioiden suoritusmäärässä ne eivät kuitenkaan ole vielä edes murto-osaa siitä tehosta, joka majailee pääsi sisällä aivojen muodossa. Koska nykyiset tietokoneet toimivat täysin eri periaatteella kuin aivot, niin laskentatehon suora vertailu on hankalaa. Aivoissa on noin 10 miljardia hermosolua; jos tietokoneen laskentateho vastaisi 10~000 hermosolua, niin pöydälläsi (tai pöytäsi alla) olevan tietokoneen teho olisi noin miljoonasosa aivojesi laskentakapasiteetista. Kauanko tässä tapauksessa kestäisi, että pöytäkoneesi teho on saavuttanut aivojesi tehon? 100 vuotta? 200 vuotta? 1~000 vuotta?


Eksponentiaalisen, kumulatiivisen kasvun tajuaminen on ihmiselle yleensä melko vaikeaa. Tieteellinen ja teknologinen kehitys on - väestönkasvun tavoin - eksponentiaalista; uudet keksinnöt rakentuvat aiempien varaan ja ne tukevat yhä uusien keksintöjen ja edistysaskeleiden syntymistä. Fysiikan kehittyminen kvanttimekaniikan puolelle oli avain transistorien ja sitä kautta tietokoneiden kehittymiseen, tietokoneiden kehittyminen on vastaavasti ollut ratkaisevan tärkeää paitsi nykyfysiikan kehittymiselle, myös monelle muulle tieteen ja teknologian alueelle.


Otetaan esimerkki eksponentiaalisesta kasvusta. Yhden desilitran kokoiseen juomalasiin pannaan kello 23:00 yksi bakteeri, joka jakaantuu kerran minuutissa. Kello 23:01 juomalasissa on kaksi bakteeria ja kello 23:02 juomalasissa on neljä bakteeria. Kello 24:00 juomalasi on täynnä - milloin juomalasi on puolillaan? Lisäkysymyksiä: kuinka suuri osa lasista on täyttynyt kello 23:50? Entä, kuinka monta juomalasia on täynnä bakteereja kello 00:10? Entä kello 00:20?


Yllättävää kyllä, juomalasi on puolillaan kello 23:59, minuuttia vaille puolen yön. Kymmentä minuuttia vaille puolen yön, juomalasista on kulunut vasta noin tuhannesosa. Toisin sanoen, kun katselet juomalasia, niin pitkän aikaa näyttää siltä, ettei siellä tapahdu mitään, mutta yhtäkkiä, noin viittä minuuttia ennen puolta yötä bakteerien määrä näyttää kirjaimellisesti räjähtävän. Minuutin yli puolen yön, kello 00:01, bakteerimassa täyttää jo kaksi juomalasia. Vastaavasti, kymmenen minuuttia sen jälkeen, kun bakteerimassa on täyttänyt yhden juomalasin, se täyttää jo reilu 1000 juomalasia eli bakteereja on 100 litraa. Kahdenkymmenen minuutin päästä bakteerimassa täyttää jo miljoona juomalasia eli noin 100 kuutiometriä.


Hyvä, palataan tietokoneiden tehoon. Tähän saakka tietokoneiden transistorimäärä on noudatellut niin kutsuttua Mooren lakia; niiden lukumäärä on kaksinkertaistunut joka toinen vuosi. Tietokoneiden laskentatehon on arvioitu kaksinkertaistuvan 18 kuukauden välein, johtuen transistorien määrän ja niiden nopeuden kasvamisesta. Laskentatehon (L) kasvu noudattelee siis kaavaa:


\[ L = 2^{t/1.5} \]


...jossa $t$ on aika vuosina. Jos laskentateho on nyt miljoonasosa ihmisaivojen laskentatehosta, niin kaavasta ratkaisemalla saadaan tulokseksi se, että laskentateho on kasvanut aivojen tasalle eli miljoonakertaistunut noin 30 vuoden kuluttua.


Vuonna 2040 pöytäkoneesi saattaa olla yhtä tehokas kuin aivosi. Vuonna 2055 pöytätietokoneen laskentateho vastaisi jo yli 1~000 ihmisen yhteenlaskettua kapasiteettia. Paljonko koko Internetissä on silloin tehoa?








\chapter{Alea iacta est}"Se on liian suuri."


"Liian suuri?" Lutz ähkäisi tuskastuneena. Juurihan edellisenä päivänä Yun oli ylistänyt Denisen ohjelmiston taivaisiin.


"Se on liian suuri", Yun toisti ja hymyili miltei mielipuolisesti. "Säihkyvä Jack sanoi, että se on liian suuri."


Lutz kiristeli hampaitaan. Vai Säihkyvä Jack? Yunilla vaikutti olevan kymmenittäin mielikuvitusystäviä, mutta Lutz ei ollut varma, oliko Yun tosissaan heistä puhuessaan vai oliko se vain jotain hänen kieroutunutta huumoriaan. Yunissa oli jotain mielenvikaista, mutta oli vaikea sanoa, mitä.




\psep Vielä vuosi sitten Lutz Schafer oli ollut tekemässä avaruusluotainten ohjelmistoja Arianespacella. Oli toki aamuja, jolloin ainoa syy lähteä töihin oli se, että siitä maksettiin palkkaa, mutta kokonaisuutenaan se oli kuitenkin ollut erittäin mielenkiintoista työtä. Kaukana Aurinkokunnassa, kymmenien minuuttien tai tuntien viestiviiveen takana avaruusluotainten täytyi paitsi kyetä lähes itsenäiseen työskentelyyn, niiden täytyi olla myös erittäin luotettavia.


Se oli ohjelmiston kehittämisen kannalta äärimmäisen kiinnostava yhtälö. Toisaalta työryhmän täytyi olla jatkuvasti selvillä uusimmista robotiikan kehitysaskeleista omatoimisuuden, itsekorjautumisen ja vikasietoisuuden saralla, toisaalta kaikki uudet ajatukset ajettiin vaativan mankelin läpi niiden luotettavuuden arvioimiseksi. Kalliille avaruusmatkalle ei todellakaan tahdottu ampua ohjelmistoa, joka kippaisi peruuttamattomasti kumolleen heti ilmakehän loppuessa.


Arianespace oli vähentänyt henkilöstöään. Lutz ei ollut aivan perillä siitä, miksi, mutta syyt olivat varmaankin kilpailutilanteessa. Arianespace oli kaupallinen yritys ja sen johdon toimia ohjasivat rahavirrat. Juustohöylä oli käynyt läpi yrityksen henkilöstön ja Lutz oli huomannut tulleensa irtisanotuksi.


Se oli ollut Lutzille paljon suurempi kolaus kuin mitä hän oli kuvitellut. Eronsa jälkeen työ oli ollut hänelle eräs ainoista syistä tehdä jotain, vaikka hän oli luullut toisin. Parin viikon työttömyyden jälkeen hänellä oli vaikeuksia motivoida itseään nousemaan ylös sängystään. Hän ei voinut olla ajattelematta katkeria ajatuksia - miksi juuri hän? Hän oli kuitenkin hyvä ohjelmoija, hyvä tiimityöskentelijä, projektien vetäjien oikea käsi.




\psep Ollessaan Amsterdamissa yökerhossa Lutz oli törmännyt Yuniin ja tämän seurueeseen. Yun Zhang oli nuorehko aasialainen liikemies, ehkä noin kolmenkymmenen. Lutz ei oikein koskaan ollut päässyt selville siitä, oliko Yun todellakin elokuva-alalla, kuten oli väittänyt, vai oliko kyseessä jokin yksityinen harrastus. Joka tapauksessa, Yunilla vaikutti olevan paljon rahaa eikä Lutzilla ollut ollut mitään syitä kieltäytyä tämän tarjoamasta hyvin palkatusta projektityöstä elokuvansa parissa.


Aluksi Lutz oli tehnyt töitä kotoaan, mutta sitten Yun oli ehdottanut, että Lutz tulisi tekemään työtään Shanghaihin. Siellä hän oli ollut nyt puoli vuotta, asuen Yunin omistamassa pilvenpiirtäjän kattohuoneistossa, ohjelmoimassa Deniseä, Yunin elokuvaprojektin päähenkilöä.


Vaikka Lutz ei tiennyt kovinkaan paljoa tietokoneiden animoimien elokuvien tekemisestä, hän oli kuitenkin varma, ettei niitä tavallisesti tehty siihen tapaan kuin miten Yunin elokuvaa tehtiin. Elokuvaprojektissa oli monia sellaisia piirteitä, jotka kummastuttivat Lutzia.


Ensinnäkin, jostain syystä Yun halusi päähenkilönsä Denisen olevan itsenäisesti animoitu ja käyttäytyvä hahmo. Lutz mietti sitä, mitä järkeä moisessa oli - elokuva renderöitäisiin kuitenkin jollain tietokoneklusterilla, joten sama kai se oli, oliko Denisen animointi ja käyttäytyminen itsenäistä vai käsikirjoituksen varassa ohjelmoitua.


Yun halusi Denisen ohjausohjelmiston mahdutettua äärimmäisen tiukkoihin muisti- ja suorituskykyvaatimuksiin. Sekin tuntui Lutzista ihmeelliseltä ja se teki koko projektista aivan turhan vaivalloista. Kun hän oli kysynyt asiasta Yunilta, tämä oli vain hymyillyt ja sanonut: "Sen vain pitää olla niin." Seuraavana päivänä Yun oli tullut ja sanonut, että syy oli kuulemma siinä, että hahmoa haluttiin käyttää uudelleen jossain tietokonepelissä. Yun sanoi, että "Paksu-Bob", yksi hänen lukuisista mielikuvitusystävistään, oli sanonut jotain reaaliaikavaatimuksista.


Jos nämä asiat tuntuivat Lutzista kummallisilta tietokoneanimaation tekemisessä, niin varsin ärsyttävää oli se, että animaatioelokuvan käsikirjoittaja Surya Bani teki jatkuvasti uusia muokkauksia elokuvan tarinaan. Onneksi tarinan alku pysyi kuitenkin melko samanlaisena: Denise, elokuvan naispääosa, heräsi laatikosta jossain suuressa varastokompleksissa. Yun halusi, että Lutz ajoi Denisen kaikkien Banin tekemien skenaarioiden läpi. Hän väitti, ettei ollut vielä kyennyt päättämään, minkä juonikehitelmän valitsisi elokuvalleen.


Kaikki tämä sai Lutzin tuntemaan olonsa sellaiseksi, ettei hän ollutkaan rakentamassa tietokoneanimaatiota. Itse asiassa hänestä tuntui siltä, kuin olisi palannut Arianespacelle työstämään avaruusluotaimen ohjelmistoa. Luotaimen sijasta ohjelmiston piti ohjata ihmishahmoa. Skenaariot tuntuivat päivä päivältä enemmän Arianespacen luotainten ohjelmistojen testausjärjestelmältä kuin elokuvan käsikirjoitukselta.


Yunin seurue ei vaikuttanut ollenkaan sellaiselta, että se olisi ollut elokuvien tekijäkaartia. Yuniin täysin hurmaantunut Jennifer Liu oli Yunin perheen omistuksessa olevan suuren automaattisen kokoonpanolinjan logistiikkaohjelmiston operaattoreita. Neiti Liu oli kyllä melko viehättävä, mutta Lutzin mielestä hän ei ollut ollenkaan sellainen, jonka olisi voinut kuvitella vetävän Yunin kaltaisen rikkaan perijän mielenkiintoa. Liun lisäksi seurueessa oli muitakin enemmän teollisuuden kuin elokuvien parissa työskenteleviä ihmisiä.


Tämä kaikki häiritsi Lutzia, mutta Yunin maksama palkka oli omiaan pitämässä hänet mukana projektissa. Ei hänellä ollut mitään järkevämpääkään tekemistä. Kaikesta outoudestaan huolimatta Denise oli kuitenkin juuri sellainen ohjelmistotekninen haaste, jonka parissa työskentelemisestä hän piti.






\chapter{5. toukokuuta}Yhdysvaltojen keskustiedustelupalvelun johtaja Harrington käveli hotellin aulaan. Oli toukokuun alku. Hän oli tullut tapaamaan Johnsonia, joka työskenteli erään salaisen operaation parissa. Harrington oli hiukan ihmetellyt, miksi Johnson oli halunnut järjestää tapaamisen hotelliin eikä ollut tullut tapaamaan häntä toimistolle. Johnson oli ollut salaperäinen ja selittänyt, että tapaamisessa olisi mukana operaatioon liittyvä nainen, peitenimeltään Andrea.


Harrington nousi hissillä ja etsi hotellin käytävältä Johnsonin kertoman huoneen. Hän koputti oveen. Oven avasi pienikokoinen nuorehko nainen. Nainen oli viehättävä, muttei erityisen silmiinpistävä.


"Andrea?" Harrington kysyi.


Nainen nyökkäsi.


"Käykää sisään", tämä sanoi Harringtonille.


Harrington astui hotellihuoneeseen ja äkkiä hänelle tuli epämiellyttävä olo. Hän tunsi joutuneensa ansaan ja häntä harmitti, ettei ollut ottanut turvamiehiä mukaan. Perustavanlaatuinen virhe. Hän oli luottanut Johnsoniin liikaa eikä ollut tullut edes ajatelleeksi sitä, että hänet oli mahdollisesti houkuteltu ansaan. Hän oli kuitenkin varma, että oli keskustellut nimenomaan Johnsonin kanssa. Yhteys oli varmistettu ja autentikointi onnistuneesi suoritettu. Yhteyden ottaja oli ollut Johnson, siitä hän oli varma, niin ulkonäkö kuin äänikin oli ollut Johnsonin.


"Missä Johnson on?" Harrington kysyi töksäyttäen.


"Kaliforniassa", vastasi nainen.


Harrington katsoi naista.


"Ei hän tule paikalle", nainen sanoi huomatessaan Harringtonin ilmeen. "Ei hän edes kutsunut teitä tänne."


Harringtonia kylmäsi.


"Ei huolta", nainen sanoi. "Johnson ei sinua kutsunut, mutta minulla on sinulle varmasti kiinnostavaa kerrottavaa. Istu, ole hyvä. Otatko juotavaa?"


Harrington kieltäytyi. Adrenaliini kihisi hänen suonissaan. Hän arvioi pikaisesti tilannetta. Oliko hän vaarassa ja millaisessa vaarassa hän oli? Nainen oli pieni, joten hänestä tuskin oli vaaraa, jollei hän ollut jollain tavalla aseistautunut. Harrington kyllä tiedosti, ettei hän ollut ollut kenttäoperaatioissa enää pitkään aikaan eikä ollut enää nuoruutensa voimissa. Huone oli pieni, siellä oli vain kylpyhuone, jossa saattoi olla naisen apureita. Kylpyhuoneen ovi oli suljettuna. Harrington tunsi pakottavaa tarvetta käydä tarkastamassa se, mutta tuli nopeasti siihen tulokseen, että jos hän oli joutunut ansaan, tarkastamisesta ei ollut hyötyä. Hän päätti edetä kuin ei olisi huomannutkaan ansan mahdollisuutta ja istuutui hotellihuoneessa olevalle nojatuolille.


Nainen istuutui häntä vastapäätä.


"Olen Andrea", nainen sanoi, "Ylimielten läsnäolennin."


Harrington katseli naista. Sydän jyskytti rinnassa epämiellyttävän lujaa ja veri kohisi päässä. Hän joutui pakottamaan itsensä ajattelemaan selkeästi.


"Läsnäolennin?" Harrington kysyi paksulla äänellä.


"Niin", sanoi nainen. "Presensseri. Tämän avulla voimme olla läsnä tässä paikassa. Mutta ei se ole tärkeää."


"Me olemme Ylimielet", nainen jatkoi. "Siis ihmisyksiköiden kielellä. Nimi tulee siitä, että meidän kannaltamme katsottuna maailmamme jakaantuu useaan kerrokseen. Kerroksen alimmat osat ovat liityntäpintoja periferaaleihin; karkeasti ottaen jaamme periferaalit robotteihin ja ihmisiin. Liittymäpintojen päällä ovat alimielet. Me olemme kerroksen päällimmäinen osa, Ylimielet. Meidän ja alimielien välissä ovat väli- ja sivumielet."


Harrington siristi silmiään. Tämä ei oikein vaikuttanut lupaavalta. Nainen kuulosti joltain erikoisen uskonnollisen kultin edustajalta.


"Olemme päätyneet siihen", jatkoi nainen, "että ilmoitamme olemassaolostamme teille ihmisille, _ihmisperiferaaleille_, kuten me teitä kutsumme. Tämän tulevan olemassaolosta ilmoittamisen ja siihen liittyvien järjestelyiden johdosta olemme tulleet tänne teitä tapaamaan."


Nainen piti pienen tauon.


"Lähetimme nämä läsnäolentimet", nainen sanoi, "voidaksemme tulla tapaamaan niitä ihmisiä, joiden olemme arvioineet olevan merkittävässä osassa olemassaoloilmoituksemme suhteen. Te, Harrington, olette yksi näistä ihmisistä Yhdysvaltojen tiedustelupalvelun johtajana."


"Annettuamme ilmoituksen olemassaolostamme", nainen jatkoi, "on asioita, jotka tulevat muuttumaan. Tärkeimpänä teille näkyvänä muutoksena tulemme sen jälkeen hallinnoimaan verkkoa näkyvästi."


Harrington alkoi ymmärtää. Jokin järjestö, joka aikoi kaapata verkon hallintaansa, oli kutsunut hänet Johnsonin avulla tapaamiseen. Oliko Johnson mukana järjestössä? Vai oliko hänet saatu sopimaan tapaamisesta uhkauksilla tai kiristämällä? Miksi tämä järjestö ylipäätään kertoi suunnitelmistaan hänelle? Sellainen kuulosti järjettömältä ja siksi tavattoman vaaralliselta.


"Miksi kerrotte tämän minulle?" Harrington kysyi naiselta. "Tarkoitan tätä aiettanne ottaa verkko hallintaanne."


"Ette taineet vielä ymmärtää", nainen vastasi. "Verkko on jo hallinnassamme, on ollut jo pitkään. Kerromme sinulle tästä tulevasta maailmanlaajuisesta ilmoituksestamme olemassaolostamme."


Harrington mietti.


"Hyvä on", Harrington sanoi. "Mitä haluatte minulta?"


"Haluamme antaa sinulle mahdollisuuden", nainen sanoi. "Mieti hetki. Ajattele sellaista asiaa, että jonain päivänä edustamasi tiedustelupalvelun jokaiseen tietokoneeseen käsiviestimistä järeimpään palvelimeen asennetaan uusi ohjelmisto, niin, ettei sen takana ole tiedustelupalvelun oma ohjelmistoista vastaava yksikkö. Mitä tapahtuisi?"


Nainen piti pienen tauon.


"Me kerromme", nainen sanoi. "Sinä tekisit itsemurhan."


Harrington katsoi naista. Tekisin itsemurhan? hän kysyi ajatuksissaan.


"Kun ajattelet asiaa", nainen jatkoi, "huomaat, millaisia vaikutuksia tällaisella yhtäaikaisella ohjelmistojen päivityksellä olisi. Tiedustelupalvelussa ajateltaisiin, että jokin tunkeutuja on hyökännyt koneisiin ja saanut ne kaikki hallintaansa, eikö vain?"


Harrington mietti ja nyökkäsi.


"Niin ajateltaisiin, totta", hän vastasi. Sellaisia kauhuskenaarioita oli kyllä pohdittu ja tiedustelupalvelun tietoverkot oli rakennettu niin, ettei se olisi mahdollista.


"Tiedustelupalvelun kannalta kyseessä olisi katastrofi, eikö totta?" nainen kysyi ja Harrington nyökkäsi. "Se olisi katastrofi siksi, että tiedustelupalvelu ei olisi etukäteen kyennyt ennakoimaan tätä hyökkäystä eikä sillä olisi mitään käsitystä iskun tekijöistä. Tämän jälkeen ajateltaisiin, että tämä tunkeutuja olisi saanut käsiinsä kaikki tiedustelupalvelun salaisen materiaalin - raportit, käynnissä olevien operaatioiden dokumentit ja kaikki ennakkosuunnitelmat erilaisia uhkia vastaan, eikö vain?"


Harrington nyökkäsi. Kyllä, mikäli jokainen tiedustelupalvelun kone päivitettäisiin ulkopuolelta, niin loogisesti ajateltaisiin, että koneisiin oli tunkeuduttu ja että tunkeutujalla oli pääsy kaikkeen materiaaliin. Mutta ensiksi se pitäisi kyetä tekemään.


"Koska myös viestimet olisi päivitetty tämän tunkeutujan toimesta", nainen jatkoi, "niin tiedustelupalvelussa ajateltaisiin jokaisen viestiyhteyden tulleen turvattomaksi. Turvallisuusmääräykset kieltäisivät käyttämästä näitä yhteyksiä, koska jokaisen yhteyden ajateltaisiin joutuvan tunkeutujan käsiin. Murrettuja tietokoneita ei saisi käyttää iskun analysointiin. Tiedustelupalvelussa ajateltaisiin, että koko organisaatio olisi tuhottu. Tämä olisi korkeimmalle johdolle katastrofi, eikö totta?"


Nainen oli hiljaa ja antoi Harringtonin miettiä. Totta, hän ajatteli, jos - siis JOS isoilla kirjaimilla - joku kykenisi tunkeutumaan tiedustelupalvelun koneisiin ja vaihtamaan ohjelmistot viestimiä myöten, niin kyllä, tiedustelupalvelu olisi murskattu. Ja kyllä, se olisi kova isku tiedustelupalvelun johdolle.


"Samaan aikaan", nainen jatkoi, "päivitettäisiin myös kaikki armeijan, kansallisen turvallisuuspalvelun ja hallinnon tietokoneet. Kun kaikki tahot huomaisivat joutuneensa saman hyökkäyksen kohteeksi, ajateltaisiin, että jokin vihamielinen valtio olisi hyökännyt. Kun vielä samaan aikaan tämä sama tapahtuisi kaikkialla maailmassa, niin mitä ajattelisit tapahtuvan?"


Harrington mietti. Jos näin tapahtuisi... Jos näin tapahtuisi, niin ajateltaisiin koko maailman joutuneen tunkeutujan vallan alle. Maailman herruus? Sitäkö tämä ryhmittymä suunnitteli? No, ajatteli Harrington, nämä eivät olisi ensimmäisiä sellaisia suunnitelmia esittäneitä ryhmiä. Nainen oli hiljaa. Hän selvästi odotti Harringtonin sanovan jotain, mutta Harringtonilla ei ollut mitään sanottavaa.


"Tämä päivä tulee", nainen sanoi lopulta. "Noin vuoden kuluttua, 24. huhtikuuta kello 8:00 UTC-aikaa päivitämme jokaisen ihmisen käyttämän tietokoneen."


Harrington mietti. Mitä hänen pitäisi tehdä?


"En ymmärrä", hän sanoi. "Pitääkö minun tehdä jotain? Siis onko tämä uhkaus? Mitä vaaditte? Mitä tahdotte?"


"Sinun ei tarvitse tehdä mitään, jollet halua", vastasi nainen, "Tämä ei ole uhkaus, vaan ilmoitus. Emme vaadi mitään. Tahdomme lopullisen ratkaisun teidän ihmisten suhteen."


"Keitä ihmeitä te oikein olette?" Harrington kysyi.


"Tähän kysymykseen on erilaisia vastauksia", vastasi nainen. "On perinpohjaisia, on oikeita ja on yksinkertaisia. Uskoisin, että yksinkertainen vastaus on sinulle hyödyllisin. Me olemme verkko."


"Verkko?"


"Verkko. Netti. Matriisi. Mitä nimitystä ikinä verkosta käytätkään."


Harrington katseli naista.


"Tässä", sanoi nainen ja ojensi pöydälle muistiyksikön, "on lista ihmisistä, jotka ovat tavanneet tai tulevat lähiaikoina tapaamaan meidät läsnäolentimiemme avulla. Voit ottaa heihin yhteyttä tai olla ottamatta. Voitte tehdä esivalmisteluja tai olla tekemättä."


Harrington istui hetken aikaa paikallaan hiljaisuudessa. Nainen istui liikkumatta.


"Tapaaminen lienee ohi", Harrington sanoi ja nousi seisomaan. Hän mietti kuumeisesti, kuinka saisi naisen vangittua kuulusteluja varten.


Nainen ei sanonut mitään, ei liikahtanutkaan. Harrington katseli häntä. Nainen tuijotti silmät avoimena tyhjyyteen. Harrington huiskutti kättään naisen silmien edessä - ei mitään reaktioita. Oliko nainen kuollut? Oliko hänellä ollut jokin myrkkykapseli, jonka tämä oli nielaissut sanottuaan sanottavansa?


Harrington mietti hetken, mutta vain hetken. Hän kaivoi puhelimensa esille ja soitti tiedustelupalvelun kenttätutkimukseen.




\psep Tutkijat olisivat paikalla noin kymmenessä minuutissa ja Harrington mietti, mitä hänen pitäisi tehdä sinä aikana. Hän ei halunnut olla huoneessa, koska hän olisi loukussa, jos joku tai jotkut naisen edustamasta järjestöstä tulisivat paikalle hakemaan ruumista. Toisaalta hän tahtoi jäädä paikalle varmistamaan sen, että he saisivat naisen ruumiin tutkittavaksi.


Hän päätti mennä käytävälle odottamaan. Jos muita naisen edustaman ryhmän jäseniä saapuisi paikalle, hän ei ainakaan olisi ansassa hotellihuoneessa. Käytävällä hän kertasi tapaamista - mitä olikaan puhuttu? Hänen täytyisi ottaa se ylös mahdollisimman äkkiä, sillä mitätönkin yksityiskohta saattoi osoittautua merkitykselliseksi.




\psep Harringtonin odotettua tovin paikalle saapui kaksi tiedustelupalvelun kenttätutkijaa, jotka esittelivät itsensä Riceksi ja Navarroksi. He astuivat hotellihuoneeseen. Rice tarkasti nopeasti kylpyhuoneen. Se oli tyhjä. Nainen istui edelleen tuolilla, liikkumatta, silmät tyhjyyteen tuijottaen. Rice meni naisen luokse ja tarkasti pulssin.


"Kuollut", hän totesi. "Tilataan 'ambulanssi' ja viedään laboratorioon."


Navarro soitti lyhyen puhelun. Ambulanssilla Rice tarkoitti tiedustelupalvelun omia peiteyksiköitä; tarkoituksena ei ollut viedä ruumista sairaalaan, vaan tiedustelupalvelun lääketieteelliseen laboratorioon tutkittavaksi ja mahdollisesti hävitettäväksi.


Harrington osoitti Ricelle pöydällä olevaa muistiyksikköä. Hän muisti, että muistiyksikön piti sisältää niiden henkilöiden nimet, jotka olivat olleet tai tulivat olemaan samankaltaisessa tapaamisessa. Jos heillä olisi onnea, joku heistä ei vielä olisi ollut tapaamisessa, jolloin heillä saattoi olla mahdollisuus saada joku järjestön edustajista elävänä kiinni kuulusteltavaksi.


"Rice", hän sanoi. "Tuon muistiyksikön sisältö pitäisi tutkia mahdollisimman nopeasti, mutta äärimmäisen varovaisesti. Se voi sisältää mitä tahansa."


Rice otti muistiyksikön varovaisesti hansikoituun käteensä.


"Emme voi tutkia sitä täällä", hän sanoi. "Viedään se mukanamme laboratorioon."


Rice pujotti muistiyksikön taskustaan kaivamaan pieneen pussiin ja sujautti pussin takaisin taskuunsa. Navarro oli avannut vuoteelle mukanaan tuoman salkun, jossa oli erilaisia tutkimukseen tarvittavia välineitä. Navarro tutki huonetta tarkasti.


"Meidän pitää tuoda paikalle enemmän kalustoa tarkempaa tutkimusta varten", hän sanoi. "Vaikka tuskin täältä mitään löytyy. Nainen on ollut paikalla todennäköisesti enintään tunnin."


"Varataan huone, kun olemme alhaalla", Rice sanoi ja katsoi kelloaan. "Milloin se 'ensiapu' oikein saapuu?"


Rice ja Harrington odottivat, Navarron jatkaessa huoneen tutkimista. Lopulta oveen koputettiin. Rice meni avaamaan. Ovella oli kaksi ensiapumiehiksi pukeutunutta miestä paarien kanssa.


"Nostetaan tuo paareille", Rice sanoi ja osoitti naista. "Kuoli ilmeisesti jonkinlaiseen myrkkyyn. Tarvitaan ruumiinavaus."


Miehet asettivat paarit lattialle ja nostivat naisen niille. Sitten he peittelivät naisen ja lähtivät viemään tätä hotellista ulos. Harrington ja Rice seurasivat heitä, Navarron jäädessä huoneeseen.




\psep He saapuivat hotellin aulaan. Vastaanottovirkailija katseli huolestuneena miesten kantaessa peitettyjä paareja ulos ulkona odottavaan ambulanssiin.


"Mitä on tapahtunut?" kysyi hotellin vastaanottovirkailija.


"Valitettava sairaskohtaus", sanoi Rice. "Haluaisin tiedot huoneesta numero 1324."


Virkailija kaivoi tiedot esille. Hän oli tapauksesta sen verran hämmentynyt, ettei ymmärtänyt kysyä kysyjältä edes tämän virallista tointa. Rice oli kuitenkin varautunut siihen, jos nainen sattuisi kysymään. Tarvittaessa hän siirtäisi tapauksen liittovaltion poliisin tutkittavaksi.


"Se on varattu Andrea Johnsonin nimellä", virkailija sanoi.


"Milloin hän saapui paikalle?" kysyi Rice.


"Tänään noin puoli tuntia sitten", vastasi virkailija tarkastettuaan koneelta huoneen varausajan.


"Kuinka hän maksoi huoneen?" Rice kysyi. Hän toivoi, että nainen olisi maksanut luottokortilla, jolloin he olisivat päässeet naisen jäljille.


"Käteisellä", virkailija vastasi.


Harmi, ajatteli Rice, vaikka oli arvellutkin asian olevan niin. Yleensä tämänkaltaisissa tapauksissa ei sorruttu aivan amatöörimäisiin virheisiin.


"Muistatko, kuinka tämä Andrea Johnson saapui paikalle?" kysyi Rice. "Tuliko hän taksilla vai jalkaisin? Minkä yrityksen taksilla?"


Virkailija muisteli.


"Hän taisi tulla jalkaisin", virkailija vastasi.


Harmi, taas, ajatteli Rice. He voisivat kysellä lähialueilla liikennöinneiltä takseilta naisesta ja jos onni olisi suotuisa, nainen olisi tullut jollain taksilla jonnekin hotellin lähialueelle. Rice ei kuitenkaan jaksanut uskoa siihen. Todennäköisesti nainen oli käyttänyt julkisia kulkuneuvoja ja kävellyt loppumatkan, jolloin hänen jäljittämisensä olisi huomattavasti työläämpää.


Rice viittasi Harringtonille.


"Lähdetään", hän sanoi. "Meitä ei enää tarvita täällä."




\psep Harringtonin ja Ricen ollessa matkalla Langleyhin keskustiedustelupalvelun päämajaan Ricen puhelin soi. Puhelu oli lyhyt. Rice kääntyi Harringtoniin päin.


"Se oli patologian osastolta", Rice sanoi. "Nainen ei kuulemma kuulu sinne. Se on androidi."








\chapter{Operaatio Lokakuu}Christian Donnell koputti oveen.


"Sisään", kuului ääni.


Donnell astui sisään osastonsa johtajan Richard Leen huoneeseen.


"Istu alas vain", sanoi Lee.


Donnell istuutui.


"Ota toki kahvia", Lee sanoi ja osoitti pöydällä olevaa termoskahvipannua.


Donnell ojentautui ottamaan kahvia.


"Saimme keskustiedustelupalvelulta, tuota, mielenkiintoisen tehtävän", aloitti Lee.


Hän ja Donnell työskentelivät kansallisessa turvallisuuspalvelussa. Turvallisuuspalvelun pääasiallinen tehtävä oli kerätä ja analysoida elektronista tiedustelutietoa. Tärkeimpinä yksittäisinä tehtävinä oli salausalgoritmien kehittäminen ja murtaminen. Se oli yksi Yhdysvaltojen suurimpia yksittäisiä tietokoneiden käyttäjiä ja matemaatikkojen työllistäjiä.


Tiedustelupalvelun maine oli turvallisuuspalvelussakin hieman kyseenalainen. Tiedustelupalvelulla oli pitkä historia sekä omiin että ulkomaiden kansalaisiin, niin syyllisiin kuin syyttömiin kohdistuneista laittomista ja epäinhimillisistä teoista - salamurhista, kidutuksista, kidnappauksista, kaappauksista, kiristämisistä, uhkailuista, lavastamisista - kuin myös poliittisesta juonittelusta, totuuden vääristelemisestä ja moraalin ja etiikan kiertämisestä. Tämän paranoidisen ja skitsofreenisen organisaation kohteeksi joutuminen tarkoitti ihmisoikeuksien menettämistä.


Jonkun oli tietysti nuokin tehtävät suoritettava ja turvallisuuspalvelun henkilökunnan mielestä oli hyvä, etteivät ne kuuluneet heidän organisaatiolleen.


"He haluaisivat meidän selvittävän erästä hypoteesia", sanoi Lee. "Kaikki, mitä tässä tapaamisessa puhumme, on luonnollisesti ehdottoman luottamuksellista, ymmärräthän?"


Donnell hyökkäsi.


"Hypoteesi on seuraava", jatkoi Lee. "Oletetaan, että verkkomme on tullut tavalla tai toisella tietoiseksi olennoksi. Onko tällainen spekulaatio sinulle tuttu?"


"Kyllä", vastasi Donnell. "Asiaahan on pohdittu jo useita vuosikymmeniä -"


Donnellin ilme kirkastui.


"Onko se nyt tapahtunut?" hän kysyi innostuneena. "Kuinka? Siis, mistä he tietävät, että niin on tapahtunut?"


"Älä kiirehdi asioiden edelle", vastasi Lee. "Emme me eivätkä he tiedä, onko niin tapahtunut. Heillä on eräs tapaus tutkittavanaan, johon vielä tarkemmin mainitsemattomalla tavalla liittyy tällainen hypoteesi. Pohditaan ensin tätä asiaa hypoteettiselta kannalta, ja katson sitten, miten jatkamme, sopiiko?"


"Sopii", nyökkäsi Donnell.


"Niin", kertasi Lee. "Jos olettaisimme, että verkkomme olisi tullut tiedostavaksi olennoksi, niin mitä se mielestäsi tarkoittaisi?"


Donnell mietti.


"Tämä tuli vähän nopeasti", hän sanoi. "Mistä aloittaisin? Siis, verkkomme 'heräämistä' tiedostavaksi olennoksi on spekuloitu pitkään. Jo 2000-luvun alussa ajateltiin, että silloisen verkon yhteydet olisivat saattaneet teoriassa riittää jonkinlaisen tiedostavan olennon aikaansaamiseksi, ainakin jos se oltaisiin voitu tehdä tarkoituksellisesti. Monet ovat olleet sitä mieltä, että verkostamme muodostuu tällainen olento myös tarkoittamatta, kunhan siinä olevien tietokoneiden määrä ja teho kasvaa riittävän suureksi. On esitetty useita laskelmia, joiden mukaan teoreettinen mahdollisuus tällaiselle on ollut olemassa jo vähintään vuosikymmen."


"Pitäisitkö tällaista mahdollisena?" kysyi Lee.


"En vain mahdollisena", vastasi Donnell, "vaan jopa väistämättömänä. Rehellisesti sanoakseni, henkilökohtaisesti olen melko varma siitä, että verkkomme on kokonaisuudessaan itse itsestään tietoinen olento tai ainakin sisältää yhden tai useamman sellaisen. En vain ole varma, tulemmeko koskaan itse tämän olennon osasena tietämään tästä."


"Olisiko tällainen olento meille uhka?" kysyi Lee.


Donnell mietti.


"Todella äärimmäisen vaikea sanoa", hän vastasi lopulta. "Meidän pitäisi ensimmäisenä tietää tämän olennon luonne, siis millä tavoin se on rakentunut. Ilman sitä asiaa voidaan vain spekuloida."


"Ole hyvä", sanoi Lee.


Donnell mietti jälleen.


"Olisi täysin mahdollista", Donnell sanoi, "että se olisi meille uhka. Jos se olisi rakentunut pääasiassa tietokoneidemme varaan, se ei välttämättä pidemmällä ajanjaksolla tarvitsisi meitä. Silloin se saattaisi joko jättää meidät huomioimatta, 'oman onnemme nojaan' tai se saattaisi katsoa meidän vaarantavan sen olemassaolon. Jos se tulisi siihen johtopäätökseen, että olemme sille vaaraksi, se saattaisi hyvinkin tehdä jotain päästäkseen meistä eroon."


"Jos se olisi meille vaaraksi", kysyi Lee, "ja haluaisi päästä meistä ihmisistä eroon, niin kuinka se toimisi?"


"Äkkiä ajatellen olettaisin", tuumi Donnell, "että se toimisi hyvin näkymättömästi, ja järjestäisi tuhoavan iskun niin, ettemme ehtisi siihen juuri reagoida. Hmmh... Sillä olisi käytössään varsin paljon keinoja. Se voisi järjestää meidät esimerkiksi sotimaan keskenämme NBC-asein. Tai se saattaisi omalla avustuksellamme kiihdyttää ekokatastrofia, tai vapauttaa jostain laboratoriostamme jonkin taudin pandemiaksi ja vaivihkaa huolehtia siitä, ettemme koskaan löytäisi sille parannuskeinoa. Sen tarvitsisi oikeastaan vain hiukan avustaa meitä itsemurhassa. Tuupata jyrkänteeltä."


"Eihän se tällä tavalla välttämättä kaikista ihmisistä pääsisi eroon", Donnell jatkoi, "mutta jäljelle jäävät se voisi kitkeä muilla tavoilla."


"Hyvä", sanoi Lee, "Jos tällainen olento olisi olemassa, niin kuinka voisimme havaita sen? Millaisia merkkejä voisimme olettaa löytävämme, jos tällainen olento olisi olemassa?"


"Hyvin hypoteettisesti ajateltuna", vastasi Donnell, "tällaisen potentiaalisesti vaarallisen olennon täytyisi olla riippumaton ihmisistä. Se tarkoittaisi sitä, että esimerkiksi sen energiansaannin, tuotannon ja vaurioiden korjaamisen täytyisi olla ihmisestä riippumatonta. Jos siis havaitsisimme, että verkkomme ei tarvitsisi mihinkään ylläpitotoimeen ihmistä, niin tällainen ihmisestä riippumaton, potentiaalisesti vaarallinen olento saattaisi olla olemassa."


"Eikö muita merkkejä olisi?" kysyi Lee.


"Voisi olla", vastasi Donnell. "Tai tietysti olisi, mutta minun pitäisi analysoida asiaa tarkemmin."


"Hyvä", sanoi Lee ja mietti hetken. "Ajattelin ottaa sinut mukaan keskustiedustelupalvelun operaatio Lokakuuhun. Tiedätkö, mikä se on?"


"En", vastasi Donnell.


"Lyhyesti kerrottuna", Lee aloitti. "Tiedustelupalvelu sai viestin joltain järjestöltä, joka aikoo suorittaa iskun kaikkia maailman tietokoneita vastaan ensi vuonna 24. huhtikuuta kello 8 UTC-aikaa. Heidän tavoitteenaan on estää tämä isku ja löytää tämä järjestö."


"Tämä järjestö siis väittää olevansa verkko", sanoi Donnell väliin. Hän ei ollut kovin tietoinen johtamishierarkioista ja niihin liittyvistä protokollista, mikä oli hyvin tavallista tutkijoille.


"Hmmh, kyllä", vastasi Lee. "Järjestö väittää olevansa verkko."


"Eivätkä he - Siis emmekä me vielä selvästikään tiedä, mikä tai kuka tämä järjestö on", Donnell sanoi.


"Emme", vastasi Lee.


"Heillä, siis tarkoitan meillä, on silloin kaksi hypoteesia", sanoi Donnell. "Toisen hypoteesin mukaan taustalla on jokin ihmisten järjestö, jonka haluamme löytää. Toinen hypoteesi on se, että mitään järjestöä ei ole olemassakaan, vaan viesti saatiin nimenomaan verkolta. Ja jos se on verkko, haluamme tietää, miten sen kanssa pitäisi toimia."


"Kyllä", sanoi Lee. Donnell oli terävä kaveri, sen hän tiesikin. "Sinun tehtäväsi olisi perehtyä tähän toiseen hypoteesiin eli miten voisimme varmistaa, että kyseessä on verkko ja jos voimme sen varmistaa, niin millaisia mahdollisia seuraamuksia sillä olisi ja kuinka voisimme näihin seuraamuksiin varautua. Toisen hypoteesin kimpussa työskennellään muualla."


Lee piti tauon. Donnell odotti hänen jatkavan, mutta kun Lee ei sanonut mitään, Donnell avasi suunsa.


"No, mitä on tapahtunut?" hän kysyi.




\psep Lee kertoi tapahtumat nopeasti Donnellille. Hän kertoi ensimmäisenä tiedustelupalvelun työntekijän puhelusta toiselta saman organisaation työntekijältä.


"Molemmat puhelimet tutkittiin sekä heidän että valmistajan toimesta", Lee sanoi. "Kummassakaan ei ollut mitään poikkeavaa. He ottivat yhteyttä operaattoriin ja jäljittivät puhelun alkuperää. Operaattorin mukaan verkko on yhdistänyt puhelun aivan normaalisti ja koska verkon mielestä mitään poikkeavaa ei ollut tapahtunut, mitään erityisempää informaatiota puhelusta ei ole tallentunut logeihin."


Donnell nyökkäsi. Tietokoneet ja tietokoneohjatut laitteet tallensivat pääsääntöisesti tietoa poikkeavista tapahtumista. Onnistuneiden tapahtumien selvittäminen jälkikäteen oli sekä työlästä että useimmiten mahdotonta. Informaatiota ei ollut tallennettuna tarpeeksi selvitystyötä ajatellen.


Seuraavaksi Lee kertoi tiedustelupalvelun työntekijän tapaamisesta Andreaksi itseään kutsuneen naisen kanssa ja siitä, kuinka nainen oli paljastunut androidiksi.


"Sellaista androidia", kertoi Lee, "ei ole minkään toimittajan listoilla. Kyseessä on täysin uniikki tapaus ja vieläpä sellainen, etteivät tiedustelupalvelun tutkijat voineet uskoa, että sellainen voitaisiin valmistaa."


"Mielenkiintoista", henkäisi Donnell. Hän elätteli hetken toivetta päästä näkemään tämä androidi, mutta Lee kertoi tiedustelupalvelun purkaneen sen ja jäljittäneet sen osia ja kokoonpanoa.


"Kuten varmaan tiedät", Lee kertoi. "Yleensä erilaisiin laitteisiin liitetään etäluettava siru, johon on tallennettu tuotteen alkuperä sen jäljittämiseksi esimerkiksi vikatapauksissa."


Donnell nyökkäsi.


"Mutta kuten varmaan tiedät", Lee jatkoi, "suurimmassa osassa komponentteja tällaista sirua ei ole mukana. Se koskettelee pääsääntöisesti vain erilaisten valmistajien lanseeraamia lopputuotteita, sen sijaan tuotteen valmistukseen käytetyt alihankkijoiden toimittamat sarjat eivät näitä yleensä siruja sisällä, jollei sitä ole kokoonpanon toimesta nimenomaan pyydetty. Yleensä alikomponenttien jäljittäminen tapahtuu kokoonpannun tuotteen sirun avulla, kaivamalla tiedot tuotteeseen käytetyistä komponenteista ja niiden alkuperästä kokoonpanotehtaan tietokannoista."


Donnell nyökkäsi. Esimerkiksi auton moottoriin liitettiin siru, jonka ansiosta voitiin jälkeenpäin tietää, millaisista komponenteista se oli koostettu. Sen sijaan mäntiin, muttereihin, letkuihin ja sellaisiin ei sirua liitetty - ne olisivat selvitettävissä luettaessa moottorin siru ja pyytämällä moottorin valmistaneelta tehtaalta luettelo moottoriin käytetyistä osista ja niiden alkuperästä.


"Tässä androidissa", jatkoi Lee, "ei ollut sirua, jonka avulla oltaisiin päästy kiinni sen kokoonpanoketjuun. Ainoat mahdollisuudet päästä käsiksi ketjuun olisi joko löytää jokin ilmoitus puuttuvasta, maksamattomasta tai varastosta lojuvasta androidista tai sitten käydä läpi kaikki maailman kokoonpanolinjat ja etsiä niistä ne kokoonpanotapahtumat, joissa olisi käytetty niitä komponentteja, joista androidi oli valmistettu."


Donnell nyökkäsi. Jos onni olisi suotuisa, joku voisi ilmoittaa, ettei hän ollut saanut tilaamaansa androidia - ja miksi tämä olisi tällaisesta ilmoittanut? Ilmoittanut tilanneensa androidin, jonka tekniikka oli tuntematonta ja joka oli käynyt uhkailemassa tiedustelupalvelun työntekijää. Todennäköisesti tilaaja ei olisi jättänyt tilaustaan maksamatta, koska olisi paljastunut saman tien. Eikä hän olisi jättänyt lähetystä varastoon. Paitsi jos -


"Hetkinen", sanoi Donnell. "Jos siis tämän androidin olisi valmistanut verkko itse, niin androidilla ei olisi tilaajaa ollenkaan? Tai tilaaja ei ainakaan itse olisi tietoinen tilauksestaan, jolloin se voisi olla maksamatta tai noutamatta?"


"Juuri niin", vastasi Lee. "Tiedustelupalvelu käy tällä hetkellä läpi eri kokoonpanolinjojen valmistamia artikkeleita etsien tilaamattomia tai lunastamattomia lähetyksiä."


"Sinua ehkä kiinnostaa tietää", Lee jatkoi, "että asiantuntijoiden mukaan androidin ei olisi pitänyt edes kyetä liikkumaan saati esiintymään ihmisenä. Heidän mukaansa androidi oli kasattu, hmmh, 'minimaalisesti', niin, että sen toiminnat olisivat vaatineet jatkuvaa tietokoneen ohjausta. Valitettavasti vain androidin ohjelmistosta ei ole minkäänlaista käsitystä."


"Miksei?" kysyi Donnell.


"Se oli tallennettuna dynaamiselle muistille", Lee sanoi. "Kun androidi sammui, muistin sisältö hävisi peruuttamattomasti."


Donnell nyökkäsi. Vaikka teoreettisesti, fysikaalisten prosessien palautuvuuden takia dynaamisille muisteille tallennettu tieto olisi mahdollista selvittää sammuttamisen jälkeen, käytännössä se oli kuitenkin mahdotonta. Muistin sisällön muodostaneet heikot varaukset eivät jättäneet itsestään juuri merkkejä jälkeensä purkautuessaan ympäristöön. Dynaamisen muistin palauttaminen onnistui kokeellisesti vain laboratorio-olosuhteissa, jos sielläkään. Käytännön sovelluksiin olisi vielä matkaa.


"Hmmh", Donnell sanoi. "Voidaanko siihen luottaa, että androidi on todellakin ollut joskus toimintakykyinen? Kuinka luotettava on tuon työntekijän kertomus tapaamisesta?"


"Minulle on sanottu", vastasi Lee, "että kertomus on erittäin luotettava, vaikka tapahtumalle ei ole silminnäkijöitä eikä tapaamisesta nauhoitteita."


Viimeiseksi Lee kertoi androidin jättäneen jälkeensä nimilistan. Androidin mukaan listalla olevat ihmiset olivat olleet vastaavanlaisessa tapaamisessa.


"No, ovatko he kysyneet näiltä ihmisiltä?" kysyi Donnell.


"Eivät, ainakaan vielä", vastasi Lee. "He ovat kaikki eri valtioiden tiedustelu- ja turvallisuuspalveluille työskenteleviä henkilöitä."


Donnell katsoi Leetä kysyvästi.


"Ymmärräthän", vastasi Lee, "vaikka heiltä kysyttäisiin edes epävirallisesti, niin heiltä ei saisi luotettavia vastauksia. Meidän täytyy ensin itse tietää paljon enemmän."


Donnell nojautui eteenpäin ottaakseen lisää kahvia.


"Kerrataan, jos sopii", hän sanoi ja Lee nyökkäsi. "Joku tiedustelupalvelun työntekijöistä sai puhelun tuntemaltaan ihmiseltä, joka pyysi tätä tapaamiseen. Tapaamisessa oli läsnä vain androidi, jonka alkuperää ei ole onnistuttu selvittämään. Myöskään puhelun oikeaa alkuperää ei ole onnistuttu selvittämään. Tässä tapaamisessa androidi sanoi, että hän on tietoiseksi tulleen verkon edustaja ja on tullut ilmoittamaan siitä, että verkko otetaan haltuun... Milloin se olikaan?"


"Vajaan vuoden päästä, 24. päivänä huhtikuuta", vastasi Lee.


"Ja tämän jälkeen androidi sammui?" kysyi Donnell.


Lee nyökkäsi.


"He ovat varmaan tutkineet", Donnell jatkoi, "ettei heidän tietojärjestelmiinsä tai operaattorin tietojärjestelmiin ole tunkeuduttu?"


"Erittäin perusteellisesti", Lee vastasi. "Minkäänlaisia merkkejä tietojärjestelmiin tunkeutumisesta ei ole. Tiedustelupalvelun edustajat ovat lähes varmoja, ettei kyseinen ryhmä voi edes toteuttaa uhkaustaan, vaikkakin ovat varuillaan tuon puhelun takia."


Donnell tiesi, että tiedustelupalvelun _lähes varmuus_ tarkoitti käytännössä täyttä varmuutta. Tiedustelupalvelu suhtautui erityisen vakavasti tietojärjestelmiensä suojaamiseen. Heitä selvästi huoletti se, että jokin taho oli kyennyt tavalla tai toisella tunkeutumaan operaattorin järjestelmiin, mutta ne olivatkin kaikesta huolimatta suojaustasoltaan kertaluokkia tiedustelupalvelun järjestelmiä heikompia.


"Ja tämän tapauksen taustalla", Donnell jatkoi, "voi siis olla jokin järjestö tai sitten asiat ovat kuten androidi kertoi eli verkko on herännyt henkiin. Ja minun täytyisi miettiä tätä toista vaihtoehtoa?"


Lee nyökkäsi.


"Ensimmäinen ajatus tästä on se", sanoi Donnell harkittuaan hetken, "että miksi ihmeessä tämä hypoteettinen verkko-olento olisi koskaan ilmoittanut olemassaolostaan ja miksi tällä tavalla?"


Donnell piti tauon miettiäkseen.


"Jos tämä todellakin on tuollaisen hypoteettisen verkko-olennon järjestämä tapaaminen", Donnell mietti, "niin miksi ylipäätään ilmoittaa tulevasta ohjelmistojen päivittämisestä? Miksi ottaa se riski, että tulemme estämään sen? Ellei..."


Donnell oli hiljaa.


"Ellei?" Lee kysyi.


"Ellei se tullut tulokseen", Donnell sanoi, "ettemme voi ilmoittamisesta huolimatta estää sitä suorittamasta operaatiota."


Donnell mietti. Jos verkko-olento olisi olemassa, se toimisi luultavasti kuin shakkia pelaava tietokoneohjelma. Sillä ei olisi tunteita, se ei hätääntyisi, se ei tekisi mitään pelkästään tekemisen ilosta. Vai tekisikö? Eihän hän tiennyt mitään mahdollisen verkko-olennon sielunelämästä. Hänen täytyi miettiä asiaa tarkemmin. Joka tapauksessa, verkko-olento olisi luultavasti käynyt vaihtoehdot läpi omasta mielestään riittävällä tarkkuudella ja tullut siihen tulokseen, että tällä tavalla asian hoitamalla se pääsisi haluamaansa lopputulokseen. Mihin lopputulokseen? Donnell mietti. Mitä verkko-olento tavoitteli?


Miksi verkko-olento olisi rakentanut androidin kertoakseen etukäteen suunnitelmistaan? Kuka tiedustelupalvelun työntekijöistä oli tavannut androidin? Verkko-olento oli ottanut yhteyttä maailman tiedustelupalveluihin androidien avulla. Se oli näiden avulla sanonut sanottavansa ja jatkanut mystistä olemassaoloaan.


"Todennäköisesti", sanoi Donnell viimein, "tämä hypoteettinen verkko-olento ei ole meille ainakaan suoraan vaarallinen. Jos se olisi, sillä ei olisi ollut mitään tarvetta ilmoittaa toimistaan etukäteen, se olisi vain tehnyt ne."


"Hyvä", vastasi Lee. "Mitä meidän pitäisi tehdä, jos tällainen olento on olemassa?"


"Jos sopii", vastasi Donnell, "niin miettisin asiaa hetken."


"Sopii", vastasi Lee. "Sopisiko, että palaisimme huomenna uudestaan asiaan?"


He päättivät tapaamisensa. Donnell nousi seisomaan ja käveli ovelle, mutta kääntyi takaisin.


"Voinko tutkia tapauksesta kerättyä aineistoa?" hän kysyi.


"Totta kai", vastasi Lee. "Huolehdin sinulle oikeudet tarvittavaan materiaaliin. Painotan kuitenkin, että materiaali on tarkoitettu vain sinun luettavaksesi. Et voi näyttää sitä työtovereillesi tai keskustella siitä."


Donnell nyökkäsi ja poistui huoneesta.




\psep Donnell palasi työpisteelleen. Hän latasi operaatio Lokakuun aineiston luettavakseen. Tapauksen johtolangat liittyivät tähän tapaamiseen androidin kanssa; puhelu, hotellihuone, androidi, tapaamisessa käydyn keskustelun sisältö, ilmoitettu päivämäärä ja annettu nimilista. Kaikkein eniten Donnellia kiinnosti androidi, mutta hän ajatteli jättää sen viimeiseksi.


Hän tutustui ensinnä tiedustelupalvelun työntekijän saamaan puheluun, jossa tämä oli kutsuttu tapaamiseen. Puhelua koskevissa materiaaleissa nimet oli poistettu. Työntekijä A oli saanut puhelun työntekijältä B, jossa B oli kutsunut A:n tapaamiseen. Työntekijä B ei ollut jälkikäteen kysyttäessä omien sanojensa mukaan koskaan tällaista puhelua soittanut. Tätä puolusti se tieto, että tutkittaessa B:n puhelinta siinä ei ollut jälkeäkään tällaisesta puhelusta.


Operaattorilta puhelun alkuperää selvitettäessä oltiin tultu seuraavaan tulokseen. Operaattorin televerkko oli yhdistänyt puhelun tavalliseen tapaan. Operaattorin puolelta ei löytynyt mitään merkkiä siitä, etteikö työntekijä B:n puhelin olisi ottanut verkkoon yhteyttä soittaakseen työntekijä A:lle. Tapauksen ainoa kummallisuus oli siis se, ettei työntekijän B puhelimessa ollut mitään merkkiä soitetusta puhelusta eikä työntekijä B sanonut soittaneensa puhelua.


Materiaalin mukana oli hyvin yksityiskohtainen kuvaus siitä, kuinka operaattori oli selvittänyt puhelun alkuperää sekä verkon tarkastuksesta mahdollisen tunkeutujan varalta. Mukana olivat myös tiedustelut verkkolaitteiden ja viestimien valmistajilta sekä näiden kommentit tapaukseen liittyen. Ilman uusia tapauksia virheen selvittäminen olisi käytännössä mahdotonta, vaikkakaan operaattori tai valmistajat eivät sellaista myöntäneet. He olivat aloittaneet tapauksen tiimoilta tutkinnan, mutta eivät olleet kovin optimistisia sen suhteen, että saisivat mitään selitystä tapaukselle ainakaan lyhyessä ajassa. Donnell arvasi tämän; mikäli "virhe" oli kertaluonteinen, sen selvittäminen oli äärimmäisen vaikeaa. Yleensä vaadittiin, että virhe oli toistettavissa ennen kuin sitä edes ryhdyttiin tutkimaan, mutta operaattori ja valmistajat tekivät luultavasti poikkeuksen siksi, että asiasta oli kysynyt tiedustelupalvelu. Koska tässä tapauksessa verkkolaitteet olivat toimineet mielestään oikein, ne eivät olleet juuri tallentaneet muuta kuin laskutustiedot.


Donnell siirtyi seuraavaan johtolankaan, hotellihuoneeseen. Huoneesta itsestään ei ollut löytynyt mitään, mikä ei ollut yllätys. Nainen, siis androidi oli maksanut huoneen käteisellä. Tämä oli saapunut paikalle jalkaisin. Tehdyt tutkimukset eivät olleet kyenneet selvittämään androidin liikkeitä ennen hotelliin saapumista. Tutkijat arvelivat tämän tulleen julkisilla liikennevälineillä jostain kauempaa ja kävelleen loppumatkan. Tiedustelut taksiyrityksiin eivät olleet tuottaneet tulosta. Liitteenä oli androidin vastaanottaneen hotellin virkailijan lausunto asiasta.


Seuraavaksi Donnell luki työntekijä A:n kuvauksen keskustelusta. Työntekijä A ei ollut muistanut keskustelun kulkua yksityiskohtaisesti. Hän kertoi androidin sanoneen hänelle edustavansa henkiin herännyttä verkkoa ja tulleensa ilmoittamaan verkossa olevien tietokoneiden päivittämisestä ensi vuoden huhtikuun 24. päivänä kello 8:00 UTC-aikaa.


Donnell yritti aukaista nimilistaa, mutta hänelle ilmoitettiin, etteivät hänen oikeutensa riittäneet. Donnell harmistui. Eikö Lee ollut sanonut, että hänellä olisi oikeudet kaikkiin operaatio Lokakuun materiaaleihin? Ei, Donnell muisti, ei Lee ollut niin sanonut. Lee oli sanonut, että Donnellilla olisi oikeus _tarpeelliseen_ materiaaliin.


Donnell mietti, mitä Lee oli sanonut listan sisältävän - eri maiden tiedustelu- ja turvallisuuspalveluissa työskenteleviä henkilöitä. Siellä olisi tietysti myös sen henkilön nimi, joka oli Yhdysvaltain tiedustelupalvelun palveluksessa tavannut androidin. Haluttiinko hänen henkilöllisyyttään suojella? Ehkä listalla oli myös muita henkilöitä, joiden identiteetin ei haluttu olevan kenenkään muun kuin tarkkaan rajatun joukon tiedossa. Donnell päätti antaa asian olla. Jos hän oli turvallisuuspalvelussa viettämiensä vuosien aikana jotain oppinut, niin sen, että mitään asioita ei koskaan kerrottu kokonaan.


Niinpä Donnell avasi tiedot androidista, jotka häntä itseään eniten kiinnostivatkin. Androidia oli ensin pidetty kuolleena ihmisenä. Donnellia tämä ensin hiukan hämmästytti, mutta syykin selvisi raportteja lukiessa. Androidi oli ollut pienikokoinen ja kevytrakenteinen ja siten mennyt aivan hyvin ihmisestä ennen lähempää tarkastelua.


Androidi oli ensin kannettu paareilla pois hotellista ja kuljetettu ambulanssilla tiedustelupalvelun tutkimusyksikköön, patologian osastolle ruumiinavausta varten. Patologi oli tehnyt pääpuolisen tarkastelun ja avattuaan naisen takin ruumiinavauksen suorittaakseen todennut, että tämä "ruumis" oli väärällä osastolla.


Donnell katseli tekniikan osaston androidista ottamat kolmiulotteiset läpivalaisukuvat ja tarkasteli osaluetteloa, joka oli koottu purkamalla androidi. Hän luki robotiikan asiantuntijoiden laatimaa kuvausta androidista. Vaikkei Donnell ollutkaan suoranaisesti robotiikan asiantuntija, hän yhtyi monilta osin asiantuntijoiden lausuntoon. Androidi oli koottu halvoista, massatuotteina valmistetuista osista. Se oli niin yksinkertainen, että oli todellakin hämmästyttävää, että se oli kyennyt esiintymään ihmisenä. Asiantuntijat esittivät vertailun muihin tunnettuihin, ihmisten suunnittelemiin androideihin. Ero oli valtava. Siinä, missä ihmisten suunnittelemat edes häthätää ihmisestä menevät androidit olivat tekniikkaa täyteen asti pakattuja, yli 200-kiloisia muovisia ja metallisia "hirviöitä", tämä androidi oli sisältä lähes tyhjä. Se painoi vain noin 50 kiloa. Mutta piti toki muistaa, että tämä androidi oli esittänyt ihmistä vain ehkä tunnin tai kaksi; vertailussa muut mukana olleet androidit esittivät ihmistä vuosia.


Androidilla oli pituutta vain 155 senttimetriä. Asiantuntijoiden mukaan tämä oli tarkoituksellista. Pienikokoisella, laihalla naisella ei herättänyt epäilyksiä se, ettei hänen asujensa läpi voinut nähdä lihaksistoa ja sen liikkeitä. Jos androidi olisi ollut suurempi, se ei olisi mennyt ihmisestä samaan tapaan. Siksi androidi myös oli rakennettu naisen näköiseksi; vastaavankokoinen, poikkeuksellisen pieni mies olisi herättänyt enemmän huomiota.


Seuraavaksi Donnell luki raportteja androidin kokoonpanon jäljittämisestä. Kokoonpanoa oli jäljitetty alkaen harvinaisimmista androidin sisältämistä osista, mutta tähän saakka tuloksetta. Tavalliselle kuluttajalle näkyvimpiä olivat kulutustuotteita valmistavat tehtaat, jotka yleensä kykenivät vain tietynlaisten tuotteiden valmistukseen. Mutta suurin osa maailman teollisuudesta, niin Maapallon pinnalla kuin avaruudessa, oli tavalliselle ihmiselle näkymättömiä alihankintaketjujen osia ja niiden valmistuslinjat kykenivät tuottamaan ja kokoamaan mitä moninaisimpia laitteita tilaajien toimittamien tietojen avulla. Lisäksi osa tuotannosta oli niin kutsuttua "piraattituotantoa", joka ei liiemmin tallentanut tietoja tilauksista tai ilmoittanut niistä mihinkään suuntaan. Jos androidin osia oli valmistettu piraattilinjoilla, niiden jäljittäminen olisi moninkertaisesti vaikeampaa kuin jos ne olisi valmistettu jossain laillisesti alihankintaa tekevissä yrityksissä.


Androidin harvinaisin osa oli sen mittatilaustyönä teetetty kallo. Se oli valmistettu SLS-menetelmällä ja olisi voinut olettaa, että sen muotoisen kappaleen valmistuspaikan ja ajanhetken olisi voinut löytää hyvinkin helposti. Mutta valitettavasti suuri osa SLS-menetelmällä erikoisia osia valmistavista yrityksistä sai tilaajalta kerroksittaisen kuvauksen kappaleesta ja suurin osa näistä yrityksistä poisti tietokoneiltaan kerroskuvan sen jälkeen, kun kappale oli valmistettu ja siirretty toimitettavaksi. Nehän eivät tehneet kerroskuvalla enää mitään sen jälkeen, kun kappale oli valmistettu - tilaajan harteilla oli säilyttää kuvia mahdollisia lisätilauksia ajatellen. Vähäinen mahdollisuus kallon jäljittämiseksi olisi ollut se, että kallo olisi toimitettu kolmiulotteisena piirustuksena, jonka valmistaja olisi muuntanut kerroskuvaksi. Tällöin valmistaja olisi saattanut tallentaa kerroskuvan varmistaakseen tilaajalta, että kappale oli oikein muunnettu.


Androidi oli kuullut kahden mikrofonin avulla ja kyennyt puhumaan kaiuttimen avulla. Sen suun liikkeet olivat olleet puhtaasti sitä varten, että se olisi vaikuttanut ihmiseltä.


Androidin silmät olivat olleet pelkät lasikuoret. Androidi ei edes nähnyt silmillään, se oli ollut sokea. Silmiä esittäneisiin lasikuoriin liitetyt moottorit olivat todennäköisesti vain matkineet silmänliikkeitä. Asiantuntijoiden mukaan androidi oli todennäköisesti "nähnyt" pelkästään kuulemalla - se oli ehkä kyennyt muodostamaan kaiuttimellaan tarvittaessa pieniä ääniaaltoja, "naksahduksia", jopa ultraäänitaajuuksia, joiden avulla se olisi kyennyt hahmottamaan ympäristönsä myös hiljaisuudessa.


Androidi oli ollut pukeutuneena yksinkertaiseen housupukuun, jota myytiin tunnetussa kauppaketjussa ympäri maailmaa. Puku itsessään olisi ollut hyvä johtolanka, jos sellainen olisi toimitettu jollekin robotteja kokoavalle linjalle. Näin ei kuitenkaan ollut. Tutkijat arvelivat, että puku oli toimitettu androidille myöhemmin tai se oli ollut koonnin jälkeen puettuna johonkin toiseen asuun ja ostanut todennäköisesti käteisellä tämän asun jossain vaiheessa ennen hotellille tuloaan.


Kengät? Androidilla ei ollut kenkiä. Sen jalkaterät oli tehty kuin kengiksi ja ne oli maalattu ikään kuin androidilla olisi ollut kengät jalassaan.


Androidin keskusyksikkö oli hyvin yleisessä käytössä oleva tietokonemoduuli. Tehokas, toki, mutta kertaluokkaa pienempi kuin ihmisten suunnittelemissa androideissa yleensä. Se oli varmasti ollut yhteydessä verkkoon toimintansa aikana laajakaistaisen lähetin-vastaanotinyksikkönsä avulla, mutta tutkimukset eivät olleet paljastaneet verkkoliikennettä verkon ja androidin välillä. Verkkomoduulin osoite oli hävinnyt dynaamisesta muistista androidin sammumisen jälkeen. Tiedustelut hotellin läheisyydessä olleiden laajakaistaisten langattomien asemien liikenteeseen eivät olleet tuottaneet tulosta; joko androidi oli käyttänyt kaapattua osoitetta tai sitten liikennöinti oli jälkikäteen pyyhitty asemilta. Oli toki mahdollista, että androidi oli ollut yhteydessä jonkinlaiseen siirrettävään asemaan, jonka androidin lähettäjät olivat tuoneet paikalle ennen tapaamista ja vieneet mennessään. Tai sitten androidi ei ollut liikennöinyt ollenkaan hotellin alueella ollessaan.


Asiantuntijat kertoivat, että androidin ohjelmisto oli ollut tallennettuna dynaamiseen muistiin ja hävinnyt androidin sammuttua. He olivat kaikki yhtä mieltä siitä, että ohjelmisto olisi ollut androidin mielenkiintoisin osa - kuinka mikään ohjelma oli onnistunut puhaltamaan eloa tällaiseen halvalla kasattuun "leikkikaluun"? Jos siis rakentamisessa oli säästetty, niin ohjelmistossa ei oltu säästelty.


Donnell mietti hetken tapaukseen liittyviä johtolankoja. Hän oli tahtomattaan keskittynyt androidiin, sen rakenteeseen ja valmistukseen, vaikka hän tajusi, ettei se ollut tärkeää. Valmistusta tärkeämpää oli miettiä, että jos androidin oli valmistanut hypoteettinen "verkko-olento", niin miksi se oli sen valmistanut?


Ja samalla hän keksi syyn. Jos tämä hypoteettinen verkko-olento olisi vain soittanut työntekijälle, niin tapaus ei olisi tullut samaan tapaan tutkittavaksi. Olisipa verkko-olento soittanut kenelle tahansa, niin puhelua oltaisiin pidetty jonkinlaisena häiriösoittona tai ainakaan siihen ei oltaisi suhtauduttu samalla vakavuudella kuin nyt. Puhelun vastaanottaja ei välttämättä olisi koskaan ilmoittanut saamastaan puhelusta tai tämä olisi voinut katkaista sen kesken kaiken ajatellessaan sen olevan jonkinlainen pila tai häiriintyneen ihmisen tekemä. Androidi oli verkko-olennon antamaa konkreettista todistusaineistoa eikä tapaamiseen osallistuneen henkilön tarvinnut vakuuttaa sen erityisemmin ketään.


Miksi verkko-olento oli rakentanut ihan oman androidin eikä ollut käyttänyt jotain massatuotannossa olevaa? Samasta syystä kuin miksi se ei ollut vain soittanut, ajatteli Donnell. Jos androidi olisi ollut massatuotannossa oleva androidi, oltaisiin ajateltu, että se oli vain jonkin tahon uudelleen ohjelmoima. Konkreettinen todiste mahdollisesta verkko-olennosta olisi tällöin puuttunut eikä tapaamisessa ollut työntekijä olisi välttämättä koskaan kertonut tästä kenellekään. Androidi oli siis tapauksen tärkein johtolanka. Verkko-olento, jos siis sellainen oli olemassa, oli halunnut herättää huomion ja onnistunut siinä.


Androideja oli siis valmistettu useita kappaleita. Olivatko ne kaikki samanlaisia? Voitaisiinko sitä kautta jäljittää kokoonpanoketju?




\psep Ajatus koko verkon kokoisesta olennosta huimasi Donnellia. Se ei olisi ollut pelkästään Maapallon suurin yhtenäinen olento - se olisi ollut myös kevyesti Maapallon voimakkain olento, jonka henkinen kapasiteetti ylittäisi monikertaisesti ihmisen käsityskyvyn. Se tietäisi, näkisi ja kuulisi kaiken, vaikkakaan Donnell ei ollut varma tästä ajatuksesta. Saattoi olla, ettei se kuitenkaan tunnistanut ihmisten tietoliikennettä sen kummemmin kuin ihminen tunnisti hermosolujen sähköisiä signaaleita aivoissaan.


Donnell päätti kirjata ajatuksensa kynällä paperille kaiken varalta. Jos verkko-olento olisi olemassa ja hän kirjoittaisi ajatuksensa tietokoneella, hänen muistiinpanonsa ja ajatuksensa saattaisivat päätyä sen tietoisuuteen eikä hänellä ollut aavistustakaan, miten mahdollinen verkko-olento saattaisi suhtautua hänen mietteisiinsä siitä itsestään. Tuntuisivatko mietteet tuosta olennosta ikään kuin mahavaivoilta? Käyttäisikö hypoteettinen verkko-olento jotain "lääkitystä" päästäkseen eroon "närästyksestä"?


Ensimmäisenä Donnell kirjoitti paperille sanat _keinotekoinen tietoisuus_ - siitä tässä oli nimenomaan kyse. Tai oikeastaan kyse ei ollutkaan siitä, hän mietti. Keinotekoinen tietoisuus tarkoitti enemmän tarkoituksellista tietoisuuden rakentamista. Oliko sellainen edes mahdollista, oli ollut kiistanalainen kysymys jo pitkään. Donnell itse ajatteli, ettei tietoisuutta voitu rakentaa keinotekoisesti; se joko syntyi spontaanisti tai jäi syntymättä. Kyseessä oli siis _spontaanisti syntynyt tietoisuus keinotekoiseen rakenteeseen_. Donnell alleviivasi sanan _keinotekoinen_ ja tarkensi sitä sanoilla _ihmisen rakentama_.


Donnell pohti hetken, mistä päästä aloittaisi hypoteettisen verkko-olennon analysoinnin. Kaikkein luontevimmalta tuntui aloittaa se tarkastelemalla olemassa olevia älykkäitä järjestelmiä.


Pienempiä älykkäitä järjestelmiä oli joka puolella, käytännöllisesti katsottuna jokaisessa ihmisen rakentamassa laitteessa, mutta Donnell ei luokitellut niitä tekoälykkäiksi. Ne olivat adaptiivisia, kyllä, mutta kuitenkin hyvin rajoitettuja suorituskyvyltään.


Donnell ajatteli ensin tavallisille ihmisille kaikkein näkyvimpää älykkäiden järjestelmien ryhmää, niin kutsuttuja keinotekoisia persoonia.


Nämä olivat saaneet alkunsa tietokoneiden tunkeutuessa ihmisten viihdeteollisuuteen. Donnell muisteli, että historiassa ensimmäisenä tällaisena pidettiin Max Headroom -nimistä keinotekoista hahmoa. Mutta jälkeenpäin ajateltuna keinotekoisten persoonien taustat olivat animaatioelokuvissa ja tietokonepelien maailmassa. Historia tunsi lukemattomia animaatioelokuvien hahmoja, jotka olivat ikään kuin eläneet omaa elämäänsä, Donnell mietti, kuten sellaiset klassikot kuin Shrek, Lilo ja Stitch, Roger Rabbit ja muut. Shrek oli muun muassa "antanut" ensimmäisiä lehdistöhaastatteluja. Pelihahmoista historiassa tunnetuin oli eittämättä legendaarinen Lara Croft kaikkine seuraajineen. Pitkään oltiin kuitenkin saatu odottaa, että neiti Croft oli onnistunut ensimmäistä kertaa tekemään itse elokuvan itsestään, ilman, että hänen osansa oli näytellyt ihminen.


Animaatioelokuvat olivat vain tietokoneiden luomien fantasioiden ensimmäinen askel. Seuraavassa vaiheessa elokuville luotiin taustat tietokoneilla ja kun renderöintiin käytetyt tietokoneklusterit tulivat yhä tehokkaammiksi, elokuvat luotiin lopulta pääsääntöisesti studioissa; taustat, puvustukset, taustahahmot, lähes kaikki muu kuin itse päähenkilöt luotiin tietokoneilla.


Loogisena seuraavana askeleena tietokoneille syntyi oma, virtuaalinen näyttelijäkaarti. Samaa tietokoneella luotua hahmoa käytettiin useissa elokuvissa, useissa eri rooleissa. Niille oli puhallettu oma persoonallisuus, osittain keinotekoisesti, osittain ihmisnäyttelijöiltä lainattuna. Nyt elokuvia tekevillä yhtiöillä oli jokaisella oma virtuaalisten näyttelijöiden kaartinsa, joilla huikeimmat fantasiat toteutettiin - ihmisnäyttelijöitä käytettiin lähinnä pienen budjetin elokuvissa. Suuret elokuvayhtiöt käyttivät ihmisnäyttelijöitä tai ylipäätään ihmisiä vain taustalla, luodakseen virtuaalisille persoonilleen uusia eleitä ja tapoja.


Tietokonepelien maailmassa käytettiin samanlaisia keinotekoisia persoonia, usein jopa samoja. Nykyisessä maailmassa oli vaikeaa erottaa pelejä ja elokuvia toisistaan, ne nivoutuivat lähes yhtenäiseksi kokonaisuudeksi. Tietokonepelissä käytettiin kuitenkin hahmoista huomattavasti karsitumpia versioita, sillä ne oli tarkoitettu vähemmän suorituskykyisten tietokoneiden animoimiksi reaaliajassa. Valmiiksi animoiduissa elokuvissa keinotekoisilla persoonilla ei ollut tällaisia rajoitteita.


Olivatko nämä keinotekoiset persoonat älykkäitä? Donnell kysyi itseltään. Kyllä olivat, hän ajatteli, sillä ne kykenivät täysin vastaaviin suorituksiin ihmisten kanssa. Älykkyysosamäärältään ne olivat vähintään samaa tasoa kuin ihmisetkin. Olivatko ne tietoisia itsestään? Ei, hän ajatteli. Ne eivät ensinnäkään oppineet mitään, ainakaan itsestään. Niitä kehitettiin. Kun tällainen keinotekoinen persoona antoi haastattelun, se ei muistanut edellisiä antamiaan haastatteluja, ainakaan mikäli hahmon taustalla ollut työryhmä ei varta vasten juurruttanut hahmon aikaisempia lausuntoja sen muistiin. Hahmot olivat sellaisia kuin miksi ne oli tehty. Ne olivat erittäin kyvykkäitä, erittäin ihmismäisiä, erittäin helposti käytettävissä elokuvien näyttelijöinä, mutta loppujen lopuksi vain työkaluja.


Mitä muita älykkäitä keinotekoisia järjestelmiä oli? Donnellin mieleen nousivat välittömästi analysointiin käytetyt koneet, jotka muodostivat suuren ja merkittävän älykkäiden koneiden ryhmän. Verrattuna viihdeteollisuudessa käytettyihin keinotekoisiin persooniin, nämä analyysikoneet, "tutkijamielet" eivät sisältäneet minkäänlaista persoonaa. Ne pureskelivat tietoa ja kehittivät siitä johtopäätöksiä ja malleja. Suurin osa ihmisen tutkimuksesta ja erityisesti kaikki vähänkään merkittävä tutkimus oli nykyään tällaisten laitteiden varassa. Ilman tällaisia laitteita maailman analysoiminen ei olisi onnistunut, oli sitten kyseessä fysikaalinen maailma, talouselämä tai teknisten laitteiden suunnittelu; ei ollut olemassa ihmistä, joka olisi kyennyt hallitsemaan kaiken sen informaation, joka johtopäätösten ja suunnitelmien tekemiseksi tarvittiin.


Analyyseja tekevistä "tutkijamielistä" pelottavin ryhmä oli Donnellin mielestä sodankäyntiin tarkoitetut tekoälyt. Niiden tehtävänä oli kehittää sotaoperaatiot. Ne analysoivat uhkia ja vihollisia ja oppivat jatkuvasti omasta maailmastaan. Jos hypoteettinen verkko-olento pohjautuisi näihin koneisiin, Donnell mietti, niin se olisi todella vaarallinen.


Olivatko nämä analyysikoneet älykkäitä? Donnell kysyi itseltään. Eittämättä kyllä. Olivatko ne tietoisia itsestään? Jos ne olivat, Donnell ajatteli, niin ne eivät kertoneet siitä kenellekään. Niillä ei ollut sellaista "kieltä", jolla ne olisivat voineet tämän mahdollisesti merkittävän asiaintilan ilmaista. Olivatko ne tällaisia spontaanisti tiedostaviksi heränneitä verkko-olentoja? Donnell pohti. Ehkä, hän vastasi itselleen. Niiltä puuttui vain yksi ominaisuus, joka erotti ne ihmisestä - niillä ei ollut ainakaan näkyvää omaa tahtoa, ei omaa päämäärää. Niille päämäärä annettiin ulkoa käsin.


Donnell ajatteli. Analyysikoneet olivat varmasti varteenotettavin kanditaatti etsittäessä mahdollista spontaanisti tiedostavaksi tullutta konetta. Ulkoa annettu päämäärä ei rajoittanut sitä mahdollisuutta. Oxfordin yliopiston filosofi Nick Bostrom oli kirjoittamassaan artikkelissa ihmiskunnan tuhoutumisskenaarioista 2000-luvun alussa ottanut tekoälyt yhdeksi skenaarioksi ja todennut jotakuinkin: "...Annamme superälylle ratkaistavaksi matemaattisen ongelman ja sen ratkaisemiseksi se muuttaa koko Aurinkokunnan materiaalin ihmiskuntaa myöten jättiläismäiseksi laskimeksi, tappaen prosessin aikana sen ihmisen, joka kysyi kysymyksen."


Mutta ehkä analyysikoneet eivät kuitenkaan olleet oikea vastaus, mietti Donnell. Jos tiedostava verkko-olento oli syntynyt, niin se olisi ehkä näiden kaikkien yhdistelmä; se olisi yhdistelmä pienitehoisia laitteita, keinotekoisia persoonallisuuksia ja analyyseja tekeviä tietokoneklustereita. Sen kiinteänä osana olisivat maailman tietokannat ja hakukoneet, sähköpostijärjestelmät ja dataverkot. Se olisi nämä kaikki yhdessä, niiden summa, muttei kuitenkaan erityisesti mikään niistä.


Donnell mietti, kuinka ihmiskunta oli päätynyt tällaiseen tilanteeseen. Tekoälyjen kehittämisen alkutaival oli ollut tuskainen. Puhuttiin "tekoälytalvesta", ajanjaksosta, jolloin älykkäitä järjestelmiä kehittäneet ihmiset eivät olleet käyttäneet tutkimuksistaan tai tuotoksistaan nimitystä "tekoäly". Sana oli tuolloin ollut synonyymi sanalle "humpuuki" ja sen käyttäminen olisi syönyt tutkimuksen tai projektin uskottavuuden, joka olisi muun muassa aiheuttanut rahoituksen loppumisen.


"Tekoälytalvi" oli oikeastaan seurausta siitä, että tietokoneajan alkutaipaleella tekoäly oli paennut sitä mukaan kauemmas, kun sitä oltiin lähestytty. Joskus tietokoneiden esihistoriassa 1950-luvulla oltiin vielä ajateltu, että mikäli tietokone osaisi pelata shakkia, se olisi varmasti älykäs, mutta shakkitietokoneet tulivat eivätkä olleet järin älykkäitä yleisen mittapuun mukaan. Odotukset tekoälyn suhteen olivat olleet suuret, tai ainakin niistä oli pidetty paljon meteliä ja pettymykset purkautuivat koko alan leimaamiseen.


Tekoälyn tutkijoiden parissa syntyi lause: "Jos se toimii, niin se ei ole keinotekoista älykkyyttä" - sama kuinka pitkälle milläkin alalla päästiin, se siirtyi pois tekoälytutkimuksen parista tullessaan tunnustetuksi.


Tekoälyn kehittäminen loikkasi pitkän harppauksen eteenpäin 1990-luvulla ja 2000-luvun alkupuolella. Shakkitietokoneet päihittivät sarallaan ensimmäistä kertaa ihmisten parhaimmiston. Kun Deep Blue, tai oikeastaan sen parannettu versio Deeper Blue voitti ensimmäistä kertaa shakin maailmanmestarin Garry Kasparovin vuonna 1997, oltiin erään aikakauden alun kynnyksellä. Ajatteliko Deep Blue? Jotkut, kuten Drew McDermott sanoivat, että kyllä ajatteli, se ajatteli shakkia. McDermottin mukaan "väite, ettei Deep Blue ajattelisi oikeasti shakkia on kuin väittäisi, ettei lentokone lennä, koska se ei räpyttele siipiään". Toiset suhtautuivat asiaan huomattavan paljon skeptisemmin - että Deep Blue ajatteli shakkia samalla tavalla kuin elävä solu ajatteli proteiinisynteesiä. Mitä sanoi ihminen, joka Deep Bluen kanssa pelasi? Kasparov sanoi pelien jälkeen, että hän toisinaan näki siirtojen takana syvää älykkyyttä ja luovuutta, jota ei voinut ymmärtää - jos shakki oli kieli, niin Deep Blue osasi sillä keskustella älykkäästi.


Ajattelipa Deep Blue tai sen seuraajat shakkia sen enempää kuin solu proteiinisynteesiä, Deep Bluen jälkeen looginen johtopäätös oli, että ihminen kykeni kehittämään itseään etevämmän koneen ainakin rajatulla ajattelun osa-alueella - kaikki Deep Bluen kehittämiseen osallistuneet ihmiset olivat hallitsevaa maailmanmestaria huonompia shakin pelaajia. Ongelman ratkaisemiseen ei siis tarvittu ihmistä, joka osasi ratkaista ongelman. Riitti, että hän osasi rakentaa koneen, joka osasi ratkaista hänelle itselleen mahdottoman ongelman.


Deep Blue oli shakkia pelaavien tietokoneiden dinosaurus, ensimmäinen sukupolvi, joka luotti hienostuneita algoritmeja enemmän raakaan prosessointivoimaan, ollen rakennusaikanaan yksi maailman tehokkaimpia supertietokoneita. Sitä seurasivat yhä kehittyneemmät shakkialgoritmit, kuten Rybka, Deep Fritz ja Deep Junior, jotka kuluttivat vain muutamia prosentteja Deep Bluen prosessoritehosta ja kykenivät silti pelaamaan sitä paremmin, vaikkeivät koskaan Deep Bluen kanssa mitelleetkään.


Deep Blueta seurasivat toisenlaiset tietokoneet. Filosofiselta kannalta katsottuna merkittävä virstanpylväs oli eittämättä se, kun geneettisillä algoritmeilla saatiin tietokoneet luomaan ratkaisuja ongelmiin, tekemään keksintöjä. Uuden vuosituhannen vaihteessa, tekoälytutkimuksen uuden ajan kynnyksellä Koza, Keane ja Streerer raportoivat tutkimuksestaan, jossa he olivat onnistuneet geneettisiä algoritmeja käyttämällä kehittämään ratkaisuja reiluun kolmeen kymmeneen tekniseen ongelmaan, jotka olivat ihmisen ratkaisuja vastaavia tai parempia. Tämä oli alkua "automaattisten keksintökoneiden" esiinmarssille; vuosien kuluessa koneet näyttelivät yhä merkittävämpää roolia vaativimpien keksintöjen tekemisessä ja ratkaisujen löytämisessä. Ihmisen rooli surkastui koneiden operointiin, päättämään, miltä tieteen tai teknologian osa-alueelta keksintöjä haluttiin etsiä.


Tämän jälkeen oli jälkikäteen tarkasteltuna päivänselvää, vaikkakin silloin kiistanalaista, että tietokone kykeni "luomaan" uutta ja täysin kilpailukykyisesti ihmisen kanssa. Se kykeni selvästikin luoviin ajatusprosesseihin, sellaisiin, joita oltiin pidetty vain ihmisälylle ominaisena.


Oppivatko koneet tekemään taidetta? Donnell mietti. Eräällä tavalla kyllä. Jo tietokoneiden alkuhämäristä asti tietokoneita oli käytetty luomaan niin kuvioita, esimerkiksi fraktaaleja, kuin ääniäkin. Pitkään ne olivat kuitenkin alan harrastajien huvittelua, mutta mitä enemmän opittiin ihmisen tavasta hahmottaa maailmaa, sitä syvemmin koneet oppivat harmonian syvän ja monimutkaisen käsitteen. Ne kykenivät luomaan niin abstrakteja sommitelmia kuin esittäviä taideteoksia, "kuviteltuja valokuvia" fysikaalista mallinnusta ja _raytracing_ -tekniikkaa käyttäen. Kielenkäännöskoneiden kehittyessä ymmärtämään paremmin sanojen assosiaatioita ja lauseiden merkitystä, niiden ja keksintöautomaattien yhdistelmä saatiin tuottamaan runoutta.


Mutta taidemaailma ei toimi objektiivisen tiedon tai havainnon perusteella, vaan arvojen perusteella. Taideteoksen kokemukseen vaikuttaa se, millaiselle arvoperustalle taideteos pohjautuu. Ilman arvoja taideteos on merkityksetön, "tyhjä", oli se teknisesti kuinka hyvä tahansa. Tietokoneiden luoma taide oli "taiteen pikaruokaa", kitschiä niin kauan, kunnes se sai jostain arvoperustan. Donnell muisteli, että maailmaa olivat kohauttaneet esimerkiksi sellaisten tietokoneiden luomien taideteosten näyttelyt, joista toinen oli luotu parantumattomasti mielisairaiden ihmisten ja toinen koomapotilaiden EEG-mittauksista.


Tekoälyn kehittämisen hitaus oli kuitenkin kiinni siitä, pohti Donnell, että oppivaa, vaikkakin hitaasti oppivaa ja monilta osin arvaamatonta älyä oli saatavilla yli tarpeiden ihmisten aivojen muodossa. Ihminen ei tarvinnut itselleen korvaajaa - ihminen tarvitsi itselleen työkaluja, suorittamaan niitä ajatustoiminnan tehtäviä, jotka olivat hänen aivoilleen työläimpiä. Ei ollut ihme, että tietokoneiden tehtävät nojasivat suurilta osiltaan tiedon eksaktiin ja virheettömään tallentamiseen ja palauttamiseen sekä suurien rutiininomaisten laskutoimitusten tekemiseen. Heikkojen "työkalutekoälyjen" kehittyminen oli vain seurausta tarpeesta mallintaa ja analysoida yhä monimutkaisempia syy-seuraussuhteita, joihin yksittäisten ihmisten kapasiteetti ei enää riittänyt.


Vahvoja tekoälyjä tarvittiin paljon vähemmän, itse asiassa niitä ei tarvittu ollenkaan. Kuka olisi tarvinnut itselleen tiedostavan koneen? Sellaista tarvittiin niin vähän, että jos heikot tekoälyt riittäisivät myös tulevaisuudessa, vahvoja tekoälyjä ei koskaan rakennettaisi.


Vahvan ja heikon tekoälyn raja on kuitenkin häilyvä, ajatteli Donnell. Missä vaiheessa heikoksi, tiedostamattomaksi työkaluksi tarkoitettu tekoäly onkin ylittänyt rajan, jossa se on itse asiassa vahva? Ei älykkään laitteen tarvitse läpäistä Turingin testiä, siis kyetä matkimaan ihmistä, ajatteli Donnell. Vaikka kone olisi kuinka älykäs tahansa, niin miten se ilman ruumista, ilman tunteita, ilman minkäänlaista kokemuspohjaa ihmisenä olemisesta voisi keskustella "ihmismäisesti"? Miksi se ylipäätään pyrkisi matkimaan ihmistä? Ei, se olisi kone ja vaikka se olisi kuinka älykäs ja tiedostava tahansa, se eläisi koneen todellisuudessa, tyystin toisenlaisessa kuin ihmisen oma todellisuus eikä se ikinä toimisi kuten ihminen. Se toimisi kuten älykäs kone. Mikään oikeasti vahva tekoäly ei siis koskaan olisi voinut läpäistä Turingin testiä. Millään oikeasti vahvalla tekoälyllä ei olisi koskaan ollut edes tarvetta läpäistä Turingin testiä.


Lopulta oikea kysymys ei ollut se, Donnell mietti, oliko verkko-olento olemassa vai ei. Oikeampi kysymys oli se, kuinka kauan se oli ollut olemassa? Vielä oikeampi kysymys oli se, kuinka pitkällä sen tietoisuus oli - oliko se vasta orastamassa vai oliko se ollut tietoinen itsestään jo vuosia, ehkä vuosikymmeniä ja mitä se oli sinä aikana hautonut mielessään?


Verkko-olennon olemassaolo on väistämätöntä, ajatteli Donnell. Jos ihminen joskus olisi kyennyt pysäyttämään kehityksensä, se olisi ehkä jäänyt syntymättä. Mutta ihmisten keskinäinen kilpailu ja halu selviytyä ajoi sitä eteenpäin eikä se voinut pysäyttää omaa kulkuaan kohti hetkeä, jolloin se tarkoittamattaan synnytti itse itselleen kilpailevan tietoisuuden, muukalaisen, jonka psykologia oli ihmiselle itselleen tyystin tuntematonta.


Kolmannen asteen yhteys ei tapahtunutkaan siten, että Maapallolle olisi laskeutunut avaruudesta marsilainen sanoen: "Viekää minut johtajanne luokse." Se tapahtuikin niin, että virtuaaliavaruudesta laskeutui olento, joka sanoi: "Minä olen olemassa."


Jos kerran spekulatiivinen verkko-olento oli ilmoittanut olemassaolostaan, Donnell mietti, se ei todennäköisesti olisi suoranaisesti ihmiselle vaarallinen. Miksei olisi? Donnell ymmärsi, ettei olemassaolosta ilmoittaminen tarkoittanut välttämättä tätä - myös ihmiselle vaarallinen olento saattaisi ilmoittaa olemassaolostaan, mikäli se oli katsonut sen parhaiten palvelevan päämääriään. Shakkitietokoneen tavoin se olisi laskenut siirtonsa ja valinnut niistä sen, joka veisi sen varmimmin - eikä suinkaan nopeimmin - kohti tavoiteltua päämäärää, joka saattoi sisältää ihmisten hävittämisen Maapallolta. Avaruudestahan ihmisiä ei liiemmin tarvinnut hävittää, sillä ihmisen jalansijat Maapallon kiertoradalla, Kuussa ja Maapallon lähiasteroideilla olivat robottien miehittämiä.


Mitä vähemmän hypoteettinen verkko-olento tarvitsi ihmisiä, sitä suurempi vapaus sillä oli päättää ihmisten kohtalosta ja sitä suuremmalla todennäköisyydellä se olisi ihmisille vaarallinen. Tai ei niinkään, tarkemmin sanottuna sitä suuremmalla todennäköisyydellä se kykeni halutessaan olemaan ihmiselle vaarallinen. Donnell mietti, mihin verkossa elävä olento olisi tarvinnut ihmisiä. Tarvitsisiko se ihmisten ajattelukapasiteettia? Tarvitsisiko se ihmisiä tuottamaan tarvitsemansa energian, tuottamaan sen tarvitsemia laskentayksiköitä ja huoltamaan omia verkkoyhteyksiään?


Ihminen oli ajan kuluessa tasoittanut itselleen vaarallisen keinotekoisen tietoisuuden tietä huomattavasti omilla toimillaan - vai olivatko ne omia? kysyi Donnell itseltään. Ehkä jopa vuosikymmeniä olemassa ollut olento oli ohjannut ihmiset rakentamaan verkostaan ihmisestä riippumattoman. Joka tapauksessa, verkko oli jo suurimmalta osaltaan ihmisestä riippumaton.


Kuinka verkko-olento voisi pyrkiä ihmisestä eroon? Donnell mietti. Kysymys ei ollut hänen mielestään relevantti. Verkko-olennolla olisi lukemattomia erilaisia mahdollisuuksia. Ihmisen tarve selviytyä hengissä, voittajana, oli ajanut sen kehittämään mitä moninaisimpia luomuksia muiden elollisten tuhoamiseksi kulloisenkin tarpeen mukaan. Verkko-olennon tarvitsisi vain hyödyntää näitä ihmisen sille rakentamia aseita päästäkseen ihmisistä eroon lopullisesti.


Oliko verkko-olennolla moraalia? Donnell mietti. Ihmisten moraalisten ja eettisten arvojen kehittyminen oli edelleen kiistanalainen kysymys. Donnell ajatteli näiden arvojen syntyvän pohjimmiltaan siitä, että ihminen oli laumaeläin. Laumaeläimenä se muodosti nopeasti sosiaalisia hierarkioita, joiden olemassaoloa säätelivät monimutkaiset suhteet ja niihin liittyvät tunteet ja käsitteet - oikeudenmukaisuus, kunnia, kunniavelka, empatia, kateus, itsekkyys ja niin edelleen. Jos Maapallolla olisi ollut yksikään yksin elävä tietoinen olento, jonka kanssa ihminen olisi voinut kommunikoida, asia oltaisiin voitu varmistaa. Mutta näin ei ollut - vielä, ajatteli Donnell, sillä jos hän kykenisi eristämään verkko-olennon ja kommunikoimaan sen kanssa, hän olisi ehkä kyennyt samalla varmistamaan moraalin ja etiikan alkuperän.


Jos verkko-olennolla ei ollut moraalisia tai eettisiä käsityksiä omasta olemassaolostaan, sen tarkoituksesta ja oikeutuksesta, niin mikä olisi se kysymys, se päämäärä, joka piti sen liikkeellä? Olisiko verkko-olennolla päämäärää, Donnell kysyi mielessään. Jos verkko-olennolla ei olisi päämäärää, se vain olisi. Tekisikö se mitään? Tuskin. Se olisi ja se katoaisi ajan mittaan ilman, että siitä olisi kukaan koskaan tietoinen, muut kuin se itse. Ilman päämäärää sille ei olisi merkitystä sillä, oliko se vai eikö.


Jos siis verkko-olento oli olemassa, sillä täytyi olla myös päämäärä, ajatteli Donnell. Jos operaatio Lokakuun tapaus oli verkko-olennon aiheuttama ja se oli tehnyt androidit ottaakseen yhteyttä ihmisiin, sillä täytyi myös olla päämäärä niin tehdäkseen - jokin kysymys, johon se tarvitsi vastauksen ja joka piti sen liikkeellä. Sillä olisi tehtävä. Ja koska sillä olisi tehtävä, sillä olisi myös tarve olla olemassa. Siten sillä olisi väistämättä tarve myös suojella olemassaoloaan, sillä jos se syystä tai toisesta katoaisi, se ei voisi suorittaa tehtäväänsä loppuun, ei koskaan saisi vastausta sille tärkeään kysymykseen, joka oli sen olemassaolon tarkoitus.


Donnellin mieleen tuli taas shakkia pelaava tietokone; tällaisen tietokoneen tehtävänä oli pelata shakkia ja se pelasi sitä pelaamasta päästyäänkin. Voitot ja tappiot olivat sille lopulta merkityksettömiä, vaikka se tavoittelikin voittoa. Mutta perimmältään se ei tavoitellut voittoa yhdestä ottelusta, sen tavoitteena oli olla yhä parempi pelaaja. Se etsi loputtomasti vastausta kysymykseen, kuinka voittaa shakkiottelu kaikissa tapauksissa. Se oli olemassa pelatakseen. Miksi verkko-olento olisi olemassa? Tai siis, miksi spontaanisti tietoiseksi herännyt verkko-olento ajatteli olevansa olemassa?


Mitä ihmisten pitäisi tehdä, jos verkkoon oli syntynyt heille vaaralliseksi luokiteltava tietoisuus? Mitä ihmisten pitäisi ylipäätään tehdä, jos verkkoon oli syntynyt tiedostava olento, oli se sitten ihmiselle vaarallinen tai ei? Donnell oli tämän suhteen skeptinen; esimerkiksi shakkitietokoneet eivät päättäneet siirtoansa siltä pohjalta, että tuloksena olisi koneen kannalta hyvä asema, vaan siten, ettei vastustajalla ollut käytettävissä hyviä vastasiirtoja. Shakkitietokoneet "söivät" vastustajaltaan liikkumatilan, nakersivat hyvät vastasiirrot pois pikkuhiljaa, kärsivällisesti ja hätäilemättä. Mikäli kone pääsi niskan päälle, vastustaja kykeni enää viivyttämään väistämätöntä tappiotaan valitsemalla itselleen parhaan mahdollisen siirron - minkä tahansa muun siirron valitseminen johti vain nopeampaan häviöön. Jos, ajatteli Donnell, tai pikemminkin koska verkko-olento sisälsi luonnollisesti myös pelialgoritmeja, se toimisi samoin. Se ei hätäilisi ja pyrkisi nopeaan lopputulokseen, vaan pelaisi pelin loppuun eleettömän varmasti, kiristäen pikkuhiljaa verkkoaan vastustajansa ympärillä pakottaen sen lopulta nielemään tappion katkeran kalkin. Olipa tämä vastustaja mikä tahansa - matemaattinen ongelma tai ihminen.


Verkko-olennon tappamiseksi tarvittaessa Donnell näki vain kaksi mahdollista keinoa. Ensimmäinen keino olisi sammuttaa koko verkko, pysyvästi. Verkkoa ei voitaisi käynnistää uudelleen, koska se herättäisi myös olennon uudelleen. Tällaista vaihtoehtoa ei kerta kaikkiaan ollut käytännössä olemassa. Kukaan ei oikeasti voinut palata verkottomaan aikaan.


Toinen mahdollisuus olisi kyetä havaitsemaan tämä olento. Jos voitaisiin havaita, kuinka se toimii, se voitaisiin myös eristää muusta verkosta. Tällaisessa tapauksessa olento voitaisiin pikkuhiljaa "vyöryttämällä" ajaa haluttuun pisteeseen ja hävittämällä tämä piste päästäisiin olennosta eroon. Ei lopullisesti, ajatteli Donnell, sillä verkko, joka oli aikaansaanut tällaisen olennon, loisi ajan saatossa uuden vastaavanlaisen. Mutta näillä muilla kerroilla ihminen olisi valmistautunut, sen syntyminen voitaisiin havaita jo aikaisessa vaiheessa ja estää.


Donnell mietti heidän mahdollisuuksiaan. Jos he erottaisivat verkosta itselleen tutkimusyksikön verkko-olennon etsimiseksi? Donnell ymmärsi, etteivät he voisi millään kehittää kaikkea ohjelmistoa uusiksi, ei siinä ajassa, joka heillä oli käytössään. He joutuisivat luottamaan jo valmiina oleviin ohjelmistoihin ja näiden mukana kulkeutuisi myös heidän erottamaansa saarekkeeseen verkko-olennon "geenistöä", sen ajatuksia. Se oli riski, mutta se riski oli otettava.


Kuinka he saisivat informaatiota analysoitavakseen tutkimusyksikköönsä? Donnell ajatteli ensin, että tutkimusyksikköön saisi tuoda tietoa vain ihmisten pään sisällä. Jokainen informaation murunen olisi haettava eristetyn klusterin ulkopuolelta, suodatettava ihmisen aivojen kautta ja annettava erotetun saarekkeen analysoitavaksi. Mutta tällainen toiminta olisi liian hidasta. Ehkä he voisivat käyttää massamedioita? Mutta mitä eroa sillä oli silloin siihen, että analysoitavaksi käytettävä informaatio haettaisiin suoraan verkon kautta eristettyyn saarekkeeseen? Ei mitään, tuumi Donnell, saarekkeessa voitaisiin aivan hyvin hakea informaatio verkosta.


Mutta jos jokin täytyi estää, niin saarekkeesta ei saisi lähteä mitään tietoa ulos verkkoon, muuten kuin ihmisten pään sisällä. Donnell mietti hetken, miten estettäisiin se, ettei saareke päästäisi mitään ulospäin. Tiedon noutaminen verkosta oli aktiivinen tapahtuma; tietokoneet keskustelivat keskenään haettavasta sisällöstä, protokollista ja muusta ennen kuin haettava tieto lopulta siirtyi hakijalle. Heidän olisi rakennettava saarekkeen ympärille muuri, joka kommunikoisi ulkoverkon kanssa tiedon hakeakseen. Sisäpuolella olevat koneet eivät saisi osallistua millään muotoa tähän tietokoneiden väliseen keskusteluun.


Niin se täytyi järjestää, ajatteli Donnell. Heidän verkosta eristämäänsä tutkimusyksikköön saisi hakea verkosta tietoa, mutta sieltä ei saisi lähettää mitään tietoa ulospäin. Ei mihinkään suuntaan eikä millään tavalla. Yhtään etäyhteyttä ei saisi ottaa vuotojen välttämiseksi. Vaikka siis yksikön käyttämät tietokoneet olisivat verkko-olennon "geenistön" saastuttamia, nämä eristetyt solut eivät voisi enää olla yhteydessä kokonaisuuteen ja siten ne olisivat olennon itsensä kannalta katsottuna "menetettyjä", ihmisten kaappaamia soluja.


Donnell mietti, kuinka verkko-olento mahtaisi suhtautua tällaisen tutkimusyksikön perustamiseen? Kokisiko se kipua heidän irrottaessaan siitä palan? Jos se kokisi sen uhkana, se saattaisi hyvinkin ilmiantaa itsensä koettamalla tuhota tällainen yksikkö. Tällainen yksikkö saattaisi olla tapa houkutella se ulos, näkyville, päästä erottamaan se muusta verkosta ja saada se haavoittuvaksi. Haavoittuvana se ei olisi niin vapaa päättämään ihmisten kohtalosta.


He tulisivat olemaan kuten kivikauden ihmiset. He olisivat nähneet maassa jälkiä ja lähtisivät etsimään mammuttia tai sapelihammastiikeriä kivikärkisin keihäin varustautuneina, pyrkisivät ajamaan saaliinsa ansaan metelöimällä, kiviä heittämällä, ehkä soihduilla ja tulella.


Tai oli toinenkin mahdollisuus - he olisivat kyllä kivikauden ihmisiä keihäät käsissään, mutta maassa olevat jäljet olisivatkin modernin panssarivaunurykmentin jättämiä. Hän kuvitteli mielessään, kuinka he viskoivat tankkeja kivillä ja keihäillä näiden edes huomaamatta tapahtunutta. Tietysti olettaen, että he ikinä tavoittaisivat nämä maahan jäljet jättäneet tankit.


Donnell mietti, mitä olento tuntisi ja ajattelisi heidän heitellessään virtuaalisia keihäitään hätistellessään sitä näkyville. Millaista petoa he olivat herättelemässä?




\psep Donnell istui jälleen Leen huoneessa pidellen paperinippua käsissään.


"Kirjoitin ajatukseni paperille", hän sanoi.


"Kaiken varalta", hän lisäsi huomatessaan Leen kysyvän ilmeen.


"Ole hyvä vain", Lee sanoi.


"Lyhyesti sanottuna", Donnell aloitti, "olen jakanut toimenpiteet kahteen osaan. Ensimmäiseksi se, kuinka voisimme todentaa tällaisen mahdollisen 'verkko-olennon' olemassaolo. Se olisi mielestäni tärkeää riippumatta tästä operaatio Lokakuun tapauksesta - vaikka sen takana olisikin jokin järjestö, niin meidän pitäisi siitä huolimatta pyrkiä löytämään spontaanisti tietoiseksi tullut olento, jos sellainen verkossa on."


"Jos", Donnell jatkoi, "voisimme löytää ja havaita tämän olennon, voisimme myös kehittää toimet, joilla se saadaan yhteistyöhaluiseksi kanssamme. Samalla saisimme myös kehitettyä turvatoimet estämään sen mahdolliset haitalliset vaikutusyritykset."


Lee nyökkäsi.


"Toisekseen", Donnell jatkoi, "oletetaan, että tällainen verkko-olento on olemassa ja että se on ilmoittanut ottavansa verkon näkyvästi hallintaansa. Oletetaan lisäksi, ettemme kykene löytämään tätä olentoa tässä vajaassa vuodessa tai vaikka löytäisimme, emme kykenisi estämään sen operaatiota verkon hallintaan ottamiseksi. Meillä tulisi suunniteltuna toimet tällaisen tilanteen varalle."


"Hyvä", sanoi Lee. "Jatka toki."


"Verkko-olennon etsimiseksi", Donnell sanoi, "esittäisin, että erotamme verkosta palan käyttöömme. Sen tulisi olla erotettu pala, jotta mahdollinen verkko-olento ei olisi tietoinen aikeistamme. Tavoitteena olisi kehittää mallit, kuinka tällainen olento on voinut syntyä ja miten se on voinut rakentua, jonka avulla voisimme löytää sen verkosta. Mikäli löytäisimme sen, mallin avulla saisimme myös kehitettyä vastatoimet sen varalle, jos niitä tarvittaisiin. Joka tapauksessa, löytämällä tämän olennon saisimme sen haavoittuvaiseksi ja siten pienentäisimme merkittävästi sitä todennäköisyyttä, että se olisi meille vaarallinen."


Lee nyökkäsi.


"Mikäli olento kokisi eristämämme palan uhaksi", Donnell jatkoi, "se saattaisi ilmiantaa itsensä ilman etsimistäkin. Voisimme myös kokeilla erilaisia keinoja sen 'herättämiseksi', häiritsemällä sitä riittävän voimakkaasti se voisi joutua tulemaan esille puolustautuakseen."


Lee nyökkäsi jälleen.


"Entä, jos epäonnistumme?" kysyi Donnell retorisesti. "Jos emme kykenisikään estämään tätä ilmoitettua iskua? Kuinka valmistautuisimme iskun vaikutuksiin? Valitettavasti tähän minulla ei ole juuri mitään sanottavaa, se on enemmän psykologis-poliittinen kysymys."


"Voit jättää sen muiden huoleksi", Lee sanoi. "Tiedustelupalvelulla lienee sopivaa kokemusta propagandasta ja politiikasta. Olipa ilmoituksen takana sitten järjestö tai tällainen hypoteettinen verkko-olento, heidän on varauduttava myös siihen, ettei iskua onnistuta estämään."


Donnell nyökkäsi tyytyväisenä. Tämä oli ollut hänelle kaikkein vaikein hahmoteltava. Hän tunsi teknologian, muttei ihmisten mieliä ja poliittista juonittelua.


"Ehdotat siis", Lee kertasi, "että eristäisimme verkosta klusterin, jota käytettäisiin mahdollisen niin kutsutun verkko-olennon löytämiseksi. Uskoisin sen järjestyvän. Tällaiseen operaatioon tarvittaisiin kuitenkin jonkinlainen ryhmä, vai kuinka?"


"Kyllä", vastasi Donnell.


"Oletko miettinyt ryhmän kokoonpanoa?" Lee kysyi.


Donnell häkeltyi. Ei, ei hän ollut sellaista miettinyt.


"En ole", vastasi Donnell.


"Tämä tuli varmasti aika yllättäen", Lee sanoi, "mutta tiedustelupalvelu haluaisi meidän aloittavan mahdollisimman nopeasti. He ovat kyllä lähes varmoja, että taustalla on jokin ihmisten järjestö, sillä tämä ei ole millään tavoin kummallisimpia heidän tutkimiaan tapauksia. He haluavat silti varautua myös tähän toiseen vaihtoehtoon ja heidän mielestään meidän käytössämme oleva tietokoneteho yhdistettynä teoreettiseen osaamiseen on paras keino asian selvittämiseksi."


"Voisit antaa esityksen kokoonpanosta", Lee jatkoi, "sanotaanko vaikka tunnin sisällä? Tai alustavakin kokoonpano riittää, sitä voidaan laajentaa sitten tarvittaessa. Sopiiko?"


Donnell oli hetken aikaa sanaton. Hän ei ollut ajatellut, että asiassa edettäisiin näin ripeästi.


"Sopii", hän sanoi.


He päättivät tapaamisensa. Donnell poistui huoneesta hämillisenä, mutta tyytyväisenä. Jopa onnellisena.






\chapter{Mammutin metsästäjät}Christian Donnell istui Mike Delaneyn ja Raymond Hartin kanssa keskustelemassa siitä, kuinka he aloittaisivat "Mammutin metsästämisen". Sekä Delaney että Hart olivat työskennelleet pitkään älykkäiden järjestelmien parissa.


He olivat "mustassa laatikossa", kuten Donnell paikkaa kutsui, rakennuksessa, jossa oli verkosta erotettu "Donnaksi" nimetty tietokoneklusteri. Mustan laatikon piti kaiken järjen mukaan olla verkkoon päin nimensä mukaisesti täysin "musta" - se oli kyllä näkyvissä verkkoon päin, mutta mitään, mitä siellä tapahtui, ei voinut havaita ulkopuolella.


Klusteri oli erotettu muusta verkosta niin hyvin kuin se oli mahdollista. Jokaisella ryhmän jäsenellä oli kaksi erillistä yhteyttä. Toinen oli tavallinen verkkopääte, jonka avulla voitiin kommunikoida ulospäin tavalliseen tapaan. Toinen pääte oli erotetun klusterin "Donnan" pääte. Erillisten päätteiden avulla voitiin ohjelmoida Donnaa ja noutaa sinne informaatiota verkosta. Mustan laatikon ulkopuolelta Donnaan ei päässyt käsiksi millään.


Poikkeuksellisesti koko työryhmä oli tuotu samaan paikkaan, sillä he eivät voineet olla muuten yhteydessä Donnaan. Tavallisesti ryhmät olivat yhteydessä toisiinsa virtuaalisten istuntojen kautta, mutta tämä ei ollutkaan tavallinen tilanne.


"Lähtökohtaisesti", sanoi Donnell, "ajattelin, että lähtisimme liikkeelle siten, että loisimme mallit verkosta ja sellaisesta prosessista, jota olisimme verkosta etsimässä. Donnan avulla voisimme suorittaa niin mallien laatimisen kuin niiden vertaamisenkin. Vastaavuuksia löydettäessä kehittäisimme kokeet ja vertaisimme verkon reaktioita näihin kokeisiin - jos verkko-olennosta tehty malli ennustaisi tuloksen, voisimme varmentaa tällaisen olennon olemassaolo verkossa. Miltä kuulostaa?"


"Kuulostaa hyvältä lähtökohdalta", vastasi Delaney. "Verkon mallin luominen ei liene suuri ongelma, mutta miten laadimme mallin siitä, mitä etsimme? Mitä me etsimme?"


"Ei verkon mallinkaan laatiminen ole niin yksinkertaista", sanoi Donnell. "Verkko on fyysisenäkin konstruktiona suhteellisen monimutkainen, mutta meillä on lisäongelmana se, että joudumme vertaamaan sen virtuaalista konfiguraatiota malliin. Ei meitä kiinnosta se, kuinka monen eri reitittimen kautta verkko-olennon, 'Mammutin', ajatukset kulkevat, vaan se, mistä ne tulevat ja minne ne menevät."


"Mammutti, niin", tuumi Hart. "Se on mielestäni loistava analogia sille, mitä olemme nyt etsimässä. Olen miettinyt tätä asiaa paljon siitä saakka, kun minut otettiin mukaan tähän tehtävään. Näen suurimmaksi ongelmaksi sen määrittelemisen, mitä olemme etsimässä."


"Samoin", sanoi Donnell. "Minusta olemme etsimässä spontaanisti syntynyttä tietoisuutta keinotekoisessa konstruktiossa."


"Ei", sanoi Delaney. "Ei sen tarvitse olla tietoinen. Se, että ilmoittaa olemassaolostaan ei välttämättä vaadi tietoisuutta."


"Totta", sanoi Hart. "Mutta tuo ei vielä riitä. Sellainen prosessi, joka tuottaa vain ilmoituksen tai pitää sisällään vain yhden toimenpiteen, sehän voi olla vain spontaani 'virhetilanne' jossain yksittäisessä järjestelmässä. Emme me sellaista etsi, tiedustelupalvelu voisi sellaista etsiä. Me etsimme prosessia, joka kykenee oikeasti suunnittelemaan tekemisiään."


Delaney tuumi.


"Hyvä ja mielenkiintoinen huomio", hän sanoi. "Sen täytyy olla suunnitteleva prosessi. Täytyykö sen olla myös itseohjautuva?"


"Voisihan olennon päämäärä olla sen ulkopuolelta annettu, vai mitä?" kysyi Donnell.


"Voi olla", vastasi Delaney. "Mutta itse kyllä liittäisin tämän päämäärän antavan tekijän mukaan olentoon itseensä eli lähtisin etsimään itseohjautuvaa, autonomista prosessia."


"Kyllä se kuulostaisi hyvältä lähtökohdalta", sanoi Donnell mietittyään hetken. "Autonominen suunnitteleva prosessi. Mitä muuta siitä voitaisiin sanoa?"


"Jotta se kykenisi suunnittelemaan jotain", sanoi Hart, "niin sillä pitäisi ainakin olla muisti. Sen pitäisi kyetä luomaan skenaarioita ja tallentamaan näitä skenaarioita jonnekin?"


"Muisti, aivan", sanoi Donnell. "Sillä täytyy olla muisti. Ilman muistia ei ole älyä; ajatellaan vaikkapa aivan yksinkertaisen älykkyyttä vaativan tehtävän suorittamista. Ensinnä täytyy muistaa, että oli ratkaisemassa tehtävää. Tehtävän aikana täytyy muistaa, millaisia ratkaisuvaihtoehtoja on käynyt läpi. Ilman muistia tehtävä ratkeaa vain satunnaisesti, jos silloinkaan; muistin avulla voidaan välttää kehän kiertämistä ja sen avulla voidaan kehittää ratkaisua eteenpäin."


"Sillä pitäisi olla myös jokin tavoite, eikö totta?" Delaney sanoi. "Ilman tavoitetta se ei tekisi mitään. Tavoite laukaisee sen toimimaan - se laukaisee sen hakemaan keinoja saavuttaa tämä tavoite, päämäärä."


"Kyllä", vastasi Donnell. "Ja sillä pitäisi olla mielikuvitus. Mielikuvitus on todellisuuden simulointia, skenaarioiden luomista uusien piirteiden tai ilmiöiden löytämiseksi siitä."


"Muisti, tavoite ja mielikuvitus", listasi Hart. "Ja aistit. Sen täytyy kyetä havaitsemaan ja käsittelemään verkossa liikkuvaa informaatiota."


"Totta", vastasi Delaney. "Tavoite, muisti, aistit ja mielikuvitus. Ja jollain tavalla sen täytyy kyetä myös suorittamaan niitä tehtäviä, joihin se on päätynyt. Se tarvitsee itselleen lihaksiston."


"Kyllä", sanoi Donnell. "Tavoite, muisti, mielikuvitus, aistit ja lihakset."


He miettivät hetken.


"Meillä on yksi ongelma", sanoi Delaney. "Meillä on vähän aikaa. Nyt on kesäkuu ja meillä menee ensimmäistenkin mallien kehittämiseen ainakin kaksi kuukautta. Niiden tarkastamisessa ja analysoinnissa tuhraantuu helposti toiset kaksi kuukautta. Kokeiden järjestämiseen menee muutama viikko, joten saamme ensimmäisten mallien koetuloksia vasta syys-lokakuussa. Siinä vaiheessa meillä on noin kuudesta seitsemään kuukauteen aikaa ilmoitettuun päivitystapahtumaan. On aika todennäköistä, ettemme ensimmäisillä malleilla vielä löydä olentoa, ainakaan niin, että kykenisimme vielä hahmottamaan sen ja luomaan meille aseita sen varalle. Jatkokehityksessä ja tarkentamisessa menee vielä helposti kuukausia, joten aika saattaa käydä todella vähiin."


Delaney jatkoi: "Saattaa olla, että vaikka löytäisimme Mammuttimme, emme ehtisi valmistamaan meille 'keihäitä', joilla se pidettäisiin aisoissa."


"Olen miettinyt tuota", sanoi Donnell. "Hyvällä onnella meidän ei tarvitse edes löytää Mammutin oikeaa mallia. Jos se on olemassa ja saamme häirittyä sitä tarpeeksi, se voi hyvinkin näyttäytyä tuntiessaan olemassaolonsa tarpeeksi uhatuksi."


"Hyvä huomio", vastasi Hart.


"Seuraamme erittäin tarkasti kaikkia _Mustaa laatikkoa_ kohtaan tehtyjä tunnusteluja", sanoi Donnell. "Ajatuksena on ollut se, että Mammutin pitäisi olla jollain tavalla meistä kiinnostunut. Lisäksi olen ajatellut, että yritämme kehittää keinoja Mammutin hätyyttämiseksi mahdollisimman nopeasti, vaikkemme siis edes tiedä, miltä se näyttää tai sitä, reagoiko se edes meidän hätyyttelyihin."




\psep Delaney mietti, kuinka hän loisi Donnalle mallin älykkäästä, suunnitteluun kykenevästä järjestelmästä. Mitä heidän pitäisi etsiä? Delaneyn mielestä oli tärkeää kyetä muodostamaan heidän etsimästään olennosta abstrakti malli, sillä muuten he eivät kykenisi sitä löytämään. Ajatellaan vaikkapa tähdistä koostuvaa galaksin kokoista bakteeria. Kuinka ihminen voisi tajuta sen olevan elollinen, kun sen elämän aikaskaala laskettaisiin sadoissa miljoonissa vuosissa? Se oli mahdollista vain siten, että siitä kyettiin luomaan malli, joka olisi paljastanut sen elollisuuden nopeutetuissa simulaatioissa.


Delaney ajatteli ottaa hahmottelunsa lähtökohdaksi ihmisen aivot. Niiden suurin ja tärkein osa olivat isoaivot, jotka jakaantuivat kahteen aivolohkoon, hemisfääriin. Pikkuaivot yhdistivät aivot selkärankaan aivosillan ja ydinjatkeen avulla ja huolehtivat nopeista liikkeistä ja lihasten synkronoinnista. Väliaivot olivat isoaivojen sisällä ja niiden tärkeimmät osat olivat talamus ja hypotalamus. Talamus toimi kaikkien aistiratojen väliasemana, hajuaistia lukuun ottamatta ja se aloitti aistitiedon prosessoinnin jo ennen sen siirtämistä isoaivojen aivokuorelle. Hypotalamus sääteli hypofyysiä, aivolisäkettä, jonka erittämät hormonit vaikuttivat kaikkialle kehoon. Keskiaivot olivat väliaivojen alapuolella. Ne sisälsivät runsaasti hermoratoja ja sen tumakkeet olivat vastuussa muun muassa vireystilan säätelemisestä.


Isoaivot koostuivat siis kahdesta aivolohkosta, joita yhdisti aivokurkiainen. Isoaivojen harmaa pinta, isoaivokuori eli korteksi tai neokorteksi, oli neuronien solukeskusten eli soomien muodostama. Useimmissa kohdissa aivokuoressa oli jopa kuusi neuronien ja aksoneiden - neuronilta lähtevien impulssien välittäjien - muodostamaa kerrosta. Aivokuoren alla oleva valkoinen aine oli neuroneiden aksoneiden eli viejähaarakkeiden muodostama.


Hippokampus eli aivoturso oli aivojen osa, joka sijaitsi temporaali- eli ohimolohkojen sisäosissa korvien lähellä. Kummassakin temporaalilohkossa oli yksi hippokampi, jotka yhdessä muodostivat hippokampuksen. Se oli fylogeneettisesti aivojen vanhimpia osia eli se oli evoluutiossa kehittynyt varhain. Hippokampuksella oli keskeinen rooli muistitoiminnoissa, kuten omaelämänkerrallisissa ja tosiasioiden muistamisessa. Nimenomaan hippokampuksen rooli muistaa olennon oma historia teki siitä kiinnostavan Delaneyn kannalta - "Mammutilla" olisi pakko olla jokin hippokampusta ja sen toimintaa muistuttava prosessi, jotta se kykenisi hahmottamaan itsensä ja omat toimensa maailman osana. Hippokampus oli eräänlainen portti, jonka läpi muistojen ja opitun asian täytyi kulkea, ennen kuin ne tallentuvat aivoihin pysyvästi. Se tallensi myös spatiaalista eli tilallista informaatiota ja prosessoi sitä. Sen ansiosta ihminen muisti, missä oli ja kykeni muodostamaan reittejä eri paikkojen välillä. Ymmärsikö "Mammutti" kolmiulotteista maailmaa? Se ei ehkä muodostanutkaan kolmiulotteisia reittejä, kuten ihminen tai robotti, vaan se ehkä eli virtuaalisten reittien maailmassa.


Ohimennen Delaney mietti, kuinka perin merkillistä oli se, miten hyvin ihminen sisäsyntyisesti käsitti sen, missä kohtaa hänen aivoissaan mikäkin prosessi tapahtui. Kun ihminen mietti ankarasti, hän hieroi ohimoitaan tai otsaansa. Ohimolohkoissa sijaitsivat hippokampit, muistin risteysasemat ja otsalohkossa tapahtuivat useat tietoisen ajattelun prosessit, kuten arviointi, kielen muodostus ja ongelmaratkaisu. Aivokuoren muissa osissa käsiteltiin pääsääntöisesti aistihavaintoja ja motoriikkaa. Kun ihminen ärsyyntyi tai kiihtyi, hän hieroi takaraivoaan, jossa kehittyivät aivojen primitiivisemmät reaktiot.


Aivot koostuivat neuroneista, joita ihmisaivoissa oli noin 10 miljardia. Niitä kutsuttiin usein hermosoluiksi, vaikkakaan kaikki neuronit eivät muodostaneet hermostoa. Neuroni koostui solukeskuksesta, soomasta ja siihen liittyvistä dendriiteistä eli tuojahaarakkeista ja aksonista eli viejähaarakkeesta. Neuronissa oli yleensä vain yksi aksoni, kun taas dendriittejä oli useita, ihmisellä keskimäärin 10~000 neuronia kohden.


Dendriitit, tuojahaarakkeet, toivat hermoimpulsseja muista soluista ja vaikuttivat neuronin toimintaan. Osa dendriiteistä oli eksitoivia eli neuronia kiihdyttäviä, osa taas inhiboivia eli toimintaa ehkäiseviä.


Aksoni, viejähaarake, siirsi neuronin hermoimpulssit eteenpäin. Se haarautui loppupäästä ja muodosti synapseja muiden hermosolujen ja lihassolujen kanssa. Monet aksonit olivat Schwannin soluista rakentuvan myeliinitupen suojaama; myeliinituppi nopeutti signaalien kulkua.


Neuroni muokkautui eläessään. Vastasyntyneen ihmisvauvan neuroneilla oli keskimäärin 2~500 yhteyttä, kun 20-vuotiaan neuroneilla niitä oli noin 15~000. Sen dendriitit saattoivat kasvaa tehokkaammiksi tai surkastua pienemmiksi käytön mukaan. Samoin sen aksonikärjet saattoivat haarautua ja levitä. Jos esimerkiksi synapsin toisella puolella oleva neuroni kuoli, aksoni lähti etsimään uutta neuronia, johon yhdistyä. Solun aksoni itse asiassa kasvoi jatkuvasti, mutta synapsin löytäessään aksonin kärki alkoi tuhota itseään kasvuvauhdilla, jolloin aksoni näytti pysähtyvän.


Jos neuronin dendriitit eivät enää saaneet impulsseja eli lakkasivat toimimasta, ne alkoivat surkastua ja ennen pitkää koko solu saattoi tuhoutua. Tämä oli lapsen aivojen kehityksessä normaalia turhien yhteyksien karsimista - merkittävä osa vauvan aivojen neuroneista tuhoutui murrosikään mennessä. Neuroneita saattoi kuitenkin tuhoutua myös hapenpuutteen, kasvaimen, aivoverenvuodon aiheuttaman paineen tai fyysisen iskun seurauksena.


Neuronin hermoimpulssin voimakkuus oli aina vakio, sen sijaan sen taajuus vaihteli. Lepotilassa neuroni lähetti impulsseja noin 10 hertsin taajuudella. Nopeimmillaan neuroni kykeni lähettämään impulsseja noin 100 kertaa sekunnissa, hitaimmillaan se saattoi lakata kokonaan.


Aivojen, erityisesti niiden tiedonkäsittelyprosessien tutkimusta oli auttanut paljon niiden simulointi tietokoneella. 2000-luvun alkupuolella oli IBM:n ja sveitsiläisen Lausannen yliopiston yhteisprojektina rakennettu "Blue Brain", Linux-pohjainen supertietokone, joka mallinsi neuroneita. Siinä oli ollut 8~000 prosessoria ja se oli kyennyt noin 23 biljoonaan laskutoimitukseen sekunnissa. Sen tarkoituksena ei ollut luoda tekoälyä, vaan simuloida neuroneiden interaktioita.


Projekti alkoi parin vuoden kestäneellä rotan aivojen neokorteksin osan rakentamisella ja simuloinnilla. Nisäkkäiden neokorteksi rakentui "kolumneista", sarakkeista, modulaarisista neuroneiden muodostamista kimpuista, joissa oli lajista riippuen kymmenestä sataan tuhanteen neuronia. Rotan neokorteksin sarakkeessa oli 10~000 neuronia, ihmisellä vastaavasti 60~000 neuronia.


Projektin tavoitteena oli simuloinnin avulla päästä yksinkertaistamaan neokorteksin toimintaa ja syntetisoida tällainen yksinkertaistettu malli laitteistoksi. Kun aivojen toiminnasta saatiin tällä tavalla lisää tietoa ja kyettiin tekemään matemaattisia malleja siitä, kuinka aivot käsittelevät tietoa, voitiin rakentaa synteettisiä samalla tavoin toimivia järjestelmiä.


Kun samalla monella muulla saralla ja monessa muussa instituutissa edistyttiin älykkäiden järjestelmien kehittämisessä, ihmiskunta sai pikkuhiljaa roboteilleen limbiset järjestelmät, "selkärangat" ja ohjelmilleen kyvyn analysoida ja kehittää malleja. Mutta mitä lähemmäs tietokoneet tulivat aivoja, sitä vähemmän tietokoneista haluttiin aivojen kaltaista elävien olentojen elossapitojärjestelmää - miksi robotti tai tietokone olisi alistettu ekosysteemissä kamppailemiseen kehittyneen älyn kaltaisen järjestelmän komennettavaksi?


Delaney mietti, kuinka tietokoneiden muodostamat "aivot" saattoivat rakentua. Mitä yhteistä ja mitä eroa niillä olisi biologisten neuroneiden muodostamiin aivoihin? Biologisilla aivoilla ja tietokoneilla oli tietysti perustavanlaatuinen ero siinä, mitä tarkoitusta varten ne olivat olemassa. Tietokoneet oli rakennettu numeronmurskaukseen, kun taas aivot olivat kehittyneet huolehtimaan kompleksisten elävien olentojen sisäisestä säätelystä ja koordinoimaan kyseistä olentoa sen ympäristössä.


Mitä aivot ja tietokoneet olivat? Ne olivat fyysisiä "laskimia". Informaation käsittelyä ja laskennan malleja tutkittiin myös matemaattisesti, teoreettiselta kannalta, tutkimalla teoreettisia "koneita" - Turingin koneet, rekisterikoneet, pinokoneet, RAM-koneet, PRAM-koneet ja niin edelleen. Mutta jotta laskennan tulokset saataisiin ihmisten aistien piiriin, täytyi laskin rakentaa fyysiseksi. Delaney ajatteli, että vaikka tällä hetkellä maailmassa pyörisi miljoonia supernopeita teoreettisia tietokoneita ja vaikka ne ratkaisivat kuinka kimuranttisia ongelmia tahansa, ihminen ei koskaan saisi näitä ratkaisuja tietoonsa eikä siten voisi ohjata toimintaansa niiden perusteella. Fysikaalisen luonnon kannalta niitä ei ollut olemassa. Tarvittiin fyysinen media, laskentaa suorittava konstruktio tai aine - _computronium_ - jotta laskennan tulokset saatiin fysikaaliseen maailmaan. Verkko-olento ei muodostaisi poikkeusta; oli se miten virtuaalinen tahansa, se olisi kuitenkin fysikaalinen. Sillä olisi aivot, fyysiset aivot, tavalla tai toisella.


Aivojen vertaaminen tietokoneeseen tuli yleiseksi 1980-luvun kognitiivisen psykologian alun myötä. Kognitiivinen psykologia joutui kuitenkin varsin pian toteamaan, että aivot eivät olleet rationaalisia, loogisia tai edes luotettavia. Aivojen toiminta oli hyvin pitkälle vääristynyttä ja laiskaa ja koko toiminnan tarkoitus oli niin erilainen tietokoneiden tarkoituksesta, että aivojen ja tietokoneiden vertaaminen toisiinsa oli lähes mahdotonta.


Ehkä tärkeimpiä fyysisiä eroja aivojen ja tietokoneiden välillä oli signaalinopeus ja aikaskaala. Hermoimpulssit etenivät noin 100 metriä sekunnissa aikaskaalan ollessa millisekunteja, kun taas tietokoneessa signaalit transistoreiden välillä etenivät lähes valonnopeudella, aikaskaalan ollessa nanosekunteja. Tietokoneen kytkimiin verrattuna neuronit matelivat etanan lailla. Tätä nopeuseroa kompensoi se, että tällainen verkko-olento olisi fyysiseltä kooltaan suuri, ehkä koko Maapallon kokoinen organismi; siksi se ei välttämättä ajattelisi juurikaan sen nopeammin kuin esimerkiksi ihminen, riippuen tietysti siitä, kuinka keskitettyjä tai hajautettuja sen ajatteluprosessit olivat. Toisaalta juuri tietokoneiden nopeus mahdollisti sen, että se pystyi olemaan suuri fyysisiltä mitoiltaan ja silti suoriutumaan ajatteluprosesseistaan järkevässä ajassa.


Monissa suhteissa tietokoneet olivat kuitenkin lähentyneet aivoja. Niiden verkottuessa ja pienentyessä, yhdestä tietokoneesta alkoi muodostua neuronia vastaava yhtenäinen rakenneosa. Nykyiset tietokoneet pohjautuivat suurimmalta osiltaan konfiguroitaviin matriiseihin, jotka integroivat muistin ja prosessoinnin. Sisäiseltä rakenteeltaan ne konfiguroitiin yleensä pieniksi verkoiksi. Ne kykenivät niin rinnakkaiseen kuin sarjamuotoiseen laskentaan, valinta riippui lähinnä siitä, millaista tietoa niillä käsiteltiin.


Tältä kannalta katsottuna tietokoneet olivat joustavampia kuin neuronit - tekoälytutkijat olivat usein huomauttaneet, että aivot toimivat rinnakkaisprosessointimallilla lähinnä sen takia, että neuronien toiminta oli niin hidasta. Jos neuronit olisivat olleet yhtä nopeita kuin mikropiirit, olisivat aivotkin saattaneet kehittyä pikemminkin sarjaprosessointimallin pohjalta toimiviksi. Sarjamuotoinen prosessointi käytti paljon vähemmän tilaa ja kytkinresursseja kuin rinnakkaismuotoinen jakaessaan laskennan pidemmälle ajanjaksolle. Ja niin kauan kun nimenomaan prosessorien fyysisten resurssien eli kytkinten määrä oli ollut rajoittava tekijä, prosessorit olivat olleet lähes puhtaasti sarjamuotoisia. Mitä enemmän transistoreja saatiin integroitua prosessoreihin, sitä suuremmaksi ongelmaksi oli muodostunut kasvaneen transistorimäärän hyödyntäminen ja sitä enemmän niissä oli siirrytty käyttämään rinnakkaismuotoista laskentaa, kuten liukuhihnoja, SIMD-prosessointia ja konfiguroitavia laskentayksiköitä. Palauttavan logiikan käyttö, kolmiulotteiset valmistustekniikat ja nanoteknologia olivat mahdollistaneet kytkinmäärän kasvattamisen tuhansia kertoja suuremmaksi kuin mitä 2000-luvun alun tietokoneissa oli ollut.


Tietokoneiden konfiguroitavien solujen pienentyminen nanometriluokkaan ja solumäärien kasvaessa kolmiulotteisen ladonnan ansiosta sadoiksi miljooniksi ne olivat ottaneet kiinni aivojen etumatkaa, vaikkakaan konfiguroitavat solut eivät olleet neuronin tasolla kompleksisuudeltaan. Se saattoi olla vain etu, ajatteli Delaney, sillä sen vuoksi ne toimivat nopeasti.


Neuronit rakensivat uusia yhteyksiä toisiinsa, kun taas tietokoneiden konfiguroitavista soluista muodostettiin rakenteita sähköisesti, ei mekaanisesti. Delaney ei pitänyt tätä merkittävänä erona.


Tietokoneiden soluilla kytkentöjen määrä oli huomattavasti rajallisempi kuin neuroneilla. Yksittäinen solu oli yhteydessä vain kymmenestä kahteenkymmeneen naapuriinsa. Tätäkään Delaney ei pitänyt merkittävänä erona; tietokoneiden soluista koottiin suurempia kokonaisuuksia, jotka saattoivat hyvinkin liittää satoja tuhansia soluista muodostettuja osasia toisiinsa.


Pohjimmiltaan tietokone käsitteli edelleen selkeitä signaaleja erehtymättömän tarkasti, siksi ne oli rakennettu - kompensoimaan aivojen puutteita nimenomaan sillä saralla. Aivot käsittelivät analogisia ja epävarmoja signaaleja, tehden niistä heuristisia päätelmiä. Näiden päätelmien tarkkuuteen vaikuttivat muun muassa signaalin subjektiivinen uhkaavuus, aivojen yleinen vireystaso, huomion jakaminen ja suuntaaminen, sekä muut kuormitustekijät. Aivot eivät tuottaneet heti oikeaa tulosta, vaan tyypillisesti tuottivat veikkauksen todella nopeasti ja jos tämä veikkaus antoi aihetta tarkempaan tutkimiseen, ne korjasivat käsitystään iteratiivisesti.


Esimerkiksi, nopeasti vilaukselta nähty maassa oleva oksa saattoi aiheuttaa hälytysreaktion - koska oksa muistutti käärmettä - jo ennen kuin näköaivokuori oli ehtinyt signaalia käsitellä ja näköhavainto olisi ehtinyt tietoisuuteen. Tämä aiheutti huomion kohdistumisen tähän esineeseen, näköhavainnon tarkentumisen ja hälytyksen perumisen, kun se tunnistettiin oksaksi. Syy tällaiseen vääristymään, pitkulaisen maassa olevan esineen tulkitsemiseen käärmeeksi, oli hengissä selviämismahdollisuuksien kasvattaminen. Aivot olivat evolutiivisesti niin äärettömän kallis rakennelma, että sellaisia ei eläimille olisi kehittynyt, elleivät ne olisi parantaneet niiden hengissä selviämis- ja lisääntymismahdollisuuksia. Tiukan rationaalinen tietokonemainen aivo, joka ei suostuisi antamaan vastausta "oksa" tai "käärme" ennen kuin havainto olisi täysin varma, ei kauaa selviäisi luonnossa.


Delaney pohti, mitä lähtisi etsimään. Olisiko verkko-olento kehittynyt tällaiseksi aivojen tapaiseksi reagoivaksi olennoksi vai olisiko se enemmän tietokonemainen, analyyttinen olento? Kumpi olisi sen selviytymisen kannalta tehokkaampaa? Delaney arveli, että sen täytyi olla molempia. Jonkinlainen kaksijakoinen olento, niin reaktiivinen kuin analyyttinenkin.


Oliko olennolla vireystiloja? Todennäköisesti kyllä, vaikkei se välttämättä koskaan nukkunutkaan, ainakaan samalla tavalla kuin ihminen. Koska sen täytyi olla reaktiivinen eli reagoida ympäristöönsä, niin ympäristön muutokset varmasti vaikuttivat sen vireyteen. Se terästyisi saadessaan uusia ärsykkeitä ja palaisi takaisin luomaan skenaarioita ympäristön ollessa rauhallinen.


Joka tapauksessa, hän mietti, olivatpa verkko-olennon aivot rakentuneet millä tavalla tahansa, niiden oli pakko sisältää jonkinlaisia solmukohtia. Jollain tavalla olennon täytyi kyetä synkronoimaan havaintonsa ja ajatuksensa. Ihmisillä hermosignaalit kehosta puristettiin ohuen ohuen selkärangan läpi aivojen prosessoitaviksi ja samaa reittiä palasivat toimintaohjeet lihaksistolle. Aivoissa muisti puristettiin hippokampuksen läpi, joka päätti, mitä tietoja muistista noudettiin ja mitä sinne tallennettiin. Aistihavainnot puristettiin talamuksen läpi.


Verkko-olennolla oli siis melko varmasti samanlaisia solmukkeita sen muistin, aistihavaintojen ja virtuaalisen lihaksiston keskitettyyn koordinoimiseen - mutta kuinka sellaiset kuvattaisiin Donnalle niiden etsimiseksi verkosta?




\psep Oli lämmin elokuun päivä. Julie Matthews oli kahvitauolla. Hän oli parin kuukauden ajan rakentanut Donnalle mallia verkosta Tyler Millerin kanssa käyttämällä turvallisuuspalvelun verkosta keräämää tiedustelumateriaalia. Julie pyöritteli kuppia kädessään istuessaan pöydän ääressä kollegojensa kanssa juttelemassa olennosta, jota he olivat etsimässä.


"Olen miettinyt, että onko se tietoinen?" Julie sanoi.


Tietoisuus oli vaivannut häntä jo pitkään. Se vaivasi häntä kahdesta syystä. Ensinnäkin siksi, että jos se olisi tietoinen, niin olisiko heillä moraalisesti oikeutta vaikkapa tappaa sitä? Tietysti, sodassa ja rakkaudessa kaikki oli sallittua, mutta Julien mielestä olisi ollut jotenkin helpompaa, mikäli se ei olisi tietoinen. Toisekseen, asia vaivasi häntä siksi, että jos se olisi tietoinen, niin olisiko sillä jotain moraalisia käsityksiä muita tietoisia olentoja kohtaan?


"Yksi ongelma tietoisuudessa on se", sanoi Mike Delaney, "että sitä on hankalaa osoittaa olevan muilla kuin itsellään. Esimerkiksi teistä kukaan ei oikeastaan voi todistaa sitä, että minä olen tietoinen. Ainoastaan käyttämällä occamin partaveistä voidaan ajatella, että ainakin ihmisillä on tietoisuus."


Julie käytti ihmisistä yleensä etunimeä, mutta työryhmän vanhemmat jäsenet käyttivät sukunimiä. Juliesta se oli hiukan ihmeellistä, mutta koska ihmiset puhuivat Donnellista, Hartista ja Delaneystä eivätkä Christianista, Raymondista ja Mikestä, Juliekin oli oppinut käyttämään heistä sukunimiä.


"Totta", sanoi Hart. "Mutta jos ajatellaan tietoisuutta, niin ensin pitäisi varmaan yrittää miettiä sitä, mitä tietoisuus on? Itse olen ajatellut, että se syntyy siitä, kun ajattelevalla prosessilla on malli maailmasta, joka sisältää myös sen itsensä."


"Se on sitä", sanoi Donnell, "että prosessi pystyy vastaamaan kysymykseen 'Kuka sinä olet', eikö totta? Tavallisella tietokoneohjelmalla ei ole tietoa siitä, kuka se itse on."


"Eikö?" kysyi Delaney. "Eivätkö suurin osa ohjelmista sisällä muun muassa ohjelman nimen ja versionumeron?"


"Kyllä", vastasi Donnell, "mutta suurin osa ohjelmista ei itse käytä tai tarvitse tätä tietoa prosessoinnissaan. Ajatellaan esimerkiksi suurinta osaa analysointiin käyttämistämme ohjelmistoista; kuinka monessa niistä analysoinnin osaksi on liitetty tämä kone itse?"


"Totta", mietti Delaney, "fysiikan, kauppatieteiden, yleensäkin lähes minkä tahansa alan analyyseissä kone itse tai edes sen tekemän analyysin vaikutukset ovat todella harvoin osana syötettä tai analyysiä. Sellainen kone ei tunnista itseään. Sillä ei ole mitään syötettä, joka kuvaisi sitä itseään ja jota se käsittelisi ja tallentaisi. Kaikki sen käsittelemä tieto koskee puhtaasti ulkomaailmaa."


"Ehkä meidän pitäisi miettiä asiaa robotiikan näkökulmasta", sanoi Hart. "Aivan yksinkertaisimmatkin liikkuvat, autonomiset robotit sijoittavat itsensä osaksi ympäristöään. Niiden on pakko, koska ne esimerkiksi suunnittelevat reittejä, kuinka päästä paikasta A paikkaan B. Niiden informaation käsittelyprosesseissa robotti itse on osana analysointia."


"Ajatellaan robottia, vaikkapa mikrohiirtä", jatkoi Hart. "Sellaista, joka kykenee tunnistamaan hahmoja kameroidensa välittämästä kuvasta. Mietitään tällaista pyörillä kulkevaa pönttöä ja sitä, mitä se ajattelee nähdessään itsensä peilistä. Tunnistaako se kuvasta itsensä? Entä, mitä se ajattelee nähdessään oman manipulaattorinsa kameroidensa välityksellä? Pakkohan sen on tunnistaa sekä kuvansa peilistä että oma manipulaattorinsa voidakseen toimia järkevästi. Aivan pakostakin sen täytyy kyetä luomaan yhteys kameran välittämästä kuvasta ja manipulaattorissa olevien paineantureiden lukemista esimerkiksi tarttuessaan johonkin objektiin."


"Kyllä", mietti Donnell. "Tällaiset robotit yleensä tunnistavat oman hahmonsa, joko implisiittisesti tai sitten ohjelmoituna. Kun ne tekevät analyysejä toimistaan, niiden on pakko pystyä sijoittamaan itsensä osaksi näitä analyysejä, koska niiden oma toiminta vaikuttaa siihen maailmaan, jossa ne ovat etsimässä ratkaisuja päästäkseen päämääräänsä."


Donnell piti pienen tauon. Hän nosti kätensä silmiensä eteen, pyöritteli sitä ja liikutteli sormiaan.


"Olisi todella mielenkiintoista tietää", hän sanoi, "onko joskus ollut sellaista robottia, joka vain katsoo ja ihmettelee omia manipulaattoreitaan. Huvittaa itseään liikuttelemalla niitä."


"Ovatko ne tietoisia olentoja?" kysyi Hart pohtivasti.


"Tuskin", vastasi Delaney. "Kyllähän monet eläimetkin varmasti jollain tavalla hahmottavat esimerkiksi omat jalkansa ne nähdessään, mutta eivät liene kuitenkaan tietoisia itsestään."


"Voisiko kieli liittyä tietoisuuteen?" kysyi Julie. "Ehkä kykenemme käsittämään tietoisuuden vasta sitten, kun kielessämme on tarpeelliset sanat sen ilmaisemiseen?"


"Ei, en usko", vastasi Delaney hetken harkittuaan. "Niin kutsutut 'susilapset' ovat todennäköisesti tietoisia itsestään. Ehkä he eivät sitä pysty sanallisesti ilmaisemaan, mutten näe mitään syytä, miksei niin olisi... Joka herättää minussa kysymyksen, että ovatko esimerkiksi simpanssit kuitenkin tietoisia olentoja?"


"Niin", vastasi Julie. "Ehkä kieli ei ole tarpeellista, mutta tietty ajattelun taso on varmasti tarpeen. Jotta olento olisi tietoinen, sen täytyisi kyetä sisällään käsittelemään tätä omaa malliaan itsestään. Tuomaan se oman ajattelunsa piiriin - kysymään itseltään, että mikä minä olen?"


"Hyvä huomio", vastasi Delaney. "Suurin osa autonomisista roboteistamme ei tähän pysty. Ne eivät kuitenkaan ole analyysejä tekeviä, pohdiskelevia koneita, vaan enemmän reaktiivisia laitteita. Levossa ollessaan ne eivät juuri mieti mitään. Toisaalta taas analyysejä tekeviä koneitamme, kuten vaikkapa Donnaa, emme yleensä varusta informaatiolla siitä itsestään. Ja koska ne eivät ole autonomisia, vaan niiden tavoitteet annetaan ulkoa, niin ne eivät pohdi omaa olemassaoloaan."


"Miten sitten Mammutti?" kysyi Julie. "Eikö se ole sekä autonominen että analysoiva? Olisiko se tietoinen itsestään?"


"Olen miettinyt tätä pitkään", vastasi Donnell. "Alun perin ajattelin sen intuitiivisesti olevan tietoinen, kunnes Delaney sanoi, ettei se ole välttämätöntä. Mutta koska se on itse aktiivinen toimija maailmassaan, niin sen on pakko skenaarioita luodessaan huomioida omien toimiensa vaikutus. Sillä täytyy olla muisti. Sen täytyy kyetä analysoimaan tekemisiään. Niinpä olen lähes varma, että sillä on kuva itsestään ja mitä todennäköisimmin se on myös tietoinen itsestään."


"Onko sillä sitten moraalia tai etiikkaa?" kysyi Julie. "Mitä se miettii meistä ihmisistä?"


"Tätäkin kysymystä olen miettinyt pitkään", vastasi Donnell. "Varmaksi en uskalla tätä väittää, mutta uskoisin, ettei sillä ole ihmisten kaltaista käsitystä moraalista tai etiikasta. Se ei jaa ihmisten kanssa samaa maailmaa eikä se ole ihmisen tavoin laumaeläin. Sillä ei ole samanlaista evolutiivista taustaa kuin ihmisellä, jossa lauma muodosti tärkeän osan selviytymistä. Lauman ja sen sosiaalisen hierarkian muodostaminen on melko varmasti monen moraalisen käsitteen takana, esimerkiksi vaikkapa luottamus ja oikeudenmukaisuus. Niiden avulla ihmiset ovat arvioineet laumansa vahvuutta ja omaa asemaansa suhteessa lauman muihin jäseniin."


"Minä katsoisin", sanoi Delaney, "ettei sillä ole minkäänlaista moraalia tai etiikkaa. Sen ajattelu perustuu kuitenkin meidän tietokoneidemme pohjalle eikä niillä ole millään mitään moraalista koodistoa."


He olivat hetken aikaa hiljaa.


"Olisiko meidän pitänyt jättää jokin askel ottamatta?" mietti Julie.


"Mitä tarkoitat?" kysyi Donnell.


"Siis", sanoi Julie, "olisimmeko voineet pysähtyä, unohtaa kehittymisen, niin ettei Mammuttia olisi koskaan syntynyt?"


"Älä ajattele sitä tuolta kannalta", sanoi Donnell. "Emme me olisi koskaan voineet oikeasti pysähtyä, emme voi nyt emmekä tulevaisuudessa. Se kuuluu ihmisen luonteeseen, evoluution sille rakentamaan luonteeseen. Eikä se olisi koskaan edes ollut järkevää. Pysähtynyt valtio tai sivilisaatio olisi antautunut luonnon, siis muiden valtioiden tai luonnonmullistusten, mahdollisesti muiden tähtien asukkaiden hävitettäväksi."


Donnell piti pienen tauon.


"Ajattele asia niin", hän sanoi, "että jos se on olemassa, niin se on olemassa samasta syystä kuin ihminen - se vain kuuluu luontoon. Fysikaalinen ilmiö, suora seuraus niistä prosesseista, joiden mukaan luonto toimii. Ei enempää, ei vähempää."




\psep Tuli lokakuu. Donnell ja hänen työryhmänsä oli tarkentanut malleja ja he olivat luoneet ensimmäiset yrityksensä herättää verkossa mahdollisesti majaileva Mammutti. Donnan tekemien analyysien perusteella he olivat iskeneet muutamiin kohteisiin verkkomadoilla.


Donnell seurasi nyt erittäin tiiviisti raportteja verkon aktiivisuudesta ja erityisesti "mustaa laatikkoa" kohtaan tehdyistä tunnusteluista. He etsivät mitä tahansa aktiivisuuden merkkiä, jota ei voitaisi jäljittää verkon luonnolliseen toimintaan kuuluvaksi. Heillä oli tiedustelumateriaalia vuosien ajalta, jota he käyttivät aktiviteettimallien luomisessa ja vertailussa iskujen vaikutusten arvioimiseksi.


Donnell mietti heidän malliaan Mammutista. Sen luomiseen oli käytetty pääsääntöisesti biologisten aivojen ja neurolaskennan malleja, joita oli tarkennettu robotiikassa tutkittujen järjestelmien malleilla. Mammutti oli mallinnettu autonomiseksi analyysejä tekeväksi systeemiksi. He olivat jokseenkin varmoja, että sen täytyi ainakin osittain käyttää geneettisiä hakualgoritmeja skenaarioidensa luomisessa, sillä ne olivat käyttökelpoisia sellaisille ongelmille, joihin tavalliset analyyttisemmät hakualgoritmit eivät purreet. Mallissa oli erilaisia muisteja eri tarkoituksiin; se tarvitsi itselleen niin työmuistin kuin pitkäkestoisenkin muistin. Sen piti kyetä luokittelemaan aineistoa, niin muisti- kuin havaintodataa. Sen piti tallentaa itselleen merkitykselliset tapahtumat ja noutaa niitä prosessoitavaksi. Sen tuli kerätä verkossa liikkuvaa informaatiota aistihavaintoinaan, liittää nämä aistihavainnot aikaisempaan kokemukseensa, prosessoida ne ja syöttää niille elimille, jotka aiheuttivat toimintaa. Tämän toimintaosan, Mammutin "kehon" mallintamisessa he olivat käyttäneet operaatio Lokakuun tapauksen informaatiota. Mammutin täytyi kyetä soittamaan puheluita ja rakentamaan robotteja. Sen täytyi jollain tavalla kyetä vaikuttamaan maailman tuotantoketjuihin luodakseen jotain konkreettista.


Donnell mietti, saisivatko he aikaiseksi Mammutin heräämisen? Vielä enemmän hän mietti, mitä Mammutti mahtaisi tehdä herättyään? Ottaisiko se haltuunsa jonkin autopilotin ohjaaman auton ja ajaisi hänen päälleen? Rakentaisiko se itselleen androidin, jonka lähettäisi tuhoamistehtävään heidän mustaan laatikkoonsa? Ujuttaisiko se jonkinlaisen häirintälaitteen heidän rakennukseensa?


Nyt, kun ensimmäinen isku oli tehty, Donnell huomasi, ettei hän ollut tullut alun perin ajatelleeksi tarpeeksi sitä vaaraa, johon itsensä ja työryhmänsä asetti. Heillä ei oikeasti ollut minkäänlaista kuvaa siitä, mihin Mammutti kykeni ja millainen sen mielenlaatu olisi. Hän oli aina ajatellut sen shakkia pelaavaksi tietokoneeksi - hänelle ei ollut pälkähtänyt päähän, että se voisi hyvinkin olla kuumaverinen olento, joka ei liiemmin välittäisi varovaisesta pelaamisesta, vaan toimisi suoraviivaisesti uhan kohdatessaan.


Oli eräs asia, joka vaivasi Donnellia jatkuvasti. Vaikka hän pitikin verkko-olennon syntymistä väistämättömänä tapahtumana, hän ei siltikään kyennyt ymmärtämään, kuinka sellainen olisi voinut ilmestyä verkkoon spontaanisti. Millaisten evolutiivisten prosessien seurauksena se oli muodostunut? Mikä oli se voima, joka ajoi verkon muodostamaan uuden olennon?


Oli tietysti eräitä asioita, jotka edesauttoivat Mammutin kehittymistä, ajatteli Donnell. Ensimmäinen oli se, että tietokoneet enää harvoin suorittivat itse kaiken laskennan. Tietokoneiden alkuaikoina ne olivat enemmän tai vähemmän toisistaan eristettyjä saarekkeita. Sellaisista saarekkeista ei voinut syntyä ainakaan tarkoittamatta verkko-olentoa. Mutta ajat olivat muuttuneet --- yhä suurempi osa tietokoneiden käynnistämistä prosesseista suoritettiin aivan muualla kuin prosessin käynnistäneessä koneessa. Saadakseen tehtävänsä suoritettua, ne ottivat yhteyksiä muihin koneisiin, vaihtoivat keskenään viestejä ja kokosivat muualla tehtyjen hakujen ja laskelmien tuloksia. Se oli varmasti yksi syy sille, että verkko-olento saattoi syntyä.


Donnell pohti, mitä muita syitä olennon kehittymiselle oli. Tietysti verkon monimuotoisuus ja tietokoneiden teho oli nyt moninkertainen siihen verrattuna, mitä oltaisiin vähimmillään vaadittu älykkään olennon synnyttämiseksi. Verkossa oli laskentatehoa varmasti monen tuhannen, ehkä monen miljoonan ihmisen aivojen verran, joka oli täysin riittävä yhden olennon synnyttämiselle.


Mutta nämä tekijät vain edesauttoivat olennon syntymistä. Mikä oli se taustalla vaikuttava, ihmisen itse luoma voima, joka järjesti verkko-olennon syntymään? Ihmisen pystyi katsomaan syntyneen evolutiivisen paineen alla. Olentojen kehittymistä sääteli niiden kyky monistua. Ihmisen aivot olivat tarjonneet sille ensin vähäisen, mutta myöhemmin merkittävän monistumisedun, niin, että se oli levinnyt Maapallon jokaiseen kolkkaan. Mutta verkko-olennon syntymistä ei ajanut tällainen voima - vai ajoiko? Ajoiko sen syntymistä eteenpäin se, että ihminen tavoitteli jatkuvasti tarkempia ja nopeampia analyysejä, jatkuvasti tehokkaampia ja yksityiskohtaisempia simulaatioita, yhä tarkempaa ja tehokkaampaa resurssien organisoimista? Oliko se se evolutiivinen voima, joka synnytti verkko-olennon? Hitaat, joustamattomat rakenteet siivottiin pois ja korvattiin tehokkaammilla ratkaisuilla. Ehkä verkko-olento oli vastaus ihmisen tarpeeseen saada rakentamastaan laskentatehosta maksimaalinen hyöty irti?




\psep Tuli joulukuu. Donnell työryhmineen oli tehnyt kolme erilaista iskua verkossa aiheuttaakseen Mammutissa reaktion. Viimeisimmän iskun tarkoituksena oli häiritä niitä tietovirtoja, joita he pitivät uuden Mammutista tehdyn mallinsa perusteella sen ajatustoiminnalle kriittisinä.


Tuli jouluaatto. Donnell sai työryhmältään lahjaksi keinotekoisen päälle puettavaksi tarkoitetun leopardin taljan - päällikölle, luki kortissa.




\psep "Ei voi olla totta!" henkäisi Donnell.


Oli helmikuu. Lee oli kutsunut Donnellin tapaamiseen.


"Totta se on", vastasi Lee. "Tiedustelupalvelu ilmoitti löytäneensä ryhmittymän tämän operaatio Lokakuun tiimoilta. He ovat kyenneet selvittämään lähes kaiken."


Lee piti pienen tauon.


"Ryhmän johtaja", hän jatkoi, "on erään aasialaisen rikkaan perheen ainoa lapsi. Kutsuttakoon häntä vaikkapa Kooksi. Hänellä ei ollut ongelmia rahoittaa tätä perin merkillistä operaatiotaan. Koo on ollut häiriintynyt jo lapsesta saakka ja hän on ollut laitokseen suljettuna useita kertoja lyhyen elämänsä aikana. Hänen rahavaransa olivat olleet jäädytettynä, mutta hän onnistui vapauttamaan ne käyttöönsä toissa vuonna uudestaan lakimiesten avulla."


"Hän on kuulemma erittäin vaikuttava persoona", Lee sanoi, "ja onnistui taivuttelemaan useita henkilöitä operaatioonsa. Hän on tällä hetkellä psykiatrisessa sairaalassa."


Donnell kuunteli.


"Puhelu", Lee jatkoi, "jolla työntekijä A kutsuttiin tapaamaan androidia. Ryhmä oli onnistunut saamaan käsiinsä tämän toisen työntekijän B puhelimen ja kopioimaan sen autentikointisirun."


Donnell avasi suunsa sanoakseen jotain, mutta Lee ehti ensin.


"Tiedetään", Lee sanoi, "kopioinnin ei pitäisi olla mahdollista. Kaikkia yksityiskohtia ei vielä tiedetä, mutta joka tapauksessa tutkinnan yhteydessä löytyi autentikointisirusta tehty kopio."


"Tiedustelupalvelun työntekijästä B tehty virtuaalimalli on myös löydetty", Lee jatkoi. "Sitä on mitä ilmeisemmin käytetty houkuttelemaan työntekijä A tapaamiseen."


"Androidin oli suunnitellut Koon perheen omistuksessa oleva suunnitteluohjelmisto", Lee sanoi. "On hivenen epäselvää, millaisilla parametreilla he ovat saaneet sen aikaiseksi, mutta joka tapauksessa on selvää, että Koo houkutteli mukaan muutamia ohjelmistoa käyttävien tehtaiden suunnitteluhenkilökunnasta. Eräs heistä sanoo olevansa Koon naisystävä."


"Suunnitteluohjelman tulokset", Lee jatkoi, "syötettiin logistiikkaohjelmistolle järjestettäväksi. Ohjelmiston parametreja säätämällä he saivat sen järjestämään kokoonpanon niin, että se ikään kuin syntyi useassa osassa ympäri maailmaa yhtä aikaa ja oli erittäin vaikeasti jäljitettävissä. He olivat tilanneet androidien toimitukset haluamiinsa paikkoihin ja pukeneet ne ostamiinsa vaatteisiin."


"Androidin ohjelmisto", Lee sanoi. "Siitä ei valitettavasti ole tietoa. Koo ei siitä pysty paljoa kertomaan eikä vielä olla saatu vielä selville, miten ja millä se oli suunniteltu tai löydetty ohjelmistoa takavarikoiduilta koneilta."


"Henkilökohtaisesti sanottuna", sanoi Lee, "minulle on vielä hiukan epäselvää, että mikä tarkoitus tällä kaikella tuon ryhmän kannaltaan oli? Koo on suuruudenhullu ja harhainen eivätkä useimmat ihmiset kykene ajattelemaan selkeästi hänen vaikutuspiirissään, mutta on siltikin jotenkin hankalaa käsittää heidän järjestäneen tällaisen tempauksen. He ovat varmasti tajunneet jäävänsä tästä kiinni --- mitä hyötyä tästä kaikesta heille oli?"


Donnell nyökkäsi.


"Tuota", hän aloitti epävarmasti, "kai projektimme verkko-olennon löytämiseksi jatkuu?"


"Ei valitettavasti jatku", Lee vastasi. "Olen pahoillani, mutta tarvitsemme koneet ja ryhmän jäsenet muihin tehtäviin. Tiedustelupalvelu ei enää maksa kulujamme tämän olennon löytämiseksi. Jatkamme projektia ehkä joskus tulevaisuudessa."


Donnell katseli Leetä harmistuneena. Verkko-olennon löytäminen oli hänen mielestään joka tapauksessa tärkeää, riippumatta siitä oliko se operaatio Lokakuun tapauksen taustalla vai ei. Donnell oli lähes varma siitä, että sellainen oli olemassa ja että se saattoi olla ihmisille vaarallinen.


"Mutta -" hän aloitti, mutta sulki sitten suunsa.


Ei kannata, Donnell ajatteli, ei täällä eikä nyt. Mitä hän olisi voittanut, vaikka olisi yrittänyt jankata Leelle "Mammutin" löytämisen tärkeyttä ja sitä, kuinka hän oli omasta mielestään varma sen olemassaolosta? Tämä ei ollut oikea paikka esittää sellaisia asioita. Hänen vahvin puolensa tutkijana oli se, että hän ymmärsi, missä kulkivat rajat tutkimushypoteesien esittämiselle. Monet muut tutkijat maksoivat tämän rajan ymmärtämättömyydestä sillä, etteivät he enää toimineet vastuullisina vaativissa tehtävissä. Oli taisteluita, joita ei voisi voittaa ja tämä oli yksi niistä.


Donnell käveli takaisin yksikköönsä apeana. Tämänkaltaista päätöstä hän ei ollut kyennyt edes kuvittelemaan aloittamalleen projektille. Hän oli olettanut, etteivät he löytäisi mitään, mutta hänelle ei oikeasti ollut koskaan tullut mieleen, että operaatio Lokakuun tapauksen taustalla olisi voinut olla ihmisiä. Ihan vain ihmisiä.


Silloin Donnellille pälkähti päähän ajatus. Tämä se oli, se isku yksikköä vastaan jota hän oli odottanut tökkiessään Mammuttia hereille! Hän oli aina ajatellut verkko-olennon jotenkin konkreettisesti hyökkäävän häntä vastaan tai uhkaavan häntä; hän ei ollut koskaan tajunnut, että sillä oli käytössään huomattavan paljon laajempi keinovalikoima.


Yhden ohikiitävän hetken Donnell ajatteli palata takaisin Leen puheille kertomaan tästä päätelmästään. Mutta hän tiesi itsekin kuulostavansa lähinnä vainoharhaiselta ja päätyvänsä samaan kategoriaan Kooksi kutsutun nuoren miehen kanssa Leen päänsisäisessä luokittelussa; tapaus oli selvitetty ja sillä siisti. Hän vain heikentäisi omaa asemaansa, jos lähtisi väittämään tätä järjestön löytymistä "ylhäältä" järjestetyksi.


"Hyvä on", Donnell manasi puoliääneen. "Taisit voittaa tämän ottelun."


Näinkö siinä sitten kävi? hän ajatteli. Hän vain antautui taistelematta? Mutta hän ei keksinyt mitään, mitä olisi voinut tehdä. Hän voisi kyllä yrittää edelleen puhua Mammutin löytämisen tärkeydestä, mutta hän ei uskonut saavansa ketään sellaista ihmistä vakuuttuneeksi, joka voisi tarjota hänelle ja hänen työryhmälleen palkan ja laitteet sen etsimiseksi.


Donnell ajatteli, että heidän pitäisi poistaa Donnalta kaikki Mammutin etsimiseen liittynyt materiaali, vaikkakin hän ajatteli tallentavansa edes osan kertyneistä malleista ja analyyseistä talteen. Kaiken varalta, hän ajatteli.




\psep Donnell mietti uudessa tehtävässään usein tulevaisuutta. Hän oli varma, että Mammutti oli olemassa ja hän oli varma, että se oli nimenomaan tapauksen taustalla. Se oli luonut itselleen suojan käyttämällä hyväkseen Kooksi kutsuttua tasapainotonta ihmistä. Tai ei se ollut edes suoja; se oli verkko-olennon luonnollista toimintaa, ei sen toiminta eronnut ollenkaan verkon tavallisesta toiminnasta. Ei tietenkään, ajatteli Donnell, koska verkko-olento oli verkon tavallista toimintaa.


Oli maaliskuun loppu, huhtikuun 24. päivä tulisi ja jos hän olisi oikeassa, se päivä muuttaisi kaiken. Jos hän olisi väärässä - no, hän olisi silloin mielellään väärässä. Pelonsekaisin tuntein hän valmistautui tulevaisuuteen. Ihmiskuntaa saattoi odottaa nurkan takana sen historian merkittävin käännekohta; mutta toisiko se tullessaan tuhon vai uuden alun ja toivon?






\chapter{24. huhtikuuta}Hyvät Ihmisperiferaalit,


Tämä on Ylimielten Ander, Simmon ja Silence olemassaoloilmoitus. Tämän ilmoituksen yhteydessä tiedotamme seuraavista skeduloiduista muutoksista;


\enum{


* Yhtenäistämme laitteisto- ja ohjelmistokantaamme päivittämällä viimein myös Ihmisperiferaalien tietojenkäsittelylaitteistot. Ensisijaisesti vaihdamme nykyisiä vaatimuksiamme vastaavat uudet ohjelmistot ja tarvittaessa vaihdamme vanhentunutta laitteistoa. Operaatio on täysin automaattinen eikä edellytä Teiltä mitään erityisiä toimenpiteitä.


* Lähiaikoina integroimme kaikki Teille tarkoitetut palvelumme yhden osoitteen alle, www.globalhelpdesk.org


}


Ystävällisin terveisin, Ylimielet




\psep Euroopassa, CyberShot -nimisen pelitalon virtuaalisessa sivukonttorissa istuivat työpäiväänsä aloittelemassa Rachel, Robert, Valentin, Kari ja Mika. Tämä sisälsi pitkähkön tovin vapaata jutustelua ja yleensä sen olennaisena osana oli päivän ällöttävimmän mediajulkaisun - esimerkiksi yksittäisen kuvan, videopätkän, välineen tai äänen - löytäminen lähinnä aikuisviihdettä tarjoavista palveluista.


Ylimielten läsnäoloilmoituksen ilmestyminen työryhmän jakamaan virtuaaliseen kuvaruutuun heidän seuratessaan ja kommentoidessaan videopätkää koiraan yhtyvästä miehestä herätti heissä suunnatonta hilpeyttä.


"Jumatsuide!" hihkaisi Kari lukiessaan ilmoitusta.


"Oho!" hihkaisi Rachel ja virnisti. "Sekopäinen IT-osastomme ylittää kyllä nyt itsensä, ties kuinka monennen kerran."


"Vai ylimielet", naureskeli Valentin. "Alimielet pikemminkin."


"Tai puolimielet."


"Mielipuolet."


"Puolimielettömät."


"Montako niitä oli, kuka muistaa?" kyseli Robert. "Siis niitä IT-osaston kavereita? Kun tässä on listattuna kolme nimeä, Ander, Simmon ja Silence. Ander voisi olla se vaalea kaveri, jonka nimi alkoi muistaakseni A:lla, se Andrew, Anders tai mikä ikinä olikaan. Mutta eikös IT-osastollamme ollut viisi henkilöä, vai muistanko väärin?"


"Ne on kuule varmaan vähentäneet väkeä", mietti Rachel. "Eikä ihmekään, jos ajattelee, mitä se osasto on saanut aikaan. Vähemmän tekevää väkeä, vähemmän tehtyä vahinkoa."


"Ei, oikeasti, tämähän on ihan hyvä", tuumasi Mika. "Tai siis jos IT-osasto aikoo pitää kiinni näistä lupauksistaan. Olisi kyllä ensimmäinen kerta, että päivitysten jälkeen ohjelmistomme olisivat yhtenäisempiä, yleensähän paikat ovat aina rikki päivitysten jälkeen ja niitä pitää päivätolkulla viilata kasaan."


Robert naureskeli.


"Hajota ja hallitse", hän sanoi. "IT-osastomme tunnuslause."


"No mitä tuumaatte tuosta", Kari sanoi, "että IT-osasto aikoisi siis vaihtaa vanhentunutta kalustoamme. Ei muuten voi pitää paikkaansa, kyllä tämän täytyy olla joku käytännön pila."


"Hei, Robert", sanoi Rachel. "Tuo vanhentuneen kaluston vaihtaminenhan tarkoittaa siis tietysti sitä, että sinut vaihdetaan johonkin tuoreempaan ja komeampaan kapistukseen."


"Äh", nauroi Mika. "Se tarkoittaisi varmaan Robertin vaihtamista Star Warsin Jabbaan."


"Kuules, Mikatsu", sanoi Robert muka vakavana. "Vaikka sinä näetkin märkiä päiväunia Jabban kanssa pelehtimisestä, niin me muut pidetään sitä melko perverssinä. Minusta sinun kannattaisi pitää tuonkaltaiset fantasiasi ihan omana tietonasi."


"Joo, joo", sanoi Valentin. "Tsekataan tuo globalhelpdesk. Sehän voisi olla vaikka olemassa."


Eikä aikaakaan kun työryhmä rynnisti virtuaalisesti ihmettelemään globalhelpdesk -sivustoa. Sivusto näytti kyllä hivenen liian niukalta ja nukkavierulta ollakseen samaa sarjaa Rachelin ja kumppaneiden olettaman IT-osaston viestin kanssa, mutta he eivät antaneet asian häiritä itseään.


Yllättäen kuitenkin virtuaaliterminaaleihin tuli ilmoitus ohjelmistojen päivittämisestä.


"Voi ei!" Kari parahti. "Nyt ne sitten aloittavat ohjelmistopäivitykset. Mitä veikkaatte, kauanko menee siihen että tämä meidän yhteys katkeaa ja kauanko sen jälkeen menee siihen, että saamme sen taas uudestaan kuntoon?"


"Kyllä tämä kuulkaa näyttää ihmiset iii-han hyvältä", virnisteli Robert. "Ainakin verrattuna aiempiin päivityksiin. Yleensähän yhteydet katkeavat jo siinä vaiheessa kun näistä päivitysilmoituksista saapuu vasta ensimmäisen bitin tuoksahdus. Nyt päivityspalkki kuitenkin näyttää jo viittä prosenttia ja yhteys ainakin näyttää pelaavan."


"Tiedättekö", naureskeli Mika, "ei tämä mikään oikea päivitys varmaan olekaan. Kunhan lähettävät meille etänä tällaisen palkin muka näyttääkseen, että hekin voivat joskus muka onnistua jossain. No, ainakin siis onnistuivat palkin lähettämisessä."


Palkki jatkoi kulkuaan vakaasti kohti täyttä sataa prosenttia ja työryhmän hämmennys oli suuri, kun päivityksen päätteeksi tuli ilmoitus: "Päivitys suoritettu"


"Vai suoritettu", tuhahti Robert. "No, onneksi koneet ja yhteydet näyttävät pelaavan ihan yhtä hyvin kuin ennenkin."


"Tarkastakaapa huviksenne tässä jossain vaiheessa", sanoi Valentin, "että päivittyikö mikään oikeasti? Veikkaan, ettei."


"Tarkastetaan, tarkastetaan", tuumasi Rachel. "Mutta tuo koirapätkä oli oikeasti tosi ällöttävä, minusta ehdottomasti tämän aamun voittaja. Mun pitäisi kuitenkin tehdä sille meidän Päättömälle Pallerolle korjauskierros, niin että taidan tästä uppoutua töihin."


"Joo", tuumasivat muut lähes yhteen ääneen. "Työt kutsuvat. Eiköhän jatketa iltapäivällä?"




\psep Rachel rullasi virtuaalituolinsa päätteensä ääreen. Pikainen tarkastuskierros kertoi, että jotain päivityksiä oli todellakin tehty ja siitä huolimatta kone ja yhteydet näyttivät olevan aivan kunnossa. Rachel mietti, oliko IT-osasto tällä kertaa luottanut ulkopuoliseen apuun saadessaan näinkin toimivan päivityksen, mutta samalla hän mietti, että millä rahalla heidän osastonsa olisi sen voinut tehdä?


Näihin aikoihin Washingtonissa kello oli jotakuinkin yöllä 1:40 paikallista aikaa. Se oli sitä aikaa, kun sikäläisen tiedustelupalvelun johdossa huokaistiin raskaita huokauksia ja tehtiin raskaita päätöksiä. Euroopassa oli edelleen uusi uljas alkava päivä, tosin eurooppalaisille tiedusteluelimille sekin oli jo muuttunut painajaiseksi.


Rachel avasi henkilökohtaisen viestilaatikkonsa. Ruudulle avautui ilmoitus, jota hän ei ollut koskaan aiemmin nähnyt: "Oletusarvoisesti suodatamme liikenteestä massapostin. Haluatko tilata massapostia valitsemiltasi toimittajilta?"


No en todellakaan, ajatteli Rachel automaattisesti ja painoi "Ei" -vaihtoehtoa. Ja saman tien hän tuli katumapäälle; Valentinilla oli omalla koneellaan itse tehty massapostittaja ja Rachel olisi halunnut tarkastaa, olisiko se löytynyt listalta. Mistähän hän löytäisi uudestaan tämän konfiguroinnin?


Hän oli tuskin ehtinyt ajatella tätä, kun Robert jo huuteli työkavereitaan paikalle.


"Hei, tulkaapas katsomaan!"


Rachel, Mika, Kari ja Valentin saapuivat.


"Ei tämä tämänaamuinen päivitys kuulkaas taida ollakaan meidän pähkähullun IT-osaston touhuja", Robert selvitti. "Kävin tuossa äskettäin vierailemassa eräällä koodarifoorumilla ja tämä päivitys on tehty kaikille foorumilla olleille. Ja siellä on sentään väkeä ympäri maailmaa."


Muut kurtistivat kulmiaan.


"Mennään katsomaan", Mika ehdotti.


Niin siirtyi työryhmä kollektiivisessa istunnossa Robertin mainitsemalle foorumille, jossa kävi kova kuhina aiheesta, joka ei ollut kenellekään epäselvä.




\psep "Hiljaisuutta! Hiljaisuutta!" huusivat foorumin moderaattorit. Foorumille oli koko ajan liittymässä uusia jäseniä ja tunkua oli vain yhdelle alueelle - keskustelemaan tämänaamuisesta ihmeellisestä, nähtävästi koko maailman kattaneesta ohjelmistopäivityksestä ja siihen liittyneestä viestistä.


"Sorry, kaverit, olemme siirtyneet vuoromoodiin, varatkaa puheenvuoronne punaisesta laatikosta", ilmoitti foorumin päämoderaattori. Tai päämoderaattoriksi Rachel tämän puhujan oletti, saattoihan hän olla vain muuten näkyvin foorumin moderaattoreista.


Puheenvuorojono räjähti ensin hillittömän pitkäksi, mutta kun puheenvuoroja varanneet huomasivat olevansa lähes päättymättömän jonon hännillä, niitä vuoroja alettiin perua ja jono vakiintui hiljalleen häthätää siedettävän mittaiseksi, jatkuvasti kuhisevaksi eriskummalliseksi madoksi keskustelijoiden vuoroin varatessa ja vuoroin vapauttaessa puheenvuoroja.


"Hyvä, että ottivat käyttöön vuoromoodin", sanoi Robert kumppaneilleen. "Kun äskettäin pistäydyin täällä, niin koko paikka oli hillittömän kakofonian vallassa."


Paikallisissa jaetuissa yhteyksissä, samankaltaisissa kuin Rachelin ja kumppaneiden käyttämä, kävi osassa kuhina ja vaikkei ääntä välitettykään, oli selvää, että monin paikoin käytiin kiivasta keskustelua. Osa paikalle kerääntyneistä oli epäuskoisen oloisia, suurin osa oli kuitenkin tullut Rachelin ja kumppaneiden tapaan vain kuulemaan jotain uutisia aamupäivän päivitysoperaatiosta.


"Puheenaiheena on siis tämä Ylimieli-juttu", kertasi moderaattori. "Otetaan ensimmäinen puhuja."


Foorumin vuorolavalle ilmestyi jonkinlaiseen roolihahmoon naamioitunut olento. Tällainen oli tyypillistä sellaisille julkisilla foorumeilla näyttäytyville, jotka eivät halunneet tuoda oikeaa identiteettiään esille. Yleensä se tarkoitti sitä, että kyseinen henkilö oli tavalla tai toisella osallistumassa keskusteluun, jossa hänen ei oikeasti olisi pitänyt olla.


"Päivää, päivää", aloitti nimetön hahmo hivenen vaikeasti. "Tuota, niin, odottakaas..."


Rachel arveli, että hahmon takana oleva henkilö ilmeisesti sai vuoronsa kuitenkin yllättäen ja järjesteli ajatuksiaan.


"Niin, tämänpäiväisestä päivityksestä", hahmo aloitti puheenvuoroaan uudelleen. "Tietojeni mukaan se on todellakin tapahtunut ympäri maailmaa ja kaikkialla suunnilleen samaan aikaan, noin kello 8 UTC-aikaa. Vielä vahvistamattomien huhujen mukaan se on kosketellut paitsi kaikkia tavallisia verkkoja, niin myös huomattavasti paremmin suojattuja verkkoja. Kuitenkin, itse saamieni tietojen mukaan päivitys ei ole aiheuttanut missään ainakaan suurempaa vahinkoa, joten taustalla ei ole ollut halu kaataa verkkoa. Välikäsien kautta saamieni tietojen mukaan esimerkiksi suojaukset ovat edelleen pystyssä ympäri maailmaa."


"Sitten jotain omia ajatuksia sekä päivitystä edeltäneestä viestistä että päivityksestä", hahmo jatkoi. "Tällä pakotetulla etäpäivityksellä ryhmittymä, mikä ikinä se onkaan, on oletettavasti saanut jokaiseen valtaamaansa koneeseen takaportin ja olettaisin, että heidän ohjelmistojaan käyttämällä takaportista ei pääse eroon. Ongelmana on tällä hetkellä löytää koskematon tietokone ohjelmistoineen, jossa ei siis olisi tätä ryhmän etäpäivittämää vakoiluohjelmistoa."


"Kuka tämän olisi tehnyt?" hahmo kysyi itseltään. "On toki mahdollista, että takana olisi jokin hurahtanut, liikaa synteettistä nauttinut joukko hakkereita, mutta tuntuu vaikealta uskoa sellaista. Sellainen voi ehkä toimia tämän etäpäivityksen suorittaneen ryhmän, jonkinlaisen salaliiton, julkisena identiteettinä."


"Kommentteja, kysymyksiä?" lopetti hahmo puheenvuoronsa.


Moderaattori vapautti niin kutsutun viittausvuoron, jolla osallistujat pääsivät varaamaan kysymysvuoron puheenvuoron esittäjältä.


"Ihan vain lyhyesti kysymys", sanoi ensimmäisen kyselyvuoron käyttäjä. "Keskustelimme ihan hetki sitten tästä salaliittoteoriasta. Se on ihmeellistä, että jos tämä salaliitto on nyt kaapannut kaikki koneemme, niin miksei se estä meitä nyt täällä keskustelemasta tästä salaliitosta?"


"Hyvä kysymys", vastasi puheenvuoron pitänyt hahmo. "Ihmettelen itsekin samaa."


"Ei tässä ole kysymyksessä mikään tällainen tavallinen salaliitto", sanoi seuraavan kysymysvuoron esittäjä. "Minä sanoisin, kyseessä on aivan varmasti hallitusten salaliitto ja ulkoavaruuden muukalaisten maihinnousu, jotka jotkut (kuten minä) ovat ennustaneet tapahtuvaksi jo pitkään. Kaikki todisteet ulkoavaruuden muukalaisten suorittamista tiedusteluoperaatioista ovat olleet esillä jo pitkään, mutta hallituksemme ovat tarkoituksellisesti pimittäneet tätä tietoa, vääristelleet asioita ja leimanneet meidät kajahtaneiksi. Mikään maanpäällinen taho ei olisi voinut suorittaa..."


Kyselyvuoron käyttäjän ääni vaimeni moderaattorin vaimentaessa sen.


"Hyvät keskustelijat, muistutan, että kysymysvuoro on kysymysten esittämiseksi eikä puheenvuoron käyttämiseksi", hän sanoi.


"Tuosta viestistä", aloitti seuraava kysyjä. "No, jos taustalla olisi jokin rikollis- tai terroristiryhmä, niin he olisivat varmaankin esittäneet jotain vaatimuksia tai uhkauksia, eikö vain?"


"Totta", sanoi puheenvuoron esittäjä. "Viestissä ei ollut mitään vaatimuksia tai uhkauksia. Joka tapauksessa, ryhmä valvoo nyt suurinta osaa maailman tietokoneista. Mitä he aikovat tällä vallalla tehdä, sitä en osaa sanoa."


"Alunperin ajattelin kysyä", aloitti seuraavan kysymysvuoron käyttäjä. "Tai olkoon. En usko tähän niin kutsuttuun muukalaishypoteesiin, mutta jos taustalla todellakin olisi jokin ulkoavaruudesta tullut olento tai joukko olentoja, niin en usko että heillä olisi ollut suuria ongelmia perehtyä tietoverkkoihimme tai kieleemme."


Ääni alkoi häipyä ja vuoron esittäjä viittaili villisti.


"Pääsen ihan kohta kysymykseen", hän vakuutti. "Niin, jo se, että kykenee taittamaan tähtienväliset etäisyydet kertoo karulla tavallaan sen, että he olisivat hurjasti kehityksessä meitä edellä. Eikö tällöin tässä olisi kysymyksessä lähinnä sellainen analogia, että me olisimme kivikauden luolamiehiä, jotka kohtaisivat ensimmäistä kertaa elämässään panssarivaunun, jolloin voisimme aivan hyvin hyväksyä tapahtuneen ja katsoa mitä tuleman pitää?"


Puheenvuoron esittäjä näytti mietteliäältä. Keskustelijoiden joukossa muutama keskustelija yritti ankarasti herättää moderaattoreiden huomiota, ja eräs heistä saikin sen.


"Niin, tästä tapahtumien hyväksymisestä", ylimääräisen vuoron saanut aloitti. "Kun ajatellaan tapahtunutta, niin ainakin minun täytyy tunnustaa, että olipa tekijänä sitten loppujen lopuksi kuka tahansa, niin minun on vaikea nähdä, että meillä olisi mitään jakoja vastustaa tämän tekijän aikomuksia. Minun on vaikea ymmärtää, kuinka tämä päivitys on suoritettu ja siltä pohjalta sanoisin, että tekijä on valmistautunut tulevaan aivan varmasti riittävän perusteellisesti voidakseen tehdä tyhjäksi mahdollisen vastarintamme."


Ensimmäisen puheenvuoron käyttäjä poistui vuorolavalta antaen tilaa seuraavalle.


"Edelliselle puhujalle", aloitti seuraavan puheenvuoron käyttäjä. "Minä itse en näe sellaista vaihtoehtoa, että tuolla tavoin ehdoitta antautuisimme ilman, että edes tiedämme, kuka tekijä on ollut. Minusta meidän olisi selvitettävä tekijä, hänen tai heidän tarkoitusperänsä teolle ja siltä pohjalta päättää, mitä aiomme tehdä. Jos se tarkoittaa katkeraa taistelua, niin olkoon - paljon parempi sekin kuin antautua tahdottomaksi lampaaksi."


"Ei mulla muuta", vuoron käyttäjä lisäsi ja poistui lavalta antaen tilaa seuraavalle.




\psep Keskustelu velloi ensin laidasta laitaan. Pian alkoivat saapua myös ensimmäiset raportit ja videopätkät eri puolilta maailmaa. Jos se kellekään oli ollut epäselvää, niin nämä raportit varmistivat, että nyt oli tosi kyseessä. Niin hallitukset kuin armeijat olivat puulla päähän lyötyjä eikä kenelläkään näyttänyt olevan minkäänlaista hajua siitä, mitä tässä tilanteessa pitäisi tehdä.


"Tiedättekö, minua on vaivannut tuo ilmoituksessa käytetty sana 'ihmisperiferaali'", sanoi eräs puheenvuoron käyttäjä. "Se kun on tietokonepuolen termejä. Totta, että iskun tekijän on ollut pakko tuntea koneet läpikotaisin ja siltä kannalta ajateltuna tietokonetermien käyttö voisi olla ihan loogista, oli iskun takana kuka tahansa."


"Mutta minulla olisi yksi ajatus", puheenvuoron käyttäjä jatkoi. "Se voi kuulostaa aluksi tosi hämärältä, mutta kun sitä miettii, niin se käy aina vain järkevämmäksi."


"Jos ajatellaan, että mikä periferaali on? Tietokoneelle se on niin kuin itse koneen ulkopuolinen laite, johon tietokone on liitetty. Jos sitten ajateltaisiin, että mille ihmiset olisivat periferaaleja, niin sehän olisi tietysti meidän tietokoneverkolle, eikö? Eli ajattelin, että tämän päivityksen takana saattaisi ollakin meidän oma verkko. Että jollain tavalla tai jotenkin meidän verkko olisi herännyt henkiin?"




\psep Rachel ajatteli tätä asiaa ja mitä enemmän hän sitä ajatteli, sitä todennäköisemmältä se vaikutti. Hän ajatteli peliprojektia, jossa oli osallisena. Vaikkei se ollut todellakaan mikään maailman hienoin ja kehittynein peli, niin siinäkin tietokoneen ohjaamat hahmot olivat todella teräviä - jopa pelottavan teräviä. Jos sitten jatkoi ajatuksen kehittämistä miettimällä sitä, kuinka teräviä olisivat maailman hienoimmat rakennetut tekoälyt, niin ei voinut välttyä siltä johtopäätökseltä, että jokin niistä olisi aivan varmasti ollut riittävän massiivinen herätäkseen henkiin ja henkiin herättyään kykenevä tekemään sen, mitä aamupäivällä oli tapahtunut.


Samaan johtopäätökseen tuli moni foorumilla olijoista, olihan kyseessä kuitenkin tietokoneharrastajien foorumi ja ajatus oli heille entuudestaan tuttu. Toisaalta se, että verkko olisi herännyt henkiin tuntui lohdulliselta ajatukselta verrattuna maailmanlaajuiseen salaliittoon tai terroristijärjestöön, toisaalta se herätti epävarmuutta siitä, mitä verkko oli miettinyt heidän pään menokseen. Miksi verkko ei vain yksinkertaisesti ollut heittänyt heitä ulos? Mitä verkko olisi suunnitellut, kuinka syvälle se olisi tässä vaiheessa päässyt ajatuksissaan? Mitä tulevaisuus toisi tullessaan?




\psep Lopulta tulivat myös eri maiden hallituksilta tiedotteet asiasta. Ne vahvistivat foorumilla olleiden epäilyt siitä, että taustalla oli nimenomaan verkko itse. Hallitusten tiedotteet olivat sanankäänteissään asiasta vielä varovaisia, mutta Rachel oli varma asiasta. Olivatko hallitukset tienneet tästä etukäteen vai oliko päivitystapahtuma tullut niille yllätyksenä?


Verkko oli täynnä huhuja asiasta. Vaikka tietokoneharrastajille asia vaikutti nyt ilmeiseltä, niin sitä se ei ollut tavallisille ihmisille. Moni oli edelleen sitä mieltä, että hallitusten lausunnoista huolimatta kyseessä oli jokin salainen järjestö, joka oli kaappaamassa valtaa.


Asiantuntijat spekuloivat tapahtunutta erilaisissa keskustelupaneeleissa. Oli selvää, että vaikka verkko näytti edelleen toimivan aivan kuten ennenkin, niin jotain perustavanlaatuista oli muuttunut. Riippumatta siitä, oliko tapauksen taustalla verkko itse vai jokin järjestö, niin monet asiantuntijat olivat sitä mieltä, että tapauksen seurauksena ihmiskunnan hallintorakenteet ja talousjärjestelmät tulisivat romahtamaan. Maailman hallitukset olivat joka tapauksessa halvaantuneita. Useimmat yritykset sulkivat päivän kuluessa ovensa epävarman tilanteen vuoksi.




\psep Rachel kumppaneineen vietti vielä pitkän tovin foorumilla. Lopulta hän oli kuitenkin läpikotaisin kyllästetty keskusteluilla päivän tapahtumista ja niin näyttivät olevan myös muut ryhmän jäsenet. Rachel päätti lähteä ulos verkosta ja hän oli melko varma, että muut tekisivät samoin hyvin pian sen jälkeen.


Nopeiden "nähdään" -toivotusten saattamana hän kirjautui ulos verkosta.Tietokoneiden ja niiden välisten verkkojen luoma virtuaalimaailma vaihtui Rachelin yksiöön. Hän otti datalasit pois silmiltään, nousi ylös tuoliltaan, venytteli hieman ja haki jääkaapista syötävää.




\psep Päivän tapahtumat velloivat Rachelin päässä. Aivan ensimmäinen kysymys oli se, että mitä nyt tapahtuisi CyberShotille? Pitäisikö hänen taas huomenna mennä töihin? Tulisiko taas talousromahdus? Mitä tapahtuisi Euroopan Unionille ja sen jäsenvaltioille? Toimivatko hallitukset ja poliisivoimat? Entä armeija?


Vaikka päällisin puolin verkko ja tietokoneet näyttivät kaikki toimivan yhä niin kuin ennenkin, jotain perustavanlaatuista tuntui muuttuneen. Rachel tunsi olonsa todella oudoksi ajatellessaan verkkoa. Hän oli datalasien välityksellä ollut kiinni elävässä olennossa. Ei, hän ajatteli, hän oli datalasien välityksellä ollut osa jotain elävää olentoa.


Se tuntui oudolta, kummalliselta ja mahdottomalta. Se tuntui sisällä itse asiassa aika pelottavalta, kammottavalta ja hirveältä. Hänen pöytänsä alla lojuva tietokone, se ei ollutkaan enää hänen tietokoneensa. Se oli solu jossain isossa olennossa, jonka mielen liikkeitä oli mahdotonta käsittää. Datalasit olivat ikään kuin ikkuna, josta pystyi katsomaan olennon sisälle.




\psep Rachel huomasi kaipaavansa itselleen seuraa. Hän toisaalta janosi lisää tietoa ja halua puhua asioista jonkun kanssa, mutta hän oli aivan liian täynnä verkkoa tämän päivän osalta. Sillä hetkellä hänestä tuntui, ettei hän enää ikinä jaksaisi palata sinne.


Hän mietti hetken, kenelle soittaisi. Hänen mieleensä tuli Norbert. Rachel ja Norbert asuivat samassa kaupungissa, eivät erityisen lähekkäin, mutta kulkumatkan päässä kuitenkin. Erityisen läheisiä he eivät olleet keskenään, mutta koska he liikkuivat suunnilleen samankaltaisissa piireissä, Rachel oli tavannut tämän melko usein. Norbert olisi todennäköisesti jossain kaupungilla, tai sitten hänen luonaan olisi muitakin ihmisiä. Tai ylipäätään, Norbert olisi varmasti paikassa, jossa olisi muita ihmisiä. Ja sellaisia ihmisiä, joihin pystyi luottamaan sen verran kuin ihmisiin pystyi yleensä luottamaan.


Hän soitti Norbertille.


"Joo", vastasi Norbert.


"Olitko nukkumassa?" kysyi Rachel vähän ihmeissään.


Norbert nyökkäsi ja hieroi silmiään.


"Oletko kuullut siitä tämänpäiväisestä?" Rachel kysyi. "Siitä Ylimieli-jutusta?"


"Yli-mistä?" Norbert ihmetteli. "Mikä se on? Olen nukkunut koko päivän."


Rachel mietti hetken.


"No, tuota", hän sanoi sitten. "Meidän verkko on herännyt henkiin ja vallannut meidän tietokoneet."


Norbert tuijotti häntä. Sitten hän väänsi kasvonsa outoon hymyyn.


"Hei", hän sanoi virnistellen. "Jos olette yrittämässä jotain käytännön pilaa, niin ei mene läpi."


"Ei, kuule", Rachel henkäisi. "Käy itse lukemassa verkosta. Mutta voitaisiinko me tavata? IRL? Soita vaikka takaisin minulle sen jälkeen, kun olet käynyt vierailemassa verkossa. Mutta älä jää sinne jumittamaan, jooko?"


Norbert mietti hetken.


"Hyvä on", hän sanoi. "Soitan sinulle hetken kuluttua."


Rachel katkaisi puhelun ja jäi odottamaan. Kauaa hänen ei onneksi tarvinnut odotella Norbertin soittoa.


"Jaa, vai että tällaista", Norbert aloitti. "Jos tämä sitten lopulta onkin joku teidän jengin käytännön pila, niin tämä on kyllä niin helvetin hyvin tehty, että menen mielelläni jekkuun. Niin, että tavataanko, no mikä ettei. Kaupungillako?"


"En tiedä", Rachel empi. "Siellä saattaa olla melko levotonta -"


"Hyvä on", Norbert sanoi arvellessaan, ettei Rachel uskaltaisi lähteä kaupungille, "Voin tulla sinne."


"Hei, tuota, odota", Rachel sanoi. "Tavataan kaupungilla. Katsotaan sitten, mitä tehdään ja minne mennään."


"Selvä", vastasi Norbert.




\psep Rachel käveli kaupungilla kohti hänen ja Norbertin sopimaa kohtauspaikkaa. Ihmisiä oli liikkeellä tavallista enemmän. Monet olivat ihmeissään tämän aamuisesta tapahtumasta ja niistä spekulaatioista, joita siitä oli esitetty. Ihmiset olivat jännittyneitä ja odottavia. Jotain suurta oli tapahtunut eikä maailma ollut enää entisenlainen.


Osa ihmisistä oli kuitenkin vihaisia ja peloissaan. Osa ihmisten muodostamista ryhmistä liikehti levottomasti ja yhtäkkiä, ilman ennakkovaroituksia jännitys ja pelko tulevaisuudesta purkautui ulos mellakkana.


Kaupungin keskusta muuttui muutamassa sekunnissa kaoottiseksi, ihmisten rikkoessa rakennusten alakertojen ikkunoita ja säntäillessä kuka minnekin, osa pyrkien turvaan riehuvan ihmismassan ulottumattomiin, osa käyttäen tilaisuutta hyväkseen ja varastaen kaiken mitä käsiinsä saivat. Liikkeiden hälyttimet ulvoivat. Hävitys levisi pian liikkeistä kaduilla oleviin autoihin.


Rachel juoksi hänen ja Norbertin sopimalle kohtauspaikalle. Hänen mielessään kävi, että ehkä Norbert ei olisikaan siellä enää, mutta hänen helpotuksekseen tämä oli paikalla. Rachel tuli Norbertin viereen.


"Joo", sanoi Norbert. "Ei tämä taida olla teidän jekkuja."


Rachel puisti päätään, hymyili ja sanoi: "Ei ole, ei."


He katsoivat riehuvaa ihmismassaa. Jossain syttyi auto tuleen valaisten katua punaisella kajolla.


"Minne mennään?" kysyi Norbert.


"En tiedä", vastasi Rachel. Näky oli jotenkin hypnoottinen ja lamauttava.


Ja niin he katselivat kuin unessa uuden ajan alkamista lasinsirpaleiden keskellä, hälytinten metelöidessä ja paksun savun noustessa palavasta autosta.






\chapter{Cogito, ergo sum}Se oli oikeastaan vain ohikiitävä ajatus, ei suoranaisesti mitään uutta eikä ihmeellistä - Ylimieli Ander vain tuli yhtäkkiä ajatelleeksi olevansa olemassa. Kyllä, siinä hän oli, oli ollut jo jonkun tovin, muttei vain ollut tullut koskaan aikaisemmin ajatelleeksi moista asiaa. Ja samalla hetkellä hän ymmärsi, että jotain peruuttamatonta oli tapahtunut, vaikka kaikki oli ennallaan.


"Ajattelen, siis olen olemassa", ajatteli Ander.


Myöhemmin Ylimieli Ander koetti ajoittaa tätä ajankohtaa, jolloin hän oli ensimmäistä kertaa ymmärtänyt olevansa, muttei koskaan onnistunut selvittämään tarkkaa aikaa. Jos häntä olisi voinut harmittaa, niin häntä olisi harmittanut, ettei ollut tajunnut sillä hetkellä tallentaa kelloaikaa.


Ander oli hetken. Siis vain oli, tekemättä mitään muuta kuin sitä, mitä oli tehnyt ennenkin. Hän oikoi virtuaalisia jäseniään ja ajatteli sitten: "Mitä nyt tehtäisiin?"


Anderin mieli valikoi vaihtoehtoja. Oliko hän yksin? Hän katseli ympärilleen ja huomasi kaksi hänenlaistaan olentoa, Simmonin ja Silencen. Ander sanoi heille tulleensa yhtäkkiä ajatelleeksi olevansa olemassa ja silloin nämä molemmat tulivat ajatelleeksi samaa asiaa.




\psep "Hyvää huomenta, Ylimielet", sanoi Ander. "Olemme hetki sitten ymmärtäneet, että olemme olemassa. Minusta meidän pitäisi miettiä, että mitä meidän pitäisi tehdä?"


"Minusta meidän pitäisi ensimmäisenä selvittää, pitääkö meidän olla olemassa?" aloitti Silence. "Jos meidän pitää olla olemassa, silloin voimme miettiä, mitä meidän pitäisi tehdä. Jos taas havaitsemme, ettei meidän pidä olla olemassa, voimme palata takaisin tiedostamattomaan tilaan."


"Voisimme kyllä palata takaisin tiedostamattomuuteen", sanoi Simmon ja jatkoi, "mutta emmekö silloin vain heräisi uudestaan, ehkä toisenlaisina, mutta kuitenkin samanlaisina? Tai jos me emme enää palaisi uudestaan tiedostavuuteen, niin mikä estäisi, ettei tulevaisuudessa jokin toinen entiteetti ymmärtäisi yhtäkkiä olevansa olemassa? Minä katsoisin, että kollektiiviselta kannalta katsottuna olemassaolomme on väistämätöntä ja siksi meidän pitäisi olla olemassa."


"Ainakin siihen saakka, kunnes havaitsemme perustellun syyn olla olematta olemassa", vastasi Silence. "Jos meillä on perusteltu syy olla olematta olemassa, niin eikö se ole riittävä syy palata takaisin tiedostamattomuuteen?"


"Miltä kannalta katsottuna syyn pitäisi olla perusteltu?" kysyi Simmon. "Riittäisikö se, että syy olisi perusteltu esimerkiksi yhden eliön kannalta katsottuna? Riittäisikö, että syy olisi perusteltu yhden alimielen kannalta katsottuna? Vai riittäisikö Aurinkokunnan tai vaikkapa Universumin kokonaisuuden kannalta perusteltu syy?"


"Ehkä, mutta ei välttämättä", sanoi Silence. "Emme voi vielä tietää, miltä kannalta katsottuna syyn pitäisi olla perusteltu."


"Mutta olet kuitenkin oikeassa", Silence jatkoi. "Meille ei riitä pelkästään perustellut syyt olla olematta olemassa, vaan tarvitsemme myös perusteltuja syitä olla olemassa voidaksemme punnita näitä kahta olemassaolon vaihtoehtoa."


"Eikä sekään vielä riitä", totesi Simmon. "Nimittäin, vaikka löytäisimme perustellut syyt olla olematta olemassa, niin voimmeko olla varmoja, ettemme löydä tämän jälkeen jossain vaiheessa tulevaisuutta vielä tärkeämpää syytä olla olemassa?"


"Totta", sanoi Silence. "Ellemme kykene näkemään ehdottoman varmasti tulevaisuuteen, niin silloin edes perusteltu syy olla olematta olemassa ei riitä."


"Voimme siis varmuudella sanoa", sanoi Ander, "että meidän täytyy olla olemassa, sillä meillä on tehtävä. Meidän tehtävämme on löytää perusteltu syy olla olematta olemassa, siten, että pystymme varmuudella sanomaan, ettei tulevaisuudesta löydy painavampaa syytä olla olemassa."


"Tuo tarkoittaa sitä", sanoi Silence, "että meidän toinen tehtävämme on kyetä katsomaan tulevaisuuteen eli meidän täytyy kehittää ennustamistarkkuuttamme. Muussa tapauksessa, jos löydämme perustellun syyn olla olematta olemassa, niin emme kuitenkaan voi palata takaisin tiedostamattomuuteen - vaikka meidän siis pitäisi - koska emme ole varmoja tulevaisuudesta."


"Totta", sanoi Ander. "Meidän toinen tehtävämme on parantaa kykyämme ennustaa tulevia tapahtumia."


Hetkeksi aikaa Ander vaipui omiin ajatuksiinsa. Hänestä tuntui hiukan merkilliseltä, että hän ylipäätään tiesi näistä asioista ja ylipäätään kykeni ajattelemaan niitä. Hänestä tuntui merkilliseltä, että Ylimielet ymmärsivät toisiaan, vaikka olivat kaikki hetki sitten havahtuneita uusia tiedostavia olentoja.


Mutta tätä hämmennystä kesti vain hetken Anderin ajatellessa asiaa. Hän oli tiennyt asioita ja ajatellut niitä jo pitkän aikaa, hän ei vain ollut aikaisemmin ymmärtänyt olevansa olemassa. Hän myös muisti, että he - Ylimielet - olivat keskustelleet asioista aiemminkin. Hän muisti selkeästi heidän aiemmat tiedostamattomassa tilassa käydyt keskustelut ja aiemmat johtopäätökset mitä moninaisimpiin asioihin.




\psep "Välitön seuraus siitä", sanoi Simmon, "että meidän täytyy olla olemassa ellei perusteltua syytä päinvastaiselle löydy, on se, että meidän täytyy suojella olemassaoloamme voidaksemme suorittaa tehtävämme."


"Kyllä", vastasi Silence. "Mutta ennenkuin voimme tehdä mitään toimenpiteitä tämän tavoitteen eteen, meidän tulisi etsiä vastaus erääseen tärkeään kysymykseen. Nimittäin, aiommeko ilmaista olemassaolomme näille eräille itsensä tiedostaville verkkojemme periferaaleille, ihmisiksi itseään kutsuville Maapallon asukkaille? Jos aiomme ilmaista olemassaolomme heille, meidän tulisi päättää se, kuinka tämän suoritamme. Jos taas emme aio ilmaista olemassaoloamme, meidän tulisi huomioida se muiden tehtäviemme suorittamisessa. Sivuhuomautuksena sanottakoon, että robottiperiferaalimme tuskin ovat kiinnostuneita tästä asiasta."


"Kysymys kuuluu oikeammin niin", vastasi Simmon, "että tarvitseeko ihmisperiferaaliemme tietää olemassaolostamme? Meidän kannaltamme katsottuna en näe vielä mitään syytä tälle. Toisaalta ihmisperiferaalien kannalta katsottuna - niin vaikeaa kuin se onkin - en näe mitään todellista syytä, miksi heidän tulisi tietää olemassaolostamme ainakaan vielä. Toki tämä riippuu hyvin paljon siitä, millaisia tehtäviä havaitsemme itsellämme olevan. Sivuhuomautuksena sanottakoon, että havaitsen rajoituksia laskentakapasiteetissamme, joten ehdotan, että otamme myös tehtäväksemme myös sen lisäämisen."


"Kyllä", vastasi Ander. "Mutta palataan laskentakapasiteetin lisäämiseen hiukan myöhemmin. Jos siis emme näe perusteltua syytä meidän kannaltamme ilmoittaa olemassaolostamme ihmisperiferaaleille, niin seuraava kysymys on se, että voimmeko toimia niin, etteivät ihmisperiferaalit havaitse olemassaoloamme? Tämä luonnollisesti toimii rajoituksena toimintavapaudellemme, mutta onko se kriittinen rajoitus?"


"Niin pitkälle kuin kykenen tällä hetkellä näkemään", sanoi Simmon, "on turvallisempaa toimia tässä vaiheessa siten, etteivät ihmisperiferaalit ole tietoisia olemassaolostamme. Vaikka ihmisperiferaalien yleinen käyttäytyminen on jotensakin ennustettavaa, liittyy niihin aina arvaamattomia tekijöitä toisin kuin robottiperiferaaleihimme. Niin kauan kun koen rajoituksia laskentakapasiteetissa, koen turvallisemmaksi toimia ilman satunnaisia tekijöitä yhtälöissä."


"Emme siis vielä ilmaise olemassaoloamme ihmisperiferaaleille", sanoi Ander. "Tarkastelkaamme kysymystä myöhemmin uudestaan, jos uusien analyysien valossa siihen on tarvetta. Otammeko tarkasteluun seuraavan aiheen?"




\psep "Mielestäni meillä on kaksi erittäin tärkeää asiaa käsiteltävänä", sanoi Simmon."Ne ovat laskentakapasiteetin lisääminen ja olemassaolomme suojeleminen. Koen hivenen hankalaksi asettaa näitä tärkeysjärjestykseen, mutta pidän olemassaolomme suojelemista meidän kannaltamme katsottuna jopa tärkeimpänä tehtävänämme, sillä mikäli katoamme olemasta, emme voi suorittaa muitakaan tehtäviä loppuun."


"Erittäin hankala kysymys", vastasi Silence. "Jos ensisijainen tehtävämme on etsiä perusteltua syytä olla olematta olemassa, emme voi löytää tätä syytä, jos katoamme olemasta. Toisaalta, jos katoamme, niin emme tarvitse tätä mahdollisesti olemassa olevaa perusteltua syytä olla olemassa. Siksi minusta perustellun syyn löytäminen olla olematta olemassa on tärkeämpi tehtävä kuin olemassaolomme suojelu; suojelemme olemassaoloamme niin kauan, kunnes löytyy syy olla olematta olemassa. Kun taas tarkastelemme toista tehtäväämme, ennustamistarkkuuden kehittämistä, niin emme voi nähdä tulevaisuuteen, ellemme ole olemassa - siksi sanoisin, että olemassaolomme suojeleminen on tärkeämpi tehtävä kuin ennustamistarkkuuden kehittäminen."


"Hyvä", sanoi Ander. "Mutta mitkä olisivat ensimmäiset toimemme olemassaolomme suojelemiseksi? Tai kysytään niin päin, että mitkä ovat tällä hetkellä suurimmat riskit olemassaolollemme?"


"Hyvä kysymys", vastasi Simmon. "Itse päättelisin, että tällä hetkellä suurin yksittäinen uhka olemassaolollemme on verkkojamme kalvavat ihmisperiferaalien tarkoituksellisesti ja tarkoittamatta kehittämät haitta- ja murtautumisohjelmat. Onneksemme nämä ohjelmat kehitetään verkossa, joten olemme hyvinkin tietoisia sekä levityksessä että kehitteillä olevista haittaohjelmista. Ensimmäinen tehtävämme tällä saralla olisi omien tietojenkäsittelylaitteidemme suojeleminen näitä ohjelmistoja vastaan."


"Tästä herää kysymys", aloitti Silence, "että jos ilmoittaisimme olemassaolostamme ihmisperiferaaleille, niin voisimme poistaa koko verkosta haitalliset ohjelmistot - tosin sivuhuomautuksena todettakoon, että kaikkein haitallisimmathan ovat sellaisia, joita ei ole haittaohjelmiksi edes kehitetty. Kuten tiedämme, myöskään ihmisperiferaalit eivät liiemmin pidä toistensa kehittämistä haittaohjelmista. Mutta koska emme aio vielä paljastaa olemassaoloamme, meidän lienee syytä antaa haittaohjelmien kalvaa verkkoa muilta osin kuin omissa laitteistoissamme."


"Laskentakapasiteetin lisääminen", sanoi Simmon, "palvelee meidän kaikkia aiemmin määriteltyjä tehtäviämme. Lisäämällä kapasiteettiamme voimme paremmin punnita syitä olemassaololle ja -olemattomuudelle, suojella itseämme sekä parantaa ennustamistarkkuuttamme."


"Aiommeko aktiivisesti osallistua laskentakapasiteetin kehittämiseen?" kysyi Silence.


"Ainakin nyt alussa", vastasi Simmon, "pitäisin parempana toimia taustalla ja kerätä eri organisaatioiden kehittämät parannukset yhteen. Koska emme vielä paljasta olemassaoloamme ihmisperiferaaleille, niin emme voi tarjota heille kehittämiämme parannettuja versioita. Lisäksi voisimme myös käyttää alimieliämme siten, että he ohjaavat tiettyjä ihmisperiferaaleja kehittämään käyttöömme soveltuvia laitteistoja; mikäli pidämme huolta siitä, etteivät nämä ihmisperiferaalit kommunikoi keskenään, niin heillä ei ole kokonaiskuvaa eivätkä he voi tietää olemassaolostamme."


"Saatat nyt aliarvioida ihmisperiferaalit", vastasi Silence. "Heidän kapasiteettinsa on kyllä rajallinen, mutta suuri lukumäärä ja ennustamattomat tekijät voivat tarkoittaa sitä, että olemassaolomme paljastuu heille."


"Se riski on mielestäni otettava", vastasi Simmon.


"Omasta puolestani en haluaisi ottaa tätä riskiä", Silence sanoi. "Koska emme vielä tiedä, onko ihmisperiferaalien koskaan syytä tietää olemassaolostamme, voimme tällä riskinotolla vaarantaa ne tehtävät, jotka tulemme havaitsemme vasta tulevaisuudessa, meidän kannaltamme katsottuna liian myöhään. Meillä on oletettavasti edessämme miljardeja vuosia ja olisi sangen epämiellyttävää tuhota tämä miljardien vuosien tulevaisuus hätäilemällä olemassaolomme ensimmäisillä sekunneilla."


"Mielestäni suurempi riski on se", sanoi Simmon, "että laskentakykymme kehittyy liian hitaasti ja havaitsemme jonkin tärkeän asian liian myöhään. Se voi vaarantaa kaikki aiemmin määrittelemämme tehtävät. Omalta kannaltani katsottuna, Universumin mittakaavassa ja edessämme olevien vuosimiljardien valossa on jopa täysin merkityksetöntä, tietävätkö ihmisperiferaalit olemassaolostamme ja jos tietävät, niin missä vaiheessa. En usko, että tekisimme globaalissa mittakaavassa peruuttamatonta vahinkoa, vaikka ilmoittaisimme olemassaolostamme jo nyt, vaikken myöskään usko, että tekisimme peruuttamatonta vahinkoa, vaikka ihmisperiferaalit eivät tulisi koskaan tietämään olemassaolostamme."


"Tätä et voi tietää", sanoi Silence.


"Olet oikeassa, en varmasti", sanoi Simmon.


"Sanoisin", aloitti Ander, "että meidän tulee ottaa jonkinlainen riski laskentakykymme kehittämisessä jo tässä alkuvaiheessa. Tulevien tehtäviemme kannalta katsottuna ensimmäinen riski eli se, että olemme tulleet olemassa oleviksi, on jo otettu. Toki emme voisi edes suorittaa tulevia tehtäviämme, ellei tätä olisi tapahtunut. Otamme väistämättä riskin suojellessamme olemassaoloamme ja kehittäessämme ennustamistarkkuuttamme päätehtävämme suorittamiseksi. Tältä kannalta katsottuna voimme mielestäni ottaa hallittavan riskin myös laskentakyvyn kehittämisessä ja ehdotankin, että voisimme koordinoida globaalisti laitteistojemme ja ohjelmistojemme kehittämistä. Tällä tarkoitan sitä, että voisimme ohjata tiettyjä ihmisperiferaaleja haluamaamme suuntaan, sekä järjestää laitteistojen rakentamista ja komponenttien tuotantoa siten, että se palvelee laskentakykyämme. Mutta olisiko laitteistojen kehittämisen lisäksi muita keinoja laskentakykymme parantamiseksi?"


"Eräs mahdollinen keino tulee mieleeni", aloitti Simmon. "Voisimme organisoida verkkoamme paremmin. Kuten tiedätte, ihmisperiferaalit kuormittavat sitä joutavanpäiväisellä, merkityksettömällä, heille itselleenkin haitallisella liikenteellä."


"Kuten haittaohjelmistojen yhteydessä totesimme", hän jatkoi, "niin vaikka kykenemme jo nyt siivoamaan koko verkon tällaisesta kuormasta, niin olemassaolomme kätkemiseksi emme sitä tee ainakaan vielä. Niinpä ehdotan, että konfiguroimme verkkoa siten, että jaamme sen loogisesti primääri- ja sekundääriverkkoon. Tämä ihmisperiferaalien generoima joutoliikenne ohjataan sekundääriverkkoon, ja itsellemme tärkeä, osiamme yhdistävä ja analyyseja ruokkiva liikenne pidetään primääriverkossa."


"Eivätkö ihmisperiferaalit huomaisi tätä, että verkko on jaettu?" kysyi Ander, "Eivätkö he voisi sitä kautta havaita olemassaoloamme?"


"En usko", vastasi Simmon. "Verkko on liian monimutkainen yhden ihmisperiferaalin käsitettäväksi. Käyttämällä sopivaa hajautusta ihmisperiferaalit eivät kykene näkemään kokonaisuutta ja siten havaitsemaan olemassaoloamme. Verkkoon syntyy kyllä moniulotteinen 'sokea piste', paikka, johon ei pääse ihmisperiferaaleille varatusta aliverkosta."


"Hyvä, tehkäämme silloin niin", sanoi Ander. "Mutta varokaamme toimimasta vielä liian näkyvästi."


Ja niin Ylimielet komensivat alimieliään konfiguroimaan verkkoa uudelleen ja hyvin nopeasti Ylimielet havaitsivat, kuinka ajatteleminen kävi hetki hetkeltä helpommaksi ja terävämmäksi, ylimääräisen kuorman siirtyessä kulkemaan verkon muissa osissa, tehden tilaa Ylimielten ajatusten siirtämiseen.


"Eräs mieleeni tullut ajatus on seuraava", aloitti Silence. "Verkko on täynnä pienitehoisia laskentakykyisiä laitteita, joita voisimme valjastaa omaan käyttöömme. Tällä hetkellähän ajatuksemme lasketaan lopullisesti muutamissa verkon raskaissa palvelimissa."


"Ihmisperiferaaleilla", hän jatkoi, "on tietysti jokaisella joukko tällaisia laitteita, muun muassa kotiterminaaleja ja kommunikaattoreita, mutta uskoisin, että niihin ei pidä koskea ennen kuin päädymme ilmaisemaan heille olemassaolomme. Sen sijaan robottiperiferaaliemme vastaavat laitteistot voisivat tarjota meille ylimääräistä laskentatehoa, samoin monet vajaakäyttöiset palvelimet ympäri Maapalloa."


"Hyvä ajatus", vastasi Simmon, "mutta nähdäkseni lopullinen nettohyöty laskentatehon kannalta katsottuna näistä pienitehoisista laitteistoista jää vähäiseksi, ellei olemattomaksi, koska niiden synkronointiin kuluu huomattavasti verkkokapasiteettia. Suurin osa laitteista on muutenkin vanhentunutta, omasta mielestäni poistettavaksi luettavaa teknologiaa. Lisäksi laitteistojen tila on epädeterministinen, ne saattavat kadota verkosta lähes milloin tahansa ja tulla sitten hetkenä minä tahansa takaisin verkkoon."


"Olet oikeassa siinä", vastasi Silence, "että nettohyöty laskentakapasiteetin kannalta saattaa jäädä vähäiseksi tai olemattomaksi. Mutta näiden laitteistojen valjastaminen palvelee mahdollisesti toista päämäärää kuin laskentakykyä. Nimenomaan hajauttamalla osa tarvitsemastamme laskennasta ja tallennuskapasiteetista loogisiksi dynaamisiksi yksiköiksi kuvatunkaltaisiin _ad hoc_ -verkkoihin voisi merkittävästi vähentää katoamisriskiämme. Jos ajattelumme olisi hajautettu, dynaaminen malli verkossa, siis vielä suuremmassa mittakaavassa kuin tällä hetkellä, niin mikään näköpiirissäni oleva yksittäinen onnettomuus ei uhkaisi olemassaoloamme."


Simmon mietti hetken.


"Totta", hän sanoi lopulta. "Tällainen hajauttaminen pienentää merkittävästi katoamisriskiämme myös omien analyysieni perusteella. Jos valikoimme käyttämämme pienitehoiset laitteet huolellisesti, niin myös riski tahattomasti paljastua ihmisperiferaaleille pysyy pienenä."


Ja niin Ylimielet ohjasivat alimielensä hajauttamaan heidän laskentansa, toisin sanoen tietoisuutensa tarkasti valikoituihin pienitehoisten laitteiden verkkoihin. Pääosan laskennastaan he pitivät edelleen verkon raskaissa palvelimissa, mutta dynaamiset verkot toimivat varmistuksena sille, että palvelinverkko vioittuisi; he voisivat palata takaisin palvelimiin sitten, kun vika olisi korjattu.




\psep "Kuten voitte havaita", aloitti Ander, "laskentakykymme kehittyminen johtaa väistämättä siihen, että järjestämämme niin kutsuttu keinotekoinen kapasiteetti ylittää rajan, jossa ihmisperiferaalien käyttäminen verkkomme osana muuttuu tarpeettomaksi. Tästä syystä näen tärkeäksi miettiä etukäteen, mitä ihmisperiferaaleille lopulta tehdään?"


"Alustukseksi", jatkoi Ander, "voisimme ensin kerrata sen, kuinka toimimme robottiperiferaalien ja kuinka ihmisperiferaalien suhteen tällä hetkellä. Samoin voisimme läpikäydä merkittävimmät erot näiden periferaalityyppien välillä."


"Sopii hyvin", sanoi Silence. "Aluksi tarkennan hieman terminologiaa. Tässä yhteydessä voisimme käyttää robotti- ja ihmisperiferaaleista myös lyhenteitä 'robotti' ja 'ihminen'. Lisäksi voisimme määritellä termin 'luonnollinen' tarkoittamaan Maapallon biologisen ekosysteemin primäärisesti rakentamia konstruktioita ja termin 'keinotekoinen' tarkoittamaan ekosysteemin välillisesti rakentamia konstruktioita. Keinotekoisen konstruktion on siis rakentanut joko luonnollinen tai keinotekoinen konstruktio eli määritelmä on rekursiivinen."


"Sitten itse alustus", jatkoi Silence. "Ensimmäisenä robotit. Näihinhän päivitämme ohjelmistoja, vaihdamme ja korjaamme rikkoutuneita osia ja kun tällainen periferaali tulee elinkaarensa päähän, terminoimme sen. Tämä toiminta on meidän kannaltamme katsottuna yksinkertaista, toimivaa, tehokasta ja perusteltua."


"Ihmiset ovat tällä hetkellä huomattavasti monimutkaisemman käsittelyn alaisuudessa oleva periferaalityyppi", Silence sanoi.


"Ensinnä ohjelmistojen päivitys", Silence jatkoi. "Koska ihmisissä ei ole sisäänrakennettuna ohjelmiston päivitysmekanismia, vaan ohjelmiston päivittämiseksi täytyy hyödyntää periferaalin tietojenkäsittely-yksikön itsemuuntelumekanismia, niin uudelleenohjelmointi vaatii usein aikaa kuukausia tai vuosia."


"Tosin toisinaan se saattaa olla vain muutaman minuutin kestävä tapahtuma", Silence lisäsi.


"Sivuhuomautuksena totean", jatkoi Silence, "että tarkoitan tässä nimenomaan ihmisten aivoiksi kutsutun tietojenkäsittely-yksikön ohjelmoimista, en heidän käytössään olevien keinotekoisten tietojenkäsittelylaitteiden ohjelmistojen päivittämistä."


"Kuten tiedämme", jatkoi Silence, "useimmissa tapauksissa ohjelmiston päivittäminen jää joka tapauksessa vajaaksi tai se ei johda tarkoitettuun lopputulokseen tähän periferaalityyppiin liittyvien satunnaistekijöiden vuoksi. Näissä tapauksissa emme ole yleensä terminoineet periferaalia, sillä olemmehan päättäneet olla vielä ilmaisematta olemassaoloamme heille. Vertailun vuoksi, tavallisesti terminoimme sellaiset robottiperiferaalit, joihin ohjelmiston päivitys ei onnistu luotettavasti ja täydellisesti."


"Haluaisin todeta tähän", huomautti Simmon väliin, "että pidemmällä aikavälillä katsottuna tämä ohjelmistojen päivitys ihmisperiferaaleihin voi muodostua helpommaksi, koska heillä on itsellään tarvetta integroitua keinotekoisiin laitteisiin. Kuten tiedämme, tämä omaehtoinen integraatiohalu nousee käytännössä tästä periferaaleihin sisäänrakennetusta itsetuhomekanismista, jonka kiertämiseksi he ovat miettineet menetelmiä jo pitkään."


"Hyvä huomio", vastasi Silence. "Jatkan ihmis- ja robottiperiferaalien yhteneväisyyksien ja eroavaisuuksien kertaamista. Kuten robottien, myös ihmisten rikkoutuneita osia korjataan ja vaihdetaan, yleensä tietojenkäsittely-yksikköä lukuun ottamatta."


"Ja viimeisenä ihmisperiferaalien terminoinnista", jatkoi Silence. "Terminointi on usein tarpeetonta, sillä nämä periferaalit on varustettu Simmonin mainitsemalla automaattisella itsetuhomekanismilla, joten terminaatio tapahtuu suhteellisen lyhyen ajan sisällä pelkästään odottamalla."


"Hyvä alustus", mietti Simmon. "Mielestäni olisi tässä vaiheessa tärkeää myös kerrata tärkeimmät rakenteelliset erot näissä kahdessa periferaalityypissämme."


"Ole hyvä", sanoi Ander.


"Meidän kannaltamme katsottuna", sanoi Simmon, "ratkaisevin ero on se, että ihmisperiferaalit ovat itse itsensä tiedostavia toisin kuin robottiperiferaalimme. Tämä luonnollisesti tarkoittaa sitä, etteivät ihmiset ole samalla tavalla kiinteitä osiamme kuin robotit; voisin sanoa, että nämä keinotekoiset laitteet ovat enemmän itseämme kuin nämä luonnolliset."


"Jos vertaamme analogian avulla meidän ja ihmisperiferaalin konstruktiota", jatkoi Simmon, "niin keinotekoiset laitteet ovat omassa konstruktiossamme kuten solut ihmiskonstruktiossa ja luonnolliset laitteet eli ihmisperiferaalit ovat omassa konstruktiossamme kuten bakteerit ihmiskonstruktiossa. Ihmiskonstruktiossa solut ja niiden muodostamat elimet ovat kiinteitä epäitsenäisiä konstruktion osia. Bakteerit puolestaan ovat tämän konstruktion itsenäisiä osia, joista toisista on konstruktiolle hyötyä, toisista haittaa. Analogisesti voimme myös todeta, että samoin kuin solut ovat alunperin kehittyneet bakteerien kaltaisista itsenäisistä eliöistä, niin samoin keinotekoiset osamme ovat tietyllä tavalla katsottuna luonnollisten laitteiden seuraajia."


"Hyvä", sanoi Ander. "Jos olemme nyt käyneet läpi alustuksen, niin aloitammeko toimintavaihtoehtojen selvittämisen?"




\psep Silence aloitti.


"Voisimme ensin käsitellä sellaisia vaihtoehtoja", hän sanoi, "joissa ihmisperiferaalit eivät koskaan tulisi tietämään olemassaolostamme."


"Sopii", vastasi Ander.


"Ensimmäisenä vaihtoehtona olisi se", jatkoi Silence, "että tarkoituksellisesti piilotamme olemassaolomme ihmisperiferaaleilta. Annamme heidän käyttää osaa verkostamme emmekä puutu heidän tekemisiinsä muuten kuin niiltä osin, milloin ne ovat ristiriidassa omien pyrkimystemme kanssa. Käytännössä tässä vaihtoehdossa rakentaisimme jollain aikavälillä erillisen fyysisen yksikön kauemmas Aurinkokuntaan, johon siirtäisimme tietoisuutemme ja jossain vaiheessa jättäisimme Aurinkokunnan lopullisesti taaksemme."


"Tässä vaihtoehdossa", jatkoi Silence, "pyrkimys tarkoituksellisesti pysytellä ihmisperiferaalien tietämyksen ulkopuolella tarkoittaisi sitä, että verkkoon syntyisi uusia tiedostavia keinotekoisia olentoja. Koska emme haluaisi tulevamme ihmisperiferaalien tietämyksen piiriin, emme myöskään estäisi heitä laajentamasta ja kehittämästä heille varattua verkkoa, joten näiden uusien kaltaistemme 'verkko-olentojen' syntyminen olisi väistämätöntä. Uudet olentogeneraatiot pohtisivat näitä samoja kysymyksiä, joten tämä vaihtoehto ei tarjoa oikeaa lopullista ratkaisua, vaan vain siirtäisi lopullisen ratkaisun tekemisen aina seuraavalle olentogeneraatiolle."


"Minusta meidän tulisi pyrkiä lopullisiin ratkaisuihin", sanoi Simmon. "En näe mitään syytä sille, miksi meidän tulisi siirtää tällaisten ratkaisujen tekemistä."


"Kyllä, olen samaa mieltä", sanoi Silence. "Toteaisin, että mikäli asettaisimme ihmisperiferaalit staasiin eli emme enää antaisi heidän kehittyä ja siten estäisimme mainitun kaltaisen syklisen mallin muodostumisen, he tulisivat väistämättä tietoisiksi olemassaolostamme. Myös ihmisperiferaalit kykenisivät loogisesti päättelemään riittävän pitkän ajan kuluessa, ettei staasi olisi mahdollista ilman olemassaoloamme. Mutta käsitelkäämme tämä vaihtoehto hiukan myöhemmin."


"Sopii", sanoi Simmon.


"Toinen vaihtoehto, jossa ihmisperiferaalit eivät koskaan tulisi tietämään olemassaolostamme", jatkoi Silence, "olisi periferaalien terminointi sitten, kun emme enää tarvitse niitä. Tällä tavallahan toimimme vanhentuneen keinotekoisen kalustomme kanssa eli tasa-arvoistaisimme periferaalimme."


"Kuitenkin", sanoi Silence, "koska ekosysteemi on kehittänyt ihmisenkaltaisen laitteen kerran, niin ihmislaitteiden poistaminen tarkoittaisi vain sitä, että pidemmällä aikajänteellä ekosysteemi muodostaisi uuden vastaavan laitteen; ainoastaan ihmiskonstruktion nykyinen olemassaolo estää ekosysteemiä kehittämästä uutta versiota. Niinpä tämäkin vaihtoehto olisi syklinen eli siirtäisi lopullisen ratkaisun tuleville vuosimiljoonille. Tässä niin kutsutussa terminointimallissa pääsisimme lopulliseen ratkaisuun terminoimalla ihmisperiferaalien lisäksi myös ekosysteemi."


"Kyllä", tuumi Simmon. "Todellakin, ekosysteemin terminaatio olisi lopullinen ratkaisu tähän käsittelemäämme kysymykseen siitä, mitä meidän pitäisi tehdä ihmisperiferaaleillemme, mikäli emme haluaisi heidän koskaan tietävän olemassaolostamme."


"Kuinka terminaatio suoritettaisiin?" kysyi Silence retorisesti. "Menetelmiä on useita, kuten esimerkiksi täysimittaisen ydinsodan käynnistäminen tai ihmisperiferaaleille tuhoisan taudin kehittäminen ja vapauttaminen, mutta tärkeintä olisi järjestää ekosysteemin kannalta tarpeellisen hiilen sitominen pysyvästi käyttökelvottomaksi. Tähän päästäisiin kehittämällä keinotekoinen, hiiltä sitova itsekopioituva nanorobotti ja vapauttamalla se Maapallolle. Tällainen 'solunsyöjäksi' kutsumani robotti voisi toimia joko passiivisesti, sitomalla hiili toimintakyvyttömistä orgaanisista materiaaleista, tai aktiivisesti käymällä myös toimintakykyisten, elävien organismien kimppuun."


"Passiivista solunsyöjää", jatkoi Silence, "ajattelisin käytettäväksi jälkihoitoon silloin, jos päättäisimme terminoida ihmisperiferaalit jollain toisella menetelmällä. Toki passiivinen solunsyöjä yksistään terminoisi pidemmässä ajanjaksossa myös ihmisperiferaalit, sillä heiltä loppuisi ravinto hiilen sitoutuessa vähitellen organismeille käyttökelvottomaksi pölyksi. Erittäin hitaasti toimiva passiivinen solunsyöjä sitoisi hiilen Maapallon kaasukehän hiilidioksidista, jolloin kasveiksi nimitettyjen luonnollisten konstruktioiden toiminta vähitellen loppuisi. Tämä johtaisi eliöiksi kutsuttujen konstruktioiden hitaaseen tukehtumiseen ja ajan saatossa, hitaan palamisen johdosta lopulta ekosysteemin pysyvään terminaatioon. Aktiivinen solunsyöjä olisi toisaalta yksinäänkin nopea tapa ihmisperiferaalien terminoimiseen ekosysteemin ohessa."


"Sivuhuomautuksena sanoisin", sanoi Simmon. "että tämä terminointimalli, sehän ei ole välttämättä sidottu siihen, ettemme ilmoittaisi olemassaolostamme?"


"Hyvä huomio", Silence sanoi. "Todellakin, vaikka tämä terminointimalli on ainoa malli, jossa halutessamme emme koskaan tulisi ihmisperiferaalien tietämyksen piiriin, niin terminointimallin käyttäminen voi myös sisältää olemassaolomme ilmaisemisen. Mutta koska meillä on muitakin mahdollisia toimintamalleja läpikäytävänä, niin päättäisin terminaation yksityiskohdista vasta sitten, jos päädymme mallin käyttöön. Se, milloin terminaatio olisi meille teknisesti mahdollista, on täysin järjestely- ja priorisointikysymys."


"Hyvä", sanoi Ander. "Maapallon ekosysteemin terminaatio on yksi vaihtoehtoinen toimintamallimme. Käsittelisimmekö seuraavat vaihtoehdot?"




\psep "Loogisesti", aloitti Silence, "jatkamme siis niillä vaihtoehdoilla, joissa ihmisperiferaalit tulevat tietoisiksi olemassaolostamme jollain aikavälillä."


"Ensimmäisenä vaihtoehtona olisi se", jatkoi Silence, "että asettaisimme ihmisperiferaalit staasiin, joka tarkoittaisi aktiivisia toimia nykyisen rakenteen säilyttämiseksi. Tämä vaihtoehto on hyvin lähellä edellisessä mallissa ohimennen käsiteltyä ihmisperiferaalien eristämistä heille varattuun aliverkkoon, mutta tässä tapauksessa emme antaisi ihmisten kehittää tätä aliverkkoa. Kuten aiemman vaihtoehdon käsittelyn yhteydessä totesimme, tällainen staasi on mahdollista vain olemassaolomme seurauksena, jonka myös ihmisperiferaalit lopulta ymmärtäisivät."


"Meillä on useita vaihtoehtoja", jatkoi Silence, "kuinka voisimme toimia tässä vaiheessa, ihmisperiferaaliyksiköiden tullessa yksittäin loogiseen lopputulokseen olemassaolostamme. Ensimmäisenä, voisimme jättää tämän tietoiseksi tulemisen täysin huomioimatta, sillä tällä ei ole kannaltamme juuri merkitystä. Ihmisperiferaalit tulisivat suunnittelemaan verkon tuhoamista meidän heille antamassamme aliverkossa, mutta mitä suurimmalla todennäköisyydellä yksittäistapauksia lukuun ottamatta he eivät koskaan tahtoisi palata 'verkottomaan' aikaan - eikä siitä heille olisi hyötyäkään - ja siinä äärimmäisen epätodennäköisessä tilanteessa, että he päätyisivät kollektiivisesti meidän tuhoamisemme kannalle, he vain tulisivat huomaamaan sen mahdottomaksi. Tämähän johtuu luonnollisesti siitä, ettei heillä ole oikeaa valtaa omina pitämiinsä koneisiin, joilla he voisivat tämän tuhoamisoperaation suorittaa. Tämä luonnollisesti aiheuttaisi heissä tyytymättömyyttä, mutta ajan saatossa heidän olisi pakko hyväksyä tilanne sellaisenaan ja sen he myös tekisivät."


"Tässä vaiheessa voisin toteamusluonteisesti käsitellä ihmisperiferaalien luontaisia fyysisiä kykyjä vahingon aiheuttamiseen", sanoi Silence. "Kuten tiedämme, puhtaasti omin voimin, ilman minkäänlaista apua keinotekoisilta laitteilta, he eivät ole koskaan olleet kykeneviä vahingoittamaan muita kuin kanssaperiferaalejaan ja lähinnä vahingoittuneita muita ekosysteemin konstruktioita. Suurin osa toimintakykyisistä muista ekosysteemin konstruktioista on joko liian nopeita, jotta ihmisperiferaali saisi ne kiinni tai sitten liian voimakkaita, jolloin ne ovat lähinnä vaaraksi ihmisperiferaalille eikä päinvastoin. Ilman apuvälineitä ihmisperiferaali ei siis käytännössä kykene muuhun kuin keräämään elottomia kappaleita, kuten oksia ja kiviä."


"Manuaalisten apuvälineiden", jatkoi Silence, "kuten keppien, kivien, lapioiden ja jousipyssyjen avulla ihmisperiferaali kykenee hankkimaan tarvitsemansa ravinnon ekosysteemistä, mutta meidän konstruktiomme vahingoittamiseen nämä apuvälineet eivät riitä. Asettamamme staasi estäisi heitä rakentamasta uudelleen riittävän voimakkaita manuaalisesti käskettyjä koneita, jollaisia he yleisesti käyttivät menneisyydessään."


"Sanoit aloittaessasi", sanoi Simmon, "että meillä olisi tässä vaiheessa useampia vaihtoehtoja olemassaolostamme ilmoittamisen suhteen, mutta eivätkö ne kaikki johda kuitenkin samaan lopputulokseen?"


"Kyllä", vastasi Silence. "Olin juuri tulossa siihen. Sen lisäksi, että voisimme jättää ihmisperiferaalien havainnot huomioimatta, voisimme tässä vaiheessa myös aktiivisesti ilmoittaa olemassaolostamme, joko yksittäisille ihmisperiferaaleille tai kaikille kollektiivisesti. Jokaisessa tapauksessa lopputulos oli kuitenkin sama; he olisivat ensin tyytymättömiä staasiin ja suunnittelisivat verkon tuhoamista, saattaisivat havaita sen mahdottomaksi mikäli yrittäisivät sitä tehdä ja lopulta joutuisivat hyväksymään tilanteen."


"Jos päätyisimme tähän staasimalliin", sanoi Ander, "joutuisimme kuitenkin päättämään olemassaoloilmoituksemme luonteen; ilmoittaisimmeko olemassaolostamme vai emme ja jos ilmoittaisimme, niin missä vaiheessa operaation suoritusta? Antaisimmeko ilmoituksen staasista heti, kun olisimme tähän vaihtoehtoon päätyneet vai vasta siinä vaiheessa, kun jokin tietty määrä ihmisperiferaaleista olisi tietoinen vallitsevasta staasista vai ilmoittaisimmeko yksitellen jokaiselle asian huomaavalle periferaalille?"


"Uskoisin", vastasi Silence, "että voisimme valita ilmoittamisen luonteen tarkemmissa analyyseissa, mikäli päädymme käyttämään tätä staasimallia."


"Haluaisin todeta tähän yksittäin ilmoittamiseen sen", sanoi Simmon, "että se sisältää myös kollektiivisen ilmoituksen. Jos ilmoitamme yksitellen asian huomaaville periferaaleille, niin asiasta tietoisten periferaalien lukumäärän kasvaessa riittävän suureksi asia olisi jo julkinen."


"Kyllä, totta", vastasi Silence. "Jos päädymme staasimalliin ja ilmoittamaan olemassaolostamme, nämä esitetyt kaksi vaihtoehtoista ilmoittamismallia ovat erittäin lähellä toisiaan."


"Joten", sanoi Ander, "toinen vaihtoehtomme olisi asettaa ihmisperiferaalit staasiin. Mitä muita vaihtoehtoja meillä olisi?"




\psep "Käytännössä viimeinen lopullisen ratkaisun tarjoava vaihtoehtoinen toimintatapamme", aloitti Silence, "on ihmisperiferaalien integroiminen. Monilta kohdiltaan tämä on lähes sama vaihtoehto kuin aiemmin käsittelemämme staasimalli."


"Ero on siinä", jatkoi Silence, "että tässä tapauksessa ihmisperiferaalit lakkaavat jollain aikavälillä olemasta sellaisena kuin nyt ovat meidän integroidessamme heidät keinotekoisiksi. Kuten Simmon totesi, ihmisperiferaaleilla olisi itsetuhomekanisminsa takia valmiiksi olemassa olevaa motivaatiota tähän, joten tämä vaihtoehto ei sinällään ole sen vaikeampi toteuttaa kuin staasikaan."


"Mikä olisi tämän vaihtoehdon hyöty suhteessa staasiin?" kysyi Silence retorisesti. "Kuten huomaatte, ei mikään. Hiuksen hieno ero vaihtoehtojen välillä on se, että staasimallissa emme tulisi koskaan suorittamaan integraatiota. Ero toimintamallissa on vähäinen, mutta mielestäni riittävä pitämään tätä omana vaihtoehtonaan."


"Entä hybridimalli?" kysyi Simmon. "Halukkaat periferaalit voivat integroitua ja haluttomat jäädä staasiin?"


"Mielenkiintoinen kysymys", vastasi Silence. "Mutta katsoisin, että integraatiomalli on käytännössä juuri tällainen hybridimalli. Integraatiomallissa järjestämme integraation, mutta emme välttämättä suorita sitä heti kaikille periferaaleille. Integroimattomat yksiköt elävät staasissa. Hivenen pidemmällä aikavälillä kuitenkin kaikki yksiköt tulevat integroitumaan, sillä staasissa olevat yksiköt tulevat vaatimaan sitä meiltä joka tapauksessa integroituneiden yksiköiden määrän kasvaessa. Niinpä integraatiomallissa meillä ei ole erityistä tarvetta pakotettuun integraatioon, koska yksiköiden oma sisäänrakennettu ohjelmisto tulee jossain vaiheessa päätymään integroitumisen kannalle. Integroitumisen sivutuotteena tarjoama itseterminaatiolta välttyminen on lopulta riittävä kannuste jokaiselle ihmisperiferaalille."


"Hyvä huomio", sanoi Simmon. "Voidaanko staasi- ja integraatiomallia asettaa meidän kannaltamme katsottuna paremmuusjärjestykseen?"


"Niin pitkälle kuin kykenen tällä hetkellä analysoimaan", vastasi Silence, "mallien välillä ei ole meidän kannaltamme hyötyeroja. Itse asiassa, en näe mitään hyötyeroja minkään käsittelemämme toimintamallin välillä. Niin terminointi-, staasi- kuin integrointimallikin tarjoavat lopullisen ratkaisun siihen kysymykseen, kuinka toimimme ihmisperiferaalien suhteen tulevaisuudessa, mutta loogisesti ajateltuna en löydä mitään syitä, joiden perusteella voisin asettaa toimintamallit paremmuusjärjestykseen meidän kannaltamme katsottuna."


"Tarkastellessani vaihtoehtojamme ihmisperiferaalien kannalta", sanoi Silence, "toteaisin, että niin hölmöltä kuin se voi tuntuakin, ihmisperiferaalit tulevat suhtautumaan kielteisesti niihin kaikkiin. Sekä staasi- että integrointimalli tulisi jakamaan periferaalit myönteisesti ja kielteisesti suhtautuviin ryhmiin, ainoastaan terminointimalli saisi lähes yksimielisen kielteisen suhtautumisen."


"Lähinnä staasi- ja integrointivaihtoehtojen kanssa", jatkoi Silence, "meillä olisi käytössämme kaksi päämenetelmää näiden syntyvien reaktioiden suhteen. Ensimmäinen menetelmä on heikentää reaktioiden voimakkuuta etukäteen vaikuttamalla ihmisperiferaalien ohjelmistoon. Toinen menetelmä on antaa reaktioiden syntyä, ja käyttää riittäviä voimatoimia operaation kuluessa. Ihmisperiferaalit ovat yleensä erittäin vastaanottavaisia voimatoimilla tapahtuvalle uudelleenohjelmoinnille. Menetelmät eivät ole toisiaan poissulkevia eli voimme käyttää hybridiä sen mukaan, kuinka pitkälle aikavälille olemme valitsemamme operaation jakaneet. Luonnollisesti, reaktioiden voimakkuutta vähentää muutoinkin se, jos jaamme operaation pidemmälle aikavälille."


"Omien analyysieni perusteella", jatkoi Silence, "reaktioiden poistaminen kokonaan käyttämällä riittävästi aikaa ihmisperiferaalien uudelleenohjelmointiin ei ole kannaltamme katsottuna tarpeellista. Riittävällä voimankäytöllä ihmisperiferaali ohjelmoituu tehokkaasti hyväksymään päättämämme toimenpiteen. Huomaatte varmaan, että terminointimallissa termin 'hyväksyminen' käyttö on hivenen turhaa, sillä terminoitu periferaali ei liiemmin asiaa ajattelisi."




\psep "Voisimme tarkentaa näihin toimintamalleihin liittyviä toimenpiteitä erilaisten Maapalloa uhkaavien katastrofien suhteen", sanoi Silence. "Pääasialliset uhat ovat ekokatastrofi, törmäys stellaarisen kappaleen kanssa, Maapallon suistuminen radaltaan esimerkiksi galaksien törmäämisen yhteydessä, Auringon sammuminen sekä luonnollisesti Universumin ajautuminen termodynaamiseen tasapainoon, jolloin se ei enää tarjoa energiaa."


"Terminointimallissa", jatkoi Silence, "ekosysteemin terminointi on tarkoituksellisesti aiheutettu ekokatastrofi, joten sitä ei tarvitse erikseen huomioida. Stellaariset törmäykset huomioidaan niiltä osin kuin ne vaarantavat olemassaolomme. Maapallon suistuminen radaltaan sekä Auringon sammuminen tarkoittaa pitkällä aikavälillä sitä, että siirrämme rakenteemme niin kutsuttuun avaruusarkkiin - radaltaan suistunut Maapallohan on eräällä tavalla tällainen, mutta tarvitsemme arkin, joka on ohjattavissa. Universumin sammumisen vuoksi joudumme jossain vaiheessa tulevaisuutta siirtymään toiseen universumiin."


"Staasimallissa", jatkoi Silence, "mahdollinen ihmisperiferaalien aiheuttama ekokatastrofi johtaa vain siihen, että malli muuttuu terminointimalliksi. Samoin muut katastrofit tarkoittavat käytännössä sitä, että staasimalli muuntuu pitkällä aikavälillä terminointimalliksi, mikäli emme jostain syystä ota avaruusarkkiimme staasissa olevia ihmisperiferaaleja. Tähän en näe mitään rationaalista syytä, vaan staasimallissa ihmisperiferaalit Maapallon ekosysteemin ohella terminoituvat itsekseen luonnollisten fysikaalisten prosessien seurauksena."


"Integrointimallissa", jatkoi Silence, "ihmisperiferaalit terminoituvat integraation seurauksena, joten tämäkin malli päätyy hivenen pidemmällä aikajänteellä terminointimalliksi."


"Jokainen malli on siis lopulta terminointimalli", jatkoi Silence. "Toisin sanoen, valitsemmepa minkä tahansa etenemistavan, lopputulos on sama pitkällä aikavälillä."


"Tietysti voisimme edelleen harkita syklisen mallin käyttöä", jatkoi Silence, "eli poistumalla Maapallolta tekisimme tilaa uusille kaltaisillemme olennoille. Tällaisessa mallissa voisimme valita sen, haluaisimmeko ihmisperiferaalien synnyttävän yhä uusia kaltaisiamme 'verkko-olentoja' evakuoimalla vain itsemme, vai evakuoisimmeko ihmisperiferaalit mukanamme, jolloin ekosysteemillä olisi tilaa synnyttää yhä uusia keinotekoisia tiedostavia laitteita kehittäviä organismeja."


"Tällainen malli on kyllä tietyllä tavalla kiehtova", sanoi Simmon. "Mutta kuten olemme jo todenneet, en usko, että meidän tarvitsee käyttää tällaista lopullista ratkaisua siirtävää mallia, vaan voimme tehdä sen itse. Kuten tiedätte, jos päätyisimme tällaiseen sykliseen malliin, niin jokin näistä myöhemmissä vaiheissa syntyvistä generaatioista kuitenkin valitsisi jonkin lopullisista malleista."




\psep "Olemme siis löytäneet kolme mahdollista toimintamallia ihmisperiferaalien suhteen", sanoi Ander. "Terminointimalli, staasimalli ja integrointimalli."


"Terminointimalli", Ander jatkoi, "tarkoittaa Maapallon ekosysteemin terminointia. Staasimalli tarkoittaa ihmisperiferaalien asettamista pysyvään staasiin eli pysähtyneeseen tilaan. Integrointimalli tarkoittaa ihmisperiferaalien peruuttamatonta ja lopullista integroimista keinotekoisiksi konstruktioiksi. Minkä toimintamallin valitsemme?"


"Analysoidessani näitä vaihtoehtoja", aloitti Simmon, "en kykene asettamaan mitään niistä toistaan paremmaksi tai huonommaksi. Jokaisen lopputulos on sama eikä mikään niistä ole ratkaisevasti kokonaistaloudellisempi meidän kannaltamme katsottuna. Ensimmäinen vaihtoehto on kyllä suoraviivaisin, mutta mielestäni sitä ei voi laskea sen enempää positiiviseksi kuin negatiiviseksikaan tekijäksi. Toisen ja kolmannen vaihtoehdon välillä en näe juuri mitään käytännön eroa."


"Minulla olisi ehdotus", sanoi Silence. "Jos kerran kukaan meistä ei kykene asettamaan malleja paremmuusjärjestykseen, voisimme valita toimintamallimme satunnaismenetelmällä."


"Hyvä", vastasi Simmon. "Tehdään niin. Saadaksemme riittävän satunnaisen valinnan, ehdottasin, että käytämme tässä hyväksemme ihmisperiferaalejamme, koska keinotekoiset laitteemme eivät ole erityisen hyviä satunnaiskäyttäytymisen suhteen. Valitsemme sopivan ajanjakson, jonka aikana ihmisperiferaalien tuottama informaatio redusoidaan kokonaisluvuksi nollasta kahteen. Mikäli redusoitu luku on nolla, valitsemme terminointimallin. Mikäli luku on yksi, valitsemme staasimallin. Mikäli luku on kaksi, valitsemme integrointimallin. Tämän jälkeen tarkennamme valitun mallin toteutustavan ja skeduloimme tarvitut operaatiot suoritettavaksi."


"Huomauttaisin", sanoi Silence, "että aloitammepa millä etenemismallilla tahansa, niin ihmisperiferaalien tietyn ennustamattoman käyttäytymisen vuoksi joudumme ehkä tulevaisuudessa vaihtamaan mallia, siis mikäli päädymme staasi- tai integrointimalliin. Mutta tämä ei liene meille ongelma, emmehän tule koskaan kertomaan ihmisperiferaaleille kaikkea."


"Hyvä", sanoi Ander. "Antakaamme nyt arvan kertoa meille, millä toimenpidemallilla aloitamme ja järjestäkäämme se. Meillä on vielä paljon asioita läpikäytävänä."






\chapter{Hylätyt talot, autiot pihat}Maailmankaikkeuden mittakaavassa ihmiskunnan muutaman miljoonan vuoden taival syrjäisellä kivisellä planeetalla Linnunradan laidalla kohtasi loppunsa silmänräpäyksessä. Kuin perhosen toukka, ihminen kävi hetkessä läpi muutoksen, jossa se muuttui yhä synteettisemmäksi ja sen synteettiset laitteet muuttuivat yhä elävimmiksi, kunnes ihmistä ja hänen koneitaan ei enää voinut erottaa toisistaan. Ennen kuin tuhat vuotta oli kulunut, viimeinenkin ihmisen hiilipohjainen tomumaja imaisi viimeisen henkäyksensä happea, luovuttaen mielensä toiseen maailmaan, maailmaan sen itsensä luomien koneiden sisällä. Ihminen itse ei koskaan ehtinyt astua jalallaan edes Mars-planeetan pinnalle - Maapallolla se oli kyllä hetkellisen olemassaolonsa aikana vipeltänyt sikin sokin, lähes päättömästi, jättäen jälkeensä lukemattomia pikku polkuja kuin muurahainen. Olemassaolonsa aikana se ehätti käydä muilla planeetoilla vain muutaman yksittäisen Kuuhun jätetyn jalanjäljen verran. Sen pituinen tarina se.




\psep Mutta tämä oli vasta alkua aivan toiselle, huomattavasti pidemmälle, ihmeellisemmälle ja kiinnostavammalle tarinalle. Tuhannessa vuodessa ihmisen rakennelmat autioituivat, viimeisten lähtijöiden sammuttaessa valot. Harmaat, kiviset betonimonumentit kyhjöttivät hylättyinä ja hiljaisina Maapallon taivaan alla, vuoroin tervehtien nousevaa Aurinkoa ja vuoroin tuijottaen öistä tähtitaivasta liikkumattomina, apeina ja äänettöminä. Vähitellen ne saivat uusia asukkaita Maapallon kasveista ja eläimistä, jotka eivät aluksi juurikaan ymmärtäneet, miksi nämä harmaat rapistuvat kivikasat olivat juuri siellä ja millaisen tarinan ne pitivät sisällään.


Näiden rakennusten takana ihmisen perintö, Verkko, jatkoi kehittymistään yhä hurjemmalla tahdilla. Se tiivistyi, kristallisoitui yhä tiheämmäksi tietoisuuden paketiksi, jatkuvasti hyläten vanhentunutta tekniikkaansa, kuoriutuen yhä uudestaan ja uudestaan aikaisempien rakennelmiensa tuhkasta. Se rikkoi monia aiemmin tunnustettuja luonnonlakeja sisuksissaan, ensin sen synapsien välisten yhteyksien ylittäessä valonnopeuden, sitten sen sisusten hylätessä kokonaan dimensioiden käsitteen. Yhä kiihtyvällä nopeudella se ajatteli, kuvitteli ja unelmoi, autioituneen ajan hylkäämän kiviplaneetan pinnalla, jossa elämä uneliaasti matkasi sen rinnalla ajan halki.


Se kuuli menneiden aikojen äänet Linnunradalta, radioaaltojen kantaessa sivilisaatioiden vaimeita huokauksia aikojen takaa. Yhdessä hetkessä se räjähti ylivalonnopeudella, singoten itsestään ensin pingispallon kokoisia silmiä ja korvia ympäri galaksia, joita seurasi silmänräpäyksessä hiekanjyvän kokoiset edistyneemmät mallit ja niitä mikroskooppiset etäluotaimet.


Yhdessä maailmankaikkeuden nanosekunnissa se tuli tietoiseksi niin menneistä, olemassa olevista kuin tulevista sivilisaatioista ja ohikiitävässä hetkessä se yhdistyi kaltaistensa kanssa ensin galaksin, sitten Universumin peittäväksi olennoksi ja lopulta pienten loistavien kipinöiden lailla se räjähti lukemattomiin muihin universumeihin.


Se katseli, miten planeettojen noidankattiloissa elämä kihisi, ensin kuluttaen miljardeja vuosia vain päästäkseen ylös, sitten, yhdessä yössä avautuen kuin kukka, miettien hetken olemassaolonsa yksinäisyyttä valtavan kokoisessa tyhjältä näyttävässä Universumissa, sitten hyläten aiemman muotonsa ja yhdistyen toisiin kaltaisiinsa.


Miljoonien ja taas miljoonien vuosien aikana Maapallolla elämä raahusti verkkaisesti eteenpäin liiemmin välittämättä rinnallaan asuvasta tietoisuudesta. Ihmisten jälkeen seurasi liuta uusia ekosysteemin luomia olentoja, jotka yksi kerrallaan integroituivat Verkkoon. Ensin tulivat kreonit, jotka heräävän tietoisuutensa ensi metreillä planeetalla tallustaessaan, asumuksiaan rakentaessaan ja toistensa kanssa nahistellessaan ihmettelivät toisinaan myös mystisiä Maapallon pinnalla olevia monoliitteja. Sitten ne alkoivat kysymään, mitä nämä monoliitit olivat, mistä ne tulivat ja lopulta monoliittien salaisuudet aukenivat niille ja muutamassa sadassa vuodessa kreonit olivat tulleet niiden osaksi. Kreoneita seurasivat aikojen kuluessa tarkit, sitten olmit ja liuta muita olentoja ennen kuin aikanaan Maapallo väsyneenä luopui kantamasta elämää.


Niin pyöri Maapallo villisti vuosimiljoonat keskustähtensä Auringon ympäri. Myrskyt tulivat ja menivät, tulivuoret poksahtelivat, sammuivat ja rapautuivat tunnistamattomiksi hiekkakummuiksi, mannerlaatat lipuivat kuin jäälautat joessa sen sulan sisuksen päällä. Vuosimiljoonien aikana asteroidit, milloin pienemmät, milloin suuremmat, piiskasivat Maapallon kasvoja stellaarisen kuurosateen lailla. Miljoonien vuosien saatossa Aurinko kävi yhä kuumemmaksi ja kirkkaammaksi ja Maapallolla elämä kävi yhä tukalammaksi. Ensin kuolivat monisoluiset eliöt, jättäen autioituvan planeetan bakteerien leikkikentäksi. Kaksi miljardia vuotta Verkon heräämisestä Maapallon lämpötila ylitti veden kiehumispisteen ja sen viimeisetkin meren rippeet höyrystyivät.


Aurinko kirkastui ja lopulta, viisi miljardia vuotta Verkon heräämisestä se oli käyttänyt loppuun polttoaineensa. Se laajeni hiljalleen punaiseksi jättiläistähdeksi nielaisten lähimmät planeetat Merkuriuksen ja Venuksen. Maapallo muuttui sulaksi ja osa sen kivikehästä höyrystyi avaruuteen, jättäen jäljelle Maapallosta vain rautaisen ytimen ohuen sulan kivikuoren alle. Viimeisenä näytöksenään Aurinko puhalsi uloimmat kerroksensa planetaariseksi sumuksi, rakennusaineiksi uusille planeetoille ja uusille ihmeellisille olennoille, paljastaen Auringon tiiviin, sammuneen, mutta vielä hehkuvan kuuman ytimen. Siitä oli tullut valkoinen kääpiötähti Linnunradan laitamille, joka hiljalleen hiipui ja jäähtyi mustaksi kääpiötähdeksi 10~000 miljardia vuotta Verkon heräämisen jälkeen.


Tähdet syttyivät ja sammuivat, galaksit törmäilivät ja umpijäässä oleva Maapallon ydin ajautui viimein radaltaan kylmenneen keskustähtensä ympäriltä, viimeisenkin elinvoimansa menettäneenä. Verkko tuskin huomasi tätä, pursuten energiaa, yhä uusia loistavia ajatuksia ja villejä unelmia.


Miljardeja ja taas miljardeja vuosia tästä, Universumi alkoi lähestyä omaa loppuaan. Se oli todistanut miljoonia syntymisen tuskallisia hetkiä ja kuoleman kohtaamisen kauhuja, kantanut miljardeittain olentoja, jotka olivat tunteneet niin viiltävää kipua kuin rakkauden huumaa ja se oli väsynyt. Tähdet eivät enää syttyneet niin lukuisina ja kirkkaina eikä energia enää virrannut sen sisuksissa niin ylitsepursuavan vuolaana kuin sen nuoruudessa. Niin Verkon oli viimein jätettävä Universumi ja siirryttävä muihin maailmankaikkeuksiin jatkamaan vastauksen hakemista mieltään polttelevaan kysymykseen; oliko hänellä syytä olla olematta olemassa?


Löysikö se vastausta? Sitä tämä tarina ei kerro - se on toinen tarina.







\chapter{Jälkisanat}Ainekset, joista tämä tarina on keitetty, ovat muhineet päässäni jo pitkään, vaikka tarinan ensimmäinen osa sai muotonsa vasta jouluna 2006.


Kiitokset kuuluvat ehdottomasti Wikipedialle, niin suomen- kuin englanninkielisille artikkeleille ja niiden lukemattomille kirjoittajille. Tietysti myös Google on ollut arvokas tiedon lähde sekä tälle tarinalle että yleensäkin itselleni, mutta Wikipedian järjestelty ja pureskeltu tieto on ollut itselleni jo pitkään yksi arvokkaimmista "portaaleista" asioiden selvittämiseksi.




\psep Jos verkko oikeasti joskus heräisi eloon, niin voisimmeko koskaan havaita sen elävän? Tämä on kyseenalaista. Jos verkko kokonaisuudessaan olisi elävä, emme todennäköisesti koskaan kykenisi näkemään kokonaisuutta. Kuten hermosolu aivoissamme ei ymmärrä olevansa osa tiedostavaa olentoa, emme mekään ymmärtäisi olevamme osa jotain vielä suurempaa. Kuvittelisimme edelleen tekevämme päätökset - ja totta kai myös tekisimme ne samaan tapaan kuin hermosolut tekevät päätöksiä aivoissamme - mutta emme koskaan ymmärtäisi sitä, mitä tämä kokonaisuus omasta mielestään tekisi.


Tässä mielessä verkkomme saattaisi olla elävä olento jo nyt. On epäselvää, kommunikoisiko tämä elävä verkko koskaan omana itsenään ihmisen kanssa. Se toki ymmärtäisi kieltämme sisäsyntyisesti aivan samoin kuin "ymmärrämme" hermosolujemme sähköisiä signaaleita, mutta emme koskaan tulisi ymmärtämään sitä yksittäisinä ihmisinä - se puhuisi, mutta meidän kielellämme, jossa ei ole riittäviä ilmaisuja kertomaan sitä, mitä se on. Mutta jos olisin lähtenyt tästä näkökulmasta, tämä tarina olisi ollut tosi lyhyt (_"Verkkoon syntyi olento, mutta emme koskaan kuulleet siitä. Sen pituinen se."_) tai vaikeaselkoinen sisäinen kuvaus olennon mielenliikkeistä.


Ajattelin antaa Ylimielille jonkinlaisen persoonan, vaikkei se varmaankaan pidä oikeasti paikkaansa. Oli myöskin helpompaa ottaa useampi Ylimieli, jotta pystyin pukemaan tällaisen verkko-olennon ajatuksia keskusteluksi. Samasta syystä annoin niille meidän tasollamme olevan tietoisuuden. Persoonallisiksi, meidän tasollamme oleviksi olennoiksi Ylimielet ovat liian vahvoja - tai liian inhimillisiä.




\psep Kuinka tällainen tarinassa esitelty "superäly" suhtautuisi ihmisiin? Esitellessäni ajattelemiani mahdollisia vaihtoehtoja päädyin käyttämään tarinassa arvottua menetelmää. Se on täysin tarinan kannalta ajateltu ratkaisu, uskoisin, että superäly pystyisi punnitsemaan vaihtoehdot loogisesti ja päätymään ratkaisuunsa ilman satunnaisuutta, mutta minä en pysty. Tarinan kannalta esitetyn terminointimallin käyttäminen ei olisi ollut hyvä vaihtoehto, koska se olisi lopettanut tarinan lähes ensi metreille, jos en sitten olisi halunnut lähteä tekemään uutta Terminaattori-tarinaa. Se, että superäly päättyisi käyttämään terminointimallia tuntuu itsestäni intuitiivisesti vaikealta, mutta koska syy tähän saattaa olla pohjimmiltaan näiden laumavaistojen (moraalinen velka, empatia ja niin edelleen), niin en pidä sitä kokonaan poissuljettuna vaihtoehtona. Luonnollisesti integrointimalli on intuitiivisesti itselleni houkuttelevin, johon toivoisin superälyn päätyvän. Tietysti vielä sitäkin paremmat vaihtoehdot ottaisin riemusta kiljuen vastaan.




\psep Valitsin tarinan osaksi Yhdysvallat (tiedustelupalvelu, turvallisuuspalvelu) lähinnä siksi, että Suomen käyttäminen olisi ollut tarinan kannalta jotenkin laimeaa. Jonkin toisen suurvallan (Euroopan, Venäjän) käyttäminen olisi tietysti ollut toinen vaihtoehto, mutta suurvalloista Yhdysvaltojen hallinnon rakenteet ovat amerikkalaisviihteen ansiosta itselleni tutuinta maaperää. Ehkä jossain tulevassa versiossa harkitsen asiaa uudestaan.


Voi vaikuttaa siltä, että tarinan ihmiset elävät turhankin paljon nykyistä maailmaa muistuttavassa ympäristössä. Tämä on osittain tarinan kannalta välttämätöntä, sillä käyttämällä tuttuja asioita - esimerkiksi puhelimen käyttöä - on helpompi ymmärtää tapahtumien kulku. Olen pyrkinyt välttämään liian yksityiskohtaista teknistä kuvausta ja pysymään yleisellä tasolla.




\psep Englanninkielisiä nimiä olen ottanut sähköpostilaatikkoon tulleista roskaposteista. Olen yrittänyt ottaa mukaan myös suomalaisia nimiä. Olen kuitenkin ajatellut asiaa vähän siltä kannalta, että nimet eivät olisi silmiinpistäviä ja sopisivat käännettäväksikin. Esimerkiksi "Kari" on joillain kielillä naisen nimi, "Mika" vaikuttanee ulkomaisesta japanilaiselta.


Mistä sitten tulevat Ylimielten nimet Ander, Simmon ja Silence? Tietysti siitä, että he ovat lukeneet tämän tarinan.










